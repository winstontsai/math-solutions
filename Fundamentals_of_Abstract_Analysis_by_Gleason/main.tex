\documentclass{report}
\usepackage[utf8]{inputenc}
\usepackage[left=1in,right=1in,top=1in,bottom=1in]{geometry}
\usepackage{fancyhdr}
\pagestyle{fancy}
\fancyhf{}
\fancyhead[L]{\leftmark}
\fancyhead[R]{\rightmark}
\fancyfoot[C]{\thepage}
\usepackage{hyperref}
\hypersetup{
    colorlinks=true, %set true if you want colored links
    linktoc=all,     %set to all if you want both sections and subsections linked
    linkcolor=blue,  %choose some color if you want links to stand out
}
\usepackage{graphicx}
\graphicspath{ {./images/} }
\usepackage{enumitem}
\setlist{nosep}
\usepackage{tasks}
\usepackage{parskip,xcolor,array,etoolbox,multicol,mathdots}
\usepackage{my_macros}

% @@@@@@@@@@@@@@@@@@@@@@@@@@@@@@@@@@@@@@@@@@@@@@@@@@@@@@@@@@@@@@@@@@@@@@@@@@@@@@@@@
% MACROS
% @@@@@@@@@@@@@@@@@@@@@@@@@@@@@@@@@@@@@@@@@@@@@@@@@@@@@@@@@@@@@@@@@@@@@@@@@@@@@@@@@
\renewcommand*{\thesection}{\thechapter-\arabic{section}}
\renewcommand*{\land}{\text{ and }}
\renewcommand*{\lor}{\text{ or }}
\renewcommand*{\lnot}{\text{not }}
\newcommand*{\lnotd}{\text{not-}}
\newcommand*{\comp}[1]{\widetilde{#1}}
\newcommand*{\Erel}{\mathrel{E}}

\newcounter{magicrownumbers}
\preto\array{\setcounter{magicrownumbers}{0}}
\preto\tabular{\setcounter{magicrownumbers}{0}}
\def\rownumber{}

% @@@@@@@@@@@@@@@@@@@@@@@@@@@@@@@@@@@@@@@@@@@@@@@@@@@@@@@@@@@@@@@@@@@@@@@@@@@@@@@@@
% TITLE PAGE
% @@@@@@@@@@@@@@@@@@@@@@@@@@@@@@@@@@@@@@@@@@@@@@@@@@@@@@@@@@@@@@@@@@@@@@@@@@@@@@@@@
\begin{document}
\begin{titlepage}
  \begin{center}
    \vspace*{1cm}
    \LARGE
    Solutions to

    \vspace{0.5cm}

    \Huge
    \textbf{Fundamentals of Abstract Analysis}

    \huge
    Book Subtitle

    \LARGE
    \textbf{by Andrew M. Gleason}

    \vspace{0.2cm}

    \large
    ISBN 9780867202090

    \vfill

    \LARGE{Winston Tsai}
  \end{center}
\end{titlepage}
\newpage

\section*{Preface}
Hints and solutions to most of the exercises are provided at the end of the book.

\newpage
\tableofcontents
\newpage

% @@@@@@@@@@@@@@@@@@@@@@@@@@@@@@@@@@@@@@@@@@@@@@@@@@@@@@@@@@@@@@@@@@@@@@@@@@@@@@@@@
% CHAPTER 1
% @@@@@@@@@@@@@@@@@@@@@@@@@@@@@@@@@@@@@@@@@@@@@@@@@@@@@@@@@@@@@@@@@@@@@@@@@@@@@@@@@
\chapter{Sets}

\section{The notion of set}

\begin{exercise}
Which of the following are bona fide sets?
\begin{itemize}
    \item The set of all even integers.
    \item The set of all large integers.
    \item The set of all irrational numbers which are square roots of integers.
    \item The set interesting numbers.
    \item The set of points in a given euclidean plane.
    \item The set of points in a given noneuclidean plane.
\end{itemize}
\end{exercise}

\begin{solution}
Per Definition 1-1.1, a set must be sufficiently well-defined. 
The phrases "large integer" and "interesting number" are much too vague. The others are all right.
\end{solution}


\begin{exercise}
Was the population of the United States at noon (EST) on January 1, 1960, an odd
or an even number? Is this question meaningful? Is the set of inhabitants of the United
States at that time a well-defined set?
\end{exercise}

\begin{solution}
We have no definition of ``inhabitant of the United States'' complete enough to
cover unambiguously all cases which arise in practice. Hence the set of all such
inhabitants is not well-defined in a mathematical sense.
\end{solution}


\section{Equality}
\section{Parentheses}
\section{Membership}
\section{The empty set}
\section{The list notation}

\begin{exercise}
List the sixteen sets that can be formed from $\eset$, using braces not more than three deep.
\end{exercise}

\begin{solution}
Here are the sets with braces at most one deep.
\begin{itemize}
    \item $\eset$.
    \item $\set{\eset}$.
\end{itemize}

Here are the sets with braces at most two deep.
\begin{itemize}
    \item $\eset$.
    \item $\set{\eset}$.
    \item $\set{\eset, \set{\eset}}$ and $\set{\set{\eset}}$.
\end{itemize}

Here are the sets with braces at most three deep.
\begin{itemize}
    \item $\eset$.
    \item $\set{\eset}$.
    \item $\set{\eset, \set{\eset}}$ and $\set{\set{\eset}}$.
    \item $\set{\set{\eset, \set{\eset}}}$ and $\set{\set{\set{\eset}}}$ and
    $\set{\set{\eset, \set{\eset}}, \set{\set{\eset}}}$.
    \item $\set{\set{\eset, \set{\eset}}, \set{\eset}}$ and $\set{\set{\set{\eset}}, \set{\eset}}$ and
    $\set{\set{\eset, \set{\eset}}, \set{\set{\eset}}, \set{\eset}}$.
    \item $\set{\set{\eset, \set{\eset}}, \eset}$ and $\set{\set{\set{\eset}}, \eset}$ and
    $\set{\set{\eset, \set{\eset}}, \set{\set{\eset}}, \eset}$.
    \item $\set{\set{\eset, \set{\eset}}, \set{\eset}, \eset}$ and $\set{\set{\set{\eset}}, \set{\eset}, \eset}$ and
    $\set{\set{\eset, \set{\eset}}, \set{\set{\eset}}, \set{\eset}, \eset}$.
\end{itemize}

\end{solution}
\section{Set inclusion}


\chapter{Logic}
\section{Propositions and logical connectives}
\section{Tautologies}
\section{The conditional}
\begin{exercise}
Assume that the following propositions are true: $x\in A$, $x\nin B$, $x\in C$, $y\in A$,
$y\in B$, $y\nin C$, $z\nin A$, $z\nin B$, and $z\in C$. Which of the following are true?
\begin{enumerate}[label=(\alph*)]
    \item $y\nin B\lor z\in A$.
    \item $x\nin C\land y\in B$.
    \item $x\in C$ implies $z\in A$.
    \item $(x\in A\lor y\in B)$ implies $z\nin C$.
    \item $x\nin A$ if and only if $y\in C$.
    \item If $x\in A$ implies $z\in C$, then $y\nin A$ implies $z\in B$.
    \item If $z\in A$ implies $x\in B$, then $y\in B$.
    \item If $x\nin B$, then $y\in C$, $z\in B$, and $x\nin A$.
    \item $x\nin A$ only if $y\nin B$.
    \item $z\in C$ is a necessary condition for $y\in C$.
    \item $z\in C$ is a sufficient condition for $x\in A$.
    \item In order that $z\nin C$, it is necessary and sufficient that $x\in A$ imply $y\in B$.
\end{enumerate}
\end{exercise}

\begin{solution}
We can replace each basic (non)membership proposition with $T$ or $F$ based on
the propositions assumed to be true. That will make it easier to determine whether
each proposition as a whole is true.
\begin{enumerate}[label=(\alph*)]
    \item $F\lor F$.
    \item $F\land T$.
    \item $T$ implies $F$.
    \item ($T\lor T$) implies $F$.
    \item $F$ if and only if $F$.
    \item If $T$ implies $T$, then $F$ implies $F$.
    \item If $F$ implies $F$, then $T$.
    \item If $T$, then $F$, $F$, and $F$.
    \item $F$ only if $F$.
    \item $T$ is a necessary condition for $F$.
    \item $T$ is a sufficient condition for $T$.
    \item In order that $F$, it is necessary and sufficient that $T$ imply $T$.
\end{enumerate}
The propositions (e), (f), (g), (i), (j), (k), and (l) are true.
\end{solution}

\begin{exercise}
Suppose it is known that one of the following cases is true:

Case I. $x\in A$, $x\in B$, $x\nin C$, $y\in A$, $y\nin B$, and $y\in C$;

Case II. $x\nin A$, $x\in B$, $x\in C$, $y\nin A$, $y\nin B$, and $y\in C$.

Which of the following propositions are certainly true? Which are certainly false?

\begin{enumerate}[label=(\alph*)]
    \item $x\in C$ if and only if $y\in C$.
    \item ($x\in B\land y\in A$) if and only if ($x\in A\lor y\in B$).
    \item If $x\in A\land y\in C$, then $x\in C\lor y\in A$.
    \item ($x\in B\land y\in B$) or ($x\in C\land y\in C$).
    \item not-(if $x\in A$, then $y\in B$) implies ($x\in C\land y\in B$).
\end{enumerate}
\end{exercise}

\begin{solution}
This is like the previous exercise, except we need to determine which of the propositions is true
for each of the two cases. Then the propositions which are certainly true are those that are true for both cases,
and the propositions which are certainly false are those that are false for both cases.

For Case I, the propositions simplify to the following.

\begin{enumerate}[label=(\alph*)]
    \item $F$ if and only if $T$.
    \item ($T\land T$) if and only if ($T\lor F$).
    \item If $T\land T$, then $F\lor T$.
    \item ($T\land F$) or ($F\land T$).
    \item not-(if $T$, then $F$) implies ($F\land F$).
\end{enumerate}

For Case I, propositions (b) and (c) are true.

For Case II, the propositions simplify to the following.

\begin{enumerate}[label=(\alph*)]
    \item $T$ if and only if $T$.
    \item ($T\land F$) if and only if ($F\lor F$).
    \item If $F\land T$, then $T\lor F$.
    \item ($T\land F$) or ($T\land T$).
    \item not-(if $F$, then $F$) implies ($T\land F$).
\end{enumerate}

For Case II, propositions (a), (b), (c), (d), and (e) are true.

Therefore, knowing that Case I or Case II is true,
the propositions (b) and (c) are certainly true. None of the others are certainly false.
\end{solution}

\begin{exercise}
Suppose that the following propositions are true:
\begin{align*}
    &\text{if $x\in A\land y\in B$, then $y\in A$;}\\
    &\text{if $x\nin A\lor y\in A$, then $y\nin B$.}
\end{align*}
What simple conclusion can we draw?
\end{exercise}

\begin{solution}
We can conclude $y\nin B$. For suppose that $y\in B$. Either $x\in A\lor x\nin A$.
If $x\in A$, then the first hypothesis gives $y\in A$, from which the second hypothesis gives
$y\nin B$.
If $x\nin A$, then the second hypothesis again gives $y\nin B$.
In both cases we have a contradiction.
Therefore $y\nin B$.

In fact, if we assume $y\nin B$, then the first hypothesis is true because the antecedent is false,
and the second hypothesis is true because the consequent is true.
Therefore $y\nin B$ is true if and only if both hypotheses are true, so $y\nin B$ is the
best conclusion we can draw.

This result can also be derived with a truth table.
If $p$ stands for $x\in A$, $q$ for $y\in B$, and $r$ for $y\in A$, the hypotheses are
$(p\land q)\implies r$ and $(\lnot p\lor r)\implies\lnot q$.
We then have the following truth table.
\[
\begin{array}{ @{\makebox[2em][c]{\rownumber\space}} ccccc } 
p & q & r & (p\land q)\implies r & (\lnot p\lor r)\implies\lnot q 
\gdef\rownumber{\stepcounter{magicrownumbers}(\arabic{magicrownumbers})} \\
\hline
T & T & T & T & F \\ 
T & T & F & F & T \\ 
T & F & T & T & T \\ 
T & F & F & T & T \\ 
F & T & T & T & F \\ 
F & T & F & T & F \\ 
F & F & T & T & T \\ 
F & F & F & T & T \\ 
\end{array}
\]
If we consider only the rows for which both hypotheses are true, we are left with
exactly the rows for which $q$ is false: (3), (4), (7), and (8).
\end{solution}


\section{Propositional schemes and quantifiers}

\begin{exercise}
Express in terms of the ordinary quantifiers:
\begin{enumerate}[label=(\alph*)]
    \item There are at most two $x$'s such that $P(x)$.
    \item There are at least two $x$'s such that $P(x)$.
\end{enumerate}
\end{exercise}

\begin{solution}
\begin{enumerate}[label=(\alph*)]
    \item $(\forall x,y,z)\prn{P(x)\land P(y)\land P(z)}\implies (x=y\lor x=z\lor y=z)$
    \item $(\exists x,y)\prn{x\neq y \land P(x)\land P(y)}$
\end{enumerate}
\end{solution}

\begin{exercise}
Let $\phi(x, y)$ mean $x$ is a parent of $y$. Let $M$ be the set of all men (living or dead)
and $W$ be the set of all women.

The proposition, \textit{$a$ is the mother of $b$}, can be written \textit{$a\in W$ and $\phi(a, b)$}, while \textit{$a$ is the
brother (possibly half-brother) of $b$} can be written \textit{$a\in M$ and $a\neq b$ and $(\exists x) (\phi(x, a)$ and
$\phi(x, b))$}.

Because ordinary language is less precise than the language of quantified statements
and because there are different ways the biological facts can enter the interpretations of
the problems, the answers to the following problems are not all unique.

Express with quantifiers
\begin{enumerate}[label=(\alph*)]
    \item $a$ is the grandfather of $b$.
    \item $a$ is the grandson of $b$.
    \item $a$ is the aunt of $b$. 
    \item $a$ and $b$ are sisters.
    \item aand dare first cousins. 
    \item $a$ has neither brothers nor sisters.
    \item $a$ is the full brother of $b$.
    \item Every person has a father.
    \item Every person has at most one father.
    \item Every person has exactly two parents.
    \item Some person has at least two children.
    \item Every person has a grandfather.
    \item No one is his own grandfather.
\end{enumerate}

Let $\psi(x, y)$ mean \textit{$x$ is the ancestor of $y$}. Express $\phi(a,b)$ in terms of $\psi(x,y)$ and
quantifiers. Can you express $\psi(a, b)$ in terms of $\phi(x,y)$ and quantifiers?
\end{exercise}

\begin{solution}
\begin{enumerate}[label=(\alph*)]
    \item $a\in M\land (\exists x)\prn{\phi(a,x)\land \phi(x, b)}$
    \item $a\in M\land (\exists x)\prn{\phi(b,x)\land \phi(x, a)}$
    \item $a\in W\land (\exists x, y)\prn{\phi(x,b)\land \phi(y, x)\land \phi(y,a)\land x\neq a}$
    \item $a,b\in W\land a\neq b\land (\exists x)\prn{\phi(x, a)\land \phi(x, b)}$
    \item $a\neq b\land (\exists x, y, z)\prn{x\neq y\land \phi(x,a)\land \phi(y,b)\land \phi(z, x)\land \phi(z,y)}$
    \item $(\forall x)\prn{(\exists y)(\phi(y,x)\land \phi(y, a))\implies x = a}$
    \item $a\in M\land a\neq b\land (\exists x,y)(x\neq y\land \phi(x,a)\land \phi(x,b)\land \phi(y,a)\land \phi(y,b))$
    \item $(\forall x)(\exists y\in M)\phi(y, x)$
    \item $(\forall x)(\forall y,z\in M)(\phi(y, x)\land \phi(z,x) \implies y = z)$
    \item $(\forall x)(\exists y,z)(y\neq z\land \phi(y, x)\land \phi(z,x)
    \land (\forall w)(\phi(w,x)\implies (w=y\lor w=z)))$
    \item $(\exists x,y,z)(y\neq z\land \phi(x,y)\land \phi(x,z))$
    \item $(\forall x)(\exists z\in M)(\exists y)(\phi(y, x)\land \phi(z,y))$
    \item $(\forall x,y,z)((\phi(y, x)\land \phi(z,y)\land z\in M) \implies x\neq z)$
\end{enumerate}

$\phi(a,b)$ can be expressed as $\psi(a,b)\land \lnotd(\exists x)(\psi(a,x)\land \psi(x,b))$.

It is impossible to express $\psi(a, b)$ in terms of $\phi(x, y)$ and quantifiers.
\end{solution}


\section{Proof and inference}

\begin{exercise}
Refer to Exercise 2, Section 2-4. Prove formally that (h) implies (l).
\end{exercise}

\begin{solution}
The hypothesis is $(\forall x)(\exists y\in M)\phi(y, x)$.

\begin{enumerate}[label=(\roman*)]
    \item Let $a$ be any person.
    \item \quad $(\exists y\in M)\phi(y, a)$ \hfill by hypothesis;
    \item \quad choose $b$ so that $\phi(b,a)$ \hfill by (ii);
    \item \quad $(\exists y\in M)\phi(y, b)$ \hfill by hypothesis;
    \item \quad choose $c$ so that $c\in M$ and $\phi(c,b)$ \hfill by (iv);
    \item \quad $c\in M\land (\exists y)(\phi(c, y)\land \phi(y, a))$ \hfill by (iii) and (v);
    \item \quad $(\exists z\in M)(\exists y)(\phi(z,y)\land \phi(y, a))$ \hfill by (vi);
    \item $(\forall x)(\exists z\in M)(\exists y)(\phi(z,y)\land \phi(y, x))$ \hfill by (i) through (vii).
\end{enumerate}
\end{solution}

\section{Set formation}

\begin{exercise}
Which of the following sets are the same? What inclusion relations hold among
these sets? (All variables have domain $\R$.)

\begin{tasks}[label=](2)
    \task $A_1=\setb{x}{0 < x <2}$
    \task $A_2=\setb{x}{\abs{x} < 1}$
    \task $A_3=\setb{x}{x < x^2}$
    \task $A_4=\setb{x}{(\exists y)x=y^2}$
    \task $A_5=\setb{x}{(\exists y)x=1/y}$
    \task $A_6=\setb{x}{(\exists y)x=3y}$
    \task $A_7=\setb{x}{(\exists y,z)x=y^2 \land y+z^2 < 1}$
    \task $A_8=\setb{x}{(\forall y)\abs{x-y} = x - y}$
    \task $A_9=\setb{x}{(\forall y)\abs{x-y} = \abs{y} - \abs{x}}$
    \task $A_{10}=\setb{x}{(\exists y)y^2 + xy + x = 1}$
\end{tasks}
\end{exercise}

\begin{solution}
We have the following equalities:
\begin{tasks}[label=](2)
    \task $A_1=\setb{x}{0 < x <2}$
    \task $A_2=\setb{x}{-1 < x < 1}$
    \task $A_3=\setb{x}{x < 0}\cup \setb{x}{x > 1}$
    \task $A_4=\setb{x}{x\geq 0}$
    \task $A_5=\setb{x}{x\neq 0}$
    \task $A_6=\R$
    \task $A_7=\setb{x}{x\geq 0}$
    \task $A_8=\eset$
    \task $A_9=\set{0}$
    \task $A_{10}=\R$
\end{tasks}

So we have the following set equalities and inclusions:
\begin{itemize}
    \item $\eset = A_8 \sss A_9 \sss A_1 \sss A_4 = A_7 \sss A_6 = A_{10} = \R$
    \item $\eset \sss A_3 \sss A_5 \sss \R$
    \item $A_9 \sss A_2 \sss \R$
\end{itemize}
\end{solution}




\section{The set-theoretic paradoxes}
\section{Dummy variables}



\chapter{The Set-Theoretic Machinery}
\section{Binary set combinations}
Prove the following set-theoretic identities. Do each one twice, once using Venn
diagrams and once by formal manipulation using the identities in the text.

\begin{exercise}
$(A\cup B\cup C)\cap(B\cup D) = (A\cap D)\cup B\cup (C\cap D)$.

Write and prove the dual identity also.
\end{exercise}

\begin{solution}
Applying the distributive law twice to the left-hand side gives
\[
    (A\cup B\cup C)\cap(B\cup D)
    = (A\cap B)\cup (A\cap D)\cup (B\cap B)\cup (B\cap D)\cup (C\cap B)\cup (C\cap D).
\]
All intersections of involving $B$ are absorbed by by term $B\cap B=B$, so the result is
$(A\cap D)\cup B \cup (C\cap D)$.
\end{solution}

\begin{exercise}
$(A\cup B)\cap(B\cup C)\cap(C\cup A) = (A\cap B)\cup(B\cap C)\cup(C\cap A)$.

Show that this identity is its own dual.
\end{exercise}

\begin{solution}
Applying the distributive law three times to the left-hand side gives
\[
    (A\cap B\cap C)\cup(A\cap B\cap A)\cup(A\cap C\cap C)\cup(A\cap C\cap A)
    \cup(B\cap B\cap C)\cup(B\cap B\cap A)\cup(B\cap C\cap C)\cup(B\cap C\cap A).
\]
Removing duplicate sets from each term gives
\[
    (A\cap B\cap C)\cup(A\cap B)\cup(A\cap C)\cup(A\cap C)\cup(B\cap C)\cup(B\cap A)
    \cup(B\cap C)\cup(B\cap C\cap A).
\]
Removing duplicate terms gives
\[(A\cap B\cap C)\cup(A\cap B)\cup(A\cap C)\cup(B\cap C).\]
Finally, $A\cap B\cap C$ is absorbed by $A\cap B$ to give
\[(A\cap B)\cup(A\cap C)\cup(B\cap C) = (A\cap B)\cup(B\cap C)\cup(C\cap A).\]

To find the dual identity, interchange the $\cup$ and $\cap$ signs. That gives
the same equation but reversed, so this identity is its own dual.
\end{solution}

\begin{exercise}
$\prn{A\cup B\cup C}^{\sim}= \widetilde{A}\cap \widetilde{B}\cap \widetilde{C}$.

This is an extended form of (19). To give a formal proof from the identities in the text,
one must agree on a definite interpretation of $A\cup B\cup C$ in terms of the binary
combination $\cup$, and similarly for $\cap$.
\end{exercise}

\begin{solution} 
Let us agree that $A\cup B\cup C$ means $(A\cup B)\cup C$ and similarly for $\cap$.
We compute
\[
\begin{aligned}
    \prn{A\cup B\cup C}^{\sim} &= \prn{(A\cup B)\cup C}^{\sim} \\
    &= \prn{A\cup B}^{\sim} \cap \widetilde{C} \\
    &= \prn{\widetilde{A}\cap \widetilde{B}} \cap \widetilde{C} \\
    &= \widetilde{A}\cap \widetilde{B} \cap \widetilde{C}.
\end{aligned}
\]
\end{solution}

\begin{exercise} \label{ex:test}
$(A - B)\cup C = (A\cup C) - (B-C)$.
\end{exercise}

\begin{solution}
We compute
\[
\begin{aligned}
    (A - B)\cup C &= (A\cap \widetilde{B})\cup C \\
    &= (A\cup C)\cap(\widetilde{B}\cup C) \\
    &= (A\cup C)\cap\prn{B\cap \widetilde{C}}^{\sim} \\
    &= (A\cup C)-\prn{B - C}.
\end{aligned}
\]
\end{solution}

\begin{exercise}
If $A\cap B\cap C = \eset$, then $(A-B)\cup(B-C)\cup(C-A)=A\cup B\cup C$.
\end{exercise}

\begin{solution}
We have
\[
\begin{aligned}
    (A-B)\cup(B-C)\cup(C-A) &= (A\cap \comp{B})\cup(B\cap \comp{C})\cup(C\cap\comp{A})\\
    &= \prn{(A\cup B) \cap (A\cup\comp C)\cap (\comp B \cup B)\cap (\comp B \cup \comp C)}\cup(C\cap\comp A)\\
    &= \prn{(A\cup B\cup C) \cap (A\cup\comp C\cup C)\cap (\comp B \cup B\cup C)\cap (\comp B \cup \comp C\cup C)}\\
    & \quad \cap\prn{(A\cup B\cup\comp A) \cap (A\cup\comp C\cup\comp A)\cap (\comp B \cup B\cup\comp A)\cap (\comp B \cup \comp C\cup\comp A)}\\
\end{aligned}
\]
Since $A\cup\comp A = B\cup\comp B =C\cup\comp C= U$, the universal set, this reduces to
$(A\cup B\cup C) \cap (\comp B \cup \comp C\cup\comp A)$.
If $A\cap B\cap C=\eset$,  then $\comp A\cup\comp B\cup\comp C = \prn{A\cap B\cap C}^{\sim}=\comp\eset=U$,
so $(A\cup B\cup C) \cap (\comp B \cup \comp C\cup\comp A) = (A\cup B\cup C)\cap U = A\cup B\cup C$, as desired.
\end{solution}

\begin{exercise}
$(A-B)\cup(B-C)\cup(C-A)=(A-C)\cup(C-B)\cup(B-A)$.
\end{exercise}

\begin{solution}
In the previous exercise we showed that
$(A-B)\cup(B-C)\cup(C-A)=(A\cup B\cup C) \cap (\comp A\cup \comp B \cup \comp C)$.
By interchanging $B$ and $C$, we also get
$(A-C)\cup(C-B)\cup(B-A)=(A\cup C\cup B) \cap (\comp A\cup \comp C \cup \comp B)$.
The right-hand side of both equations are equal, and the result follows.
\end{solution}

\begin{exercise}
$\prn{A\cup (\comp B\cap C)}^{\sim} = (B-A)\cup(\widetilde A-C)$.
\end{exercise}

\begin{solution}
We have $\prn{A\cup (\comp B\cap C)}^{\sim} = \comp A \cap \prn{\comp B \cap C}^{\sim}
=\comp A \cap (B \cup \comp C)
=(\comp A \cap B)\cup(\comp A \cap \comp C)
=(B-A)\cup(\comp A - C)$.
\end{solution}

\begin{exercise}
The \textit{symmetric difference} of two sets, often denoted by $A\oplus B$, is defined
as $(A-B)\cup(B-A)$. Show that for all $A$, $B$, $C$,
\begin{gather*}
    A\oplus A=\eset, \qquad A\oplus\eset = A, \qquad A\oplus B = B\oplus A,\\
    A\oplus(B\oplus C) = (A\oplus B)\oplus C, \qquad A\cap(B\oplus C)=(A\cap B)\oplus(A\cap C).
\end{gather*}
\end{exercise}

\begin{solution}
The first three identifies are straightforward.
We have $A\oplus A = (A-A)\cup(A-A)=\eset\cup\eset=\eset$.
We have $A\oplus\eset = (A-\eset)\cup(\eset-A)=A\cup\eset=A$.
We have $A\oplus B=(A-B)\cup(B-A)=(B-A)\cup(A-B)=B\oplus A$.

Now note that for any $A,B$, we have $A\oplus B=(A\cap\comp B)\cup (B\cap\comp A)$ and
$(A\oplus B)^{\sim}=(\comp A\cup B)\cap(\comp B\cup A)
=(\comp A\cap\comp B)\cup(\comp A\cap A)\cup(B\cap\comp B)\cup(B\cap A)=(\comp A\cap\comp B)\cup (B\cap A)$.

For the fourth identity, the left-hand side is
\[
\begin{aligned}
A\oplus(B\oplus C) &= (A\cap (B\oplus C)^{\sim})\cup((B\oplus C)\cap\comp A)\\
&=(A\cap\comp B\cap\comp C)\cup(A\cap C\cap B)\cup(B\cap\comp C\cap\comp A)\cup(C\cap \comp B\cap\comp A).
\end{aligned}
\]
An analogous computation shows that $(A\oplus B)\oplus C=C\oplus(A\oplus B)
=(C\cap\comp A\cap\comp B)\cup(C\cap B\cap A)\cup(A\cap\comp B\cap\comp C)\cup(B\cap \comp A\cap\comp C)$.
By commutativity, we see that $A\oplus(B\oplus C) = (A\oplus B)\oplus C$.

For the fifth identity, the left-hand side is $(A\cap B\cap\comp C)\cup (A\cap C\cap\comp B)$ and
the right-hand side is
\[
\begin{aligned}
(A\cap B\cap (A\cap C)^{\sim})\cup(A\cap C\cap(A\cap B)^{\sim})
&=(A\cap B\cap(\comp A\cup\comp C))\cup(A\cap C\cap(\comp A\cup\comp B))\\
&=(\eset \cup A\cap B\cap\comp C)\cup(\eset\cup A\cap C\cap\comp B)\\
&=(A\cap B\cap\comp C)\cup(A\cap C\cap\comp B).
\end{aligned}
\]

The result shows that the subsets of a set form a ring with $\oplus$ as the addition
and $\cap$ as the multiplication.
\end{solution}


\section{The power set }
\begin{exercise}
Prove: If $A\sse B$, then $\power(A)\sse\power(B)$.
\end{exercise}

\begin{solution}
Suppose $A\sse B$. Let $X\in\power(A)$. Then $X\sse A\sse B$, so $X\sse B$. Hence $X\in\power(B)$.
Since $X$ was arbitrary, we have $\power(A)\sse\power(B)$.
\end{solution}

\begin{exercise}
Prove: $(\forall n\in N)\power^{n}(\eset)\sse\power^{n+1}(\eset)$.
\end{exercise}

\begin{solution}
We use the previous exercise and induction on $n$.

For $n=1$, the assertion reduces to $\power(\eset)\sse\power\power((\eset))$.
Since $\eset\sse\power(\eset)$, we have $\power(\eset)\sse\power(\power(\eset))$, so the base case holds.

Now let $n\in N$ and assume $\power^{n}(\eset)\sse\power^{n+1}(\eset)$.
We show that $\power^{n+1}(\eset)\sse\power^{n+2}(\eset)$.
Let $X\in \power^{n+1}(\eset)$. Then $X\sse \power^{n}(\eset)\sse \power^{n+1}(\eset)$.
Then $X\in \power(\power^{n+1}(\eset))=\power^{n+2}(\eset)$, as desired.
\end{solution}

\begin{exercise}
How many elements are there in $\power^n(\eset)$?
\end{exercise}

\begin{solution}
In general, if $A$ has $k$ elements, then $\power(A)$ has $2^k$ elements.
Hence $\power(\eset)$ has $2^0=1$ element, $\power^2(\eset)$ has $2^{2^0}=2$ elements,
$\power^3(\eset)$ has $2^{2^{2^0}}=4$ elements
and the number of elements in $\power^n(\eset)$ is
\[2^{2^{\iddots^{2^0}}},\]
an exponential tower of $n$ 2's and a zero.
As $2^0=1$ and an exponent of $1$ has no effect, this is equivalent to an exponential tower of
$(n-1)$ 2's, for $n>1$.
\end{solution}

\begin{exercise}
Suppose that $A$ is a set which can be built up from $\eset$ using the list notation
repeatedly, as in Section 1-6. Show that for some $n$, $A \in\power^n(\eset)$.
\end{exercise}

\begin{solution}
If there are braces $n$ deep when $A$ is written out in full, then $A\in\power^{n+1}(\eset)$.
\end{solution}

\section{Ordered pairs and direct products}
\begin{exercise}
Why do we not define $\aprn{a,b} = \set{a, \set{a,b}}$ instead of 3-3.1?
\end{exercise}

\begin{solution}
If $\set{a, \set{a,b}} = \set{c, \set{c,d}}$, we could not eliminate the possibility that $a = \set{c,d}$
and $\set{a, b} = c$ on the basis of membership arguments alone. It is true that in this case
both $a\in c$ and $c\in a$, and often it is assumed that such circular chains of membership
are impossible, but we can prove Theorem 3.3.2 under the other definition without this
assumption.
\end{solution}

\begin{exercise}
Would it be appropriate to define $\aprn{a,b,c} = \set{\set{a},\set{a,b},\set{a,b,c}}$ instead of 3-3.4?
\end{exercise}

\begin{solution}
    No, because then $\aprn{x,y,y}=\aprn{x,x,y}$.
\end{solution}

\begin{exercise}
Show that $A\times B\sse\power^2(A\cup B)$.
\end{exercise}

\begin{solution}
Suppose $x\in A\times B$. Then $x=\aprn{a,b}=\set{\set{a},\set{a,b}}$ for some $a\in A$ and $b\in B$.
Now $x$ is a set of subsets of $A\cup B$, i.e. $x\in\power^2(A\cup B)$.
Hence $A\times B\sse \power^2(A\cup B)$.
\end{solution}

\begin{exercise}
Prove the following identities:
\begin{itemize}
    \item $(A\cup B)\times C = (A\times C)\cup (B\times C)$
    \item $(A\cap B)\times C = (A\times C)\cap (B\times C)$
    \item $(A\times B)\cap (C\times D) = (A\cap C)\times (B\cap D)$
\end{itemize}
\end{exercise}

\begin{solution}
We have six inclusions to prove, two for each of the identities.

Suppose $x\in (A\cup B)\times C$. Then $x=\aprn{w, c}$ for some $w\in A\cup B$ and some $c\in C$.
If $w\in A$, then $x=\aprn{w, c}\in A\times C$. If $w\in B$, then $x=\aprn{w, c}\in B\times C$.
In either case, we have $x\in (A\times C)\cup (B\times C)$.
Hence $(A\cup B)\times C \sse (A\times C)\cup (B\times C)$.

Conversely, suppose $x\in (A\times C)\cup (B\times C)$. Assume $x\in A\times C$ (the argument for $B\times C$ is the same).
Then $x=\aprn{a,c}$ for some $a\in A\sse A\cup B$ and $c\in C$.
Hence $x\in (A\cup B)\times C$ and we have $(A\times C)\cup (B\times C) \sse (A\cup B)\times C$.

Suppose $x\in (A\cap B)\times C$. Then $x=\aprn{w, c}$ for some $w\in A\cap B$ and some $c\in C$.
Since $w\in A$, we have $x=\aprn{w, c}\in A\times C$. Similarly, we have $x\in B\times C$.
Therefore $x\in (A\times C)\cap (B\times C)$ and so $(A\cap B)\times C \sse (A\times C)\cap (B\times C)$.

Conversely, suppose $x\in (A\times C)\cap (B\times C)$. Since $x\in A\times C$ we have $x=\aprn{a,c}$ for
some $a\in A$ and $c\in C$.
Similarly, we have $x=\aprn{b,c_2}$ for some $b\in B$ and $c_2\in C$.
Then $\aprn{a,c}=\aprn{b,c_2}$, which implies $a=b\in A\cap B$.
Therefore $x=\aprn{a,c}\in (A\cap B)\times C$ and so $(A\times C)\cap (B\times C)\sse (A\cap B)\times C$.

Suppose $\aprn{x, y} \in (A \times B)\cap (C \times D)$. Then $\aprn{x, y}\in A\times B$,
so $x\in A$ and $y \in B$. Similarly, $x \in C$ and $y\in D$.
Therefore, $x\in A\cap C$ and $y\in B\cap D$.
This shows that $\aprn{x, y}\in(A\cap C)\times (B\cap D)$.
Hence $(A \times B)\cap (C \times D) \sse (A\cap C)\times (B\cap D)$.

Conversely, suppose $\aprn{u,v}\in (A\cap C)\times (B\cap D)$.
Then $u\in A$ and $v\in B$, so $\aprn{u,v}\in A\times B$. Similarly, $\aprn{u,v}\in C\times D$.
Hence $\aprn{u,v}\in (A\times B)\cap (C\times D)$ and so
$(A\cap C)\times (B\cap D)\sse (A\times B)\cap (C\times D)$.
\end{solution}

\begin{exercise}
Prove: $A\times B = \eset$ if and only if $A=\eset$ or $B=\eset$.
\end{exercise}

\begin{solution}
If both $A$ and $B$ are not void, choose $a \in A$ and $b\in B$. Then $\aprn{a,b}\in A \times B$,
and the latter is not void. Thus $(A\neq\eset \land B\neq\eset)\implies A\times B\neq\eset$.
This is the contrapositive of $A\times B = \eset\implies (A=\eset\lor B=\eset)$.
The reverse implication is trivial. If $A\times B\neq \eset$, then there exists $a\in A$
and $b\in B$ such that $\aprn{a,b}\in A\times B$. In particular, both $A$ and $B$ are not void.

\end{solution}

\begin{exercise}
The following cancellation law is invalid. If $X \times Y = X \times Z$, then $Y = Z$.
Prove a corrected statement.
\end{exercise}

\begin{solution}
The cancellation law is invalid when $X=\eset$ since $\eset\times Y = \eset=\eset\times Z$
for any sets $Y$ and $Z$.
However, if $X\neq\eset$ and $X \times Y = X \times Z$, then $Y = Z$.

\textit{Proof}. Suppose $X\neq\eset$ and $X \times Y = X \times Z$. Choose some $x\in X$.
Let $y\in Y$ be arbitrary. Then $\aprn{x, y}\in X \times Y = X \times Z$, so $y\in Z$.
Hence $Y\sse Z$. Similarly, $Z\sse Y$, and we have $Y=Z$.
\end{solution}


\section{Functions}
\begin{exercise}
The following sets are all functions:
\begin{gather*}
f_1 = \set{\aprn{1,1},\aprn{2,1},\aprn{3,2},\aprn{4,0}},\qquad f_2 = \set{\aprn{2,1},\aprn{4,1},\aprn{1,2},\aprn{4,1}},\\
f_3 = \set{\aprn{0,2},\aprn{2,2},\aprn{1,4},\aprn{3,0}},\qquad f_4 = \set{\aprn{4,1},\aprn{1,2},\aprn{2,1},\aprn{0,5}}.
\end{gather*}
Write out the domain and range of each of them. Which of them are injective? Is any
one a restriction of another? Compute $f_1\circ f_2$ and $f_2\circ f_3$. Then compute
$(f_1\circ f_2)\circ f_3$ and $f_1\circ(f_2\circ f_3)$.

If $f$ stands for $f_3$ restricted to $\set{0,1}$, what is the map $\overbar f$ induced by $f$
on $\power(\set{0,1})$?
\end{exercise}

\begin{solution}
We have the following domains and ranges:
\begin{align*}
\dom f_1 &= \set{1, 2, 3, 4} & \ran f_1 &= \set{1,2,0} \\
\dom f_2 &= \set{2, 4, 1} & \ran f_2 &= \set{1,2} \\
\dom f_3 &= \set{0, 2, 1, 3} & \ran f_3 &= \set{2,4,0} \\
\dom f_4 &= \set{4,1,2,0} & \ran f_4 &= \set{1,2,5}
\end{align*}
None are injective (the range of an injective function with a finite domain must be the same size).
The function $f_2$ is a restrction of $f_4$.

We compute $f_1\circ f_2=\set{\aprn{2,1},\aprn{4,1},\aprn{1,1}}$
and $f_2\circ f_3=\set{\aprn{0,1},\aprn{2,1},\aprn{1,1}}$.
Then $(f_1\circ f_2)\circ f_3=\set{\aprn{0,1},\aprn{2,1},\aprn{1,1}}=f_1\circ (f_2\circ f_3)$.

$\overbar f=\set{\aprn{\eset,\eset},\aprn{\set{0},\set{2}},\aprn{\set{1},\set{4}},\aprn{\set{0,1},\set{2,4}}}$.
\end{solution}

\begin{exercise}
Are the coordinate projections of a direct product $A\times B$ injective? surjective?
\end{exercise}

\begin{solution}
Let $f$ from $A\times B$ to $A$ be the first coordinate projection.
Then $f$ is injective if and only if $B$ has at most one element.
And $f$ is surjective if and only if $B$ is not void or $A$ is void.
\end{solution}

\begin{exercise}
Using the notation of 3-4.12, we find that $f\mapsto \overbar f$
is a map from the set of all functions
from $A$ to $B$ to the set of all functions from $\power(A)$ to $\power(B)$.
Is it injective? surjective?
\end{exercise}

\begin{solution}
It is always injective. It is surjective if and only if $A$ and $B$ are both empty.
\end{solution}

\begin{exercise}
Let $f$ be a map from $A$ to $B$. Let $f^*$ be the induced map from $A\times A$ to $B\times B$ as
in 3-4.14. Let $\overbar f$ be the induced map from $\power(A)$ to $\power(B)$ as in 3-4.12
and let $\overbar{\overbar{f}}$ be the
induced map from $\power^2(A)$ to $\power^2(B)$.
Show that $f^*$ is $\overbar{\overbar{f}}$ restricted to $A \times A$. 
\end{exercise}

\begin{solution}
Let $\aprn{a_1,a_2}\in A\times A$.
Then we have 
\[
\begin{aligned}
\overbar{\overbar{f}}(\aprn{a_1,a_2})
&=\overbar{\overbar{f}}(\set{\set{a_1},\set{a_1,a_2}})\\
&=\set{\overbar{f}(\set{a_1}), \overbar{f}(\set{a_1,a_2})}\\
&=\set{\set{f(a_1)}, \set{f(a_1),f(a_2)}}\\
&=\aprn{f(a_1),f(a_2)}\\
&=f^*(\aprn{a_1,a_2}).
\end{aligned}
\]
Hence $f^*$ is $\overbar{\overbar{f}}$ restricted to $A \times A$.
\end{solution}

\begin{exercise}
Let $\mc F$ be the set of all functions from $A$ to $B$.
Let $\mc G$ be the set of all functions from
$\mc F$ to $B$. Explain fully how the formula $\phi(a)(f)=f(a)$
defines a function $\phi$ from $A$ to $\mc G$.
Is $\phi$ injective? surjective?
\end{exercise}

\begin{solution}
We can define a function $\phi$ from $A$ to $\mc G$ by defining, for each $a\in A$,
a member $\phi(a)$ of $\mc G$. Since a member of $\mc G$ is itself a function from $\mc F$
to $B$, we may define $\phi(a)$ by giving its values at each point of $\mc F$.
So we put $\phi(a)(f) = f(a)$ for all $f\in \mc F$. Now, $f\in\mc F$
and $a\in A$ imply that $f(a)\in B$, so $\phi(a)$ is a function from $\mc F$ to $B$---that is, a member of $\mc G$.
In symbols, $\phi = \setb{\aprn{a, \setb{\aprn{f, f(a)}}{f\in \mc F}}}{a\in A}$.

We show that $\phi$ is injective if and only if $B$ has more than one element or $A$ has at most one element.

Suppose $\phi$ is not injective. Then there exist $a_1,a_2\in A$ such that $a_1\neq a_2$ and
$\phi(a_1)= \phi(a_2)$. In particular, $A$ has more than one element.
Also, since $\phi(a_1)= \phi(a_2)$, for all $f\in\mc F$, $f(a_1)=\phi(a_1)(f)=\phi(a_2)(f)=f(a_2)$.
Now suppose $b_1,b_2\in B$. Define $f_1\in\mc F$ by $f_1(a_1)=b_1$ and $f_1(x)=b_2$ for $x\neq a_1$.
Since $f_1\in\mc F$, we have $b_1=f_1(a_1)=f_1(a_2)=b_2$. Hence $B$ has at most one element.

Conversely, suppose $B$ has at most one element and $A$ has more than one element.
Choose $a_1,a_2\in A$ with $a_1\neq a_2$.
If $\phi$ were injective, then we would have $\phi(a_1)\neq\phi(a_2)$, which means
there exists $f\in\mc F$ such that $f(a_1)=\phi(a_1)(f)\neq\phi(a_2)(f)=f(a_2)$.
In particular, $f(a_1)$ and $f(a_2)$ would be distinct elements of $B$.
But this contradicts the assumption that $B$ has at most one element. Therefore $\phi$ is not injective.

Now we show that $\phi$ is surjective only if $B$ has at most one element.

Suppose $\phi$ is surjective. Then for every $g\in \mc G$, there is some $a\in A$
such that $\phi(a) = g$, i.e. $f(a)=\phi(a)(f)=g(f)$ for all $f\in\mc F$.
Let $b_1,b_2\in B$.
Let $g_1\in\mc G$ be defined by $g_1(f)=b_1$ for all $f\in\mc F$.
Then since $g_1\in\mc G$, there exists $a_1\in A$ such that $f(a_1)=g_1(f)=b_1$ for all $f\in \mc F$.
In particular, if $f_1$ is the constant function from $A$ to $B$ with value $b_2$,
we have $b_2=f_1(a_1)=b_1$. This shows that $B$ has at most one element.

Now we show that if $A\neq\eset$ and $B$ has at most one element, then $\phi$ is surjective.

Suppose $B$ has at most one element and choose $a_1\in A$. Let $g\in\mc G$ be arbitrary.
Then for any $f\in\mc F$, we have $\phi(a_1)(f)=f(a_1)\in B$ and $g(f)\in B$.
Since $B$ has at most one element, we must have $\phi(a_1)(f)=g(f)$.
Since $\phi(a_1)$ and $g$ are equal on all $f\in\mc F$, we have $\phi(a_1)=g$.
Therefore $\phi$ is surjective.

Finally, suppose $A=B=\eset$.
Then $\mc F=\set{\eset}$.
Then $\mc G=\eset$, since there are no functions from $\mc F\neq\eset$ to $B=\eset$.
Therefore $\phi$ is trivially surjective.
\end{solution}

\begin{exercise}
Let $f$ be a function from $A$ to $B$. Show that
\begin{enumerate}[label=(\alph*)]
    \item $f$ is injective if and only if (2) is an equality for every subset $X$ of $A$;
    \item $f$ is surjective if and only if (3) is an equality for every subset $Y$ of $B$;
    \item $f$ is injective if and only if (5) is an equality for every two subsets $X_1$ and $X_2$ of $A$.
\end{enumerate}
\end{exercise}

\begin{solution}
We prove (a).

Let us first prove (2) which is $X\sse f^{-1}(f(X))$ for all $X\sse A$.
Suppose $x\in X$. Then $f(x)\in f(X)$,
which means $x\in f^{-1}(f(X))$.

Now suppose $f$ is injective, and let $X$ be any subset of $A$. We must show $f^{-1}(f(X)) \sse X$.
Suppose $z\in f^{-1}(f(X))$. Then $f(z)\in f(X)$, so $f(z)=f(x)$ for some $x\in X$.
Since $f$ is injective, this implies $z=x\in X$. Therefore $f^{-1}(f(X)) \sse X$. Combined with (2)
we get $X = f^{-1}(f(X))$.
Conversely, suppose $X = f^{-1}(f(X))$ for all $X\sse A$. Suppose $f(a)=f(b)$.
Then $a\in f^{-1}f(\set{b})$ since $f(a)=f(b)\in f(\set{b})$.
But $f^{-1}f(\set{b})=\set{b}$, so $a\in\set{b}$ and $a=b$.
Therefore $f$ is injective. This proves (a).

We prove (b).

Let us first prove (3) which is $f(f^{-1}(Y))\sse Y$ for all $Y\sse B$.
Suppose $z\in f(f^{-1}(Y))$. Then $z=f(x)$ for some $x\in f^{-1}(Y)$.
Then $z=f(x)\in Y$, as desired.

Now suppose $f$ is surjective. Let $Y$ be any subset of $B$.
We must show that $Y\sse f(f^{-1}(Y))$. Let $y\in Y$.
Since $f$ is surjective, there exists $a\in A$ with $f(a)=y$.
Then $f(a)\in Y$, so $a\in f^{-1}(Y)$. Then $y=f(a)\in f(f^{-1}(Y))$.
Thus $Y\sse f(f^{-1}(Y))$. Combined with (3) we get $Y= f(f^{-1}(Y))$.
Conversely, suppose $Y= f(f^{-1}(Y))$ for all $Y\sse B$.
Let $b\in B$ be given. Then $f(f^{-1}(\set{b}))=\set{b}$.
Hence $f^{-1}(\set{b})$ is not empty, i.e. there exists $a\in A$ such that $f(a)=b$.
Therefore, $b\in\ran f$, and $f$ is surjective.

We prove (c).

Let us prove (5) which is $f(X_1\cap X_2)\sse f(X_1)\cap f(X_2)$ for all $X_1,X_2\sse A$.
Suppose $b\in f(X_1\cap X_2)$. Then $b=f(x)$ for some $x\in X_1\cap X_2$.
Since $x\in X_1$, we have $b=f(x)\in f(X_1)$. Similarly we have $b\in f(X_2)$.
Hence $b\in f(X_1)\cap f(X_2)$ and we get $f(X_1\cap X_2)\sse f(X_1)\cap f(X_2)$.

Now suppose $f$ is injective, and let $X_1$ and $X_2$ be subsets of $A$.
Let $y\in f(X_1)\cap f(X_2)$ be given.
Since $y\in f(X_1)$, there exists $x_1\in X_1$ such that $y=f(x_1)$.
Similarly, there exists $x_2\in X_2$ such that $y=f(x_2)$.
Since $f$ is injective, this means $x_1=x_2$. Then $x_1\in X_1\cap X_2$.
So $y=f(x_1)\in f(X_1\cap X_2)$. Combined with (5) we get $f(X_1\cap X_2)= f(X_1)\cap f(X_2)$.
Conversely, suppose $f(X_1\cap X_2)= f(X_1)\cap f(X_2)$ for all $X_1,X_2\sse A$.
Assume $f(a)=f(b)$. Then $a\in f(\set{a})\cap f(\set{b})=f(\set{a}\cap\set{b})$.
Then $\set{a}\cap\set{b}\neq\eset$, so $a=b$. Therefore $f$ is injective.
\end{solution}

\begin{exercise}
Suppose that $f$ is a function from $A$ to $B$. If $X\sse A$ and $Y\sse B$, show that
$f(X\cap f^{-1}(Y)) = f(X)\cap Y$.
\end{exercise}

\begin{solution}
By (5) we have $f(X\cap f^{-1}(Y)) \sse f(X)\cap f(f^{-1}(Y))$.
Using (3) we get $f(X)\cap f(f^{-1}(Y))\sse f(X)\cap Y$.
Thus $f(X\cap f^{-1}(Y)) \sse f(X)\cap Y$.

Conversely, let $b\in f(X)\cap Y$. Since $b\in f(X)$, we have $b=f(x)$ for some $x\in X$.
Since $f(x)=b\in Y$, we have $x\in f^{-1}(Y)$. Therefore $x\in X\cap f^{-1}(Y)$ and we have
$b=f(x)\in f(X\cap f^{-1}(Y))$.
This shows that $f(X\cap f^{-1}(Y)) \Sse f(X)\cap Y$.
\end{solution}

\section{Relations}
\begin{exercise}
If $R$ and $S$ are relations, then
\[R\circ S = \setb{\aprn{x,y}}{(\exists z)\aprn{x,z}\in S\land\aprn{z,y}\in R}\]
is a relation, called the composition of $R$ and $S$. If $R$ and $S$ should be functions, show
that this definition coincides with the previous definition of the composition of functions.
Show also that the composition of relations is associative; that is,
\[(R\circ S)\circ T=R\circ (S\circ T).\]
\end{exercise}

\begin{solution}
Suppose $R$ and $S$ are functions and let $Q$ be their composition in the sense of 3-4.7.
Let $\aprn{x,y}\in R\circ S$. Choose $z$ so that $\aprn{x,z}\in S$ and $\aprn{z,y}\in R$.
Then $y=R(z)$ and $z=S(x)$, so $y=R(S(x))=Q(x)$. Therefore $\aprn{x,y}\in Q$.
Conversely, if $\aprn{x,y}\in Q$, then $y=R(S(x))$, so $\aprn{x,S(x)}\in S$ and $\aprn{S(x),y}\in R$.
Therefore, $\aprn{x,y}\in R\circ S$. Thus $R\circ S=Q$.

Suppose $\aprn{x, y}\in (R\circ S)\circ T$. Then there exists $z_1$ with $\aprn{x,z_1}\in T$
and $\aprn{z_1, y}\in R\circ S$.
Since $\aprn{z_1, y}\in R\circ S$, there exists $z_2$ with $\aprn{z_1,z_2}\in S$ and $\aprn{z_2, y}\in R$.
Now $\aprn{x,z_1}\in T$ and $\aprn{z_1,z_2}\in S$ together imply $\aprn{x,z_2}\in S\circ T$.
Then combined with $\aprn{z_2, y}\in R$, we get $\aprn{x, y}\in R\circ(S\circ T)$.
Thus $(R\circ S)\circ T\sse R\circ(S\circ T)$.
The reverse inclusion can be shown similarly.
\end{solution}

\begin{exercise}
If $R$ is a relation and $X$ is any set, define $\overbar{R}(X) = \setb{y}{(\exists x \in X) \aprn{x, y} \in R}$.
Show that $\overbar{R}(\overbar{S}(X))=\overbar{(R\circ S)}(X)$ for any relations $R$ and $S$ and any set $X$.
Restricting our attention to sets $X$ which are subsets of the domain of $R$, we can regard $\overbar R$ as a function
from $\power(\dom R)$ to $\power(\ran R)$. If $\overbar R = \overbar S$, must $R = S$?
\end{exercise}

\begin{solution}
Let $R$ and $S$ be relations and $X$ be a set.

Suppose $z\in \overbar{R}(\overbar{S}(X))$. Then there exists $x\in \overbar{S}(X)$ such that
$\aprn{x, z}\in R$. Then there exists $x_2\in X$ such that $\aprn{x_2,x}\in S$.
Now $\aprn{x_2, z}\in R\circ S$ and $x_2\in X$, so $z\in\overbar{(R\circ S)}(X)$.
Thus $\overbar{R}(\overbar{S}(X))\sse \overbar{(R\circ S)}(X)$.

Conversely, suppose $z\in \overbar{(R\circ S)}(X)$. Then there exists $x\in X$ such that
$\aprn{x, z}\in R\circ S$. Then there exists $x_2$ such that $\aprn{x,x_2}\in S$ and
$\aprn{x_2,z}\in R$.
Since $\aprn{x,x_2}\in S$ and $x\in X$, we have $x_2\in\overbar{S}(X)$.
Since $\aprn{x_2,z}\in R$ and $x_2\in\overbar{S}(X)$, we have $z\in\overbar{R}(\overbar{S}(X))$.
Thus $\overbar{R}(\overbar{S}(X))\Sse \overbar{(R\circ S)}(X)$.

Suppose $\overbar R=\overbar S$. We show that $R=S$.
Let $\aprn{x,y}\in R$ be given. We have $y\in \overbar R(\set{x})=\overbar S(\set{x})$, which means
$\aprn{x,y}\in S$. THus $R\sse S$ and the reverse inclusion follows by symmetry.
\end{solution}

\section{Indexed unions and intersections}\
\begin{exercise}
Prove the following theorem which concerns a construction of frequent applicability.

\textbf{Theorem.} Suppose $\setb{f_i}{i\in I}$ is a family of functions and $A_i=\dom f_i$.
Then $\bigcup_i f_i$ is a function if and only if
\[(\forall i,j\in I)(\forall x\in A_i\cap A_j) f_i(x)=f_j(x).\tag{11} \]
When (11) holds it is appropriate to call $\bigcup_i f_i$ the
\textit{least common extension} of the functions $f_i$. Why? What is its domain? its range?
\end{exercise}

\begin{solution}
Let $g=\bigcup_i f_i$. Suppose $g$ is a function. Let $i,j\in I$ and let $x\in A_i\cap A_j$.
Then $\aprn{x, f_i(x)}\in f_i\sse g$ and $\aprn{x, f_j(x)}\in f_j\sse g$.
Since $g$ is a function, we have $f_i(x)=f_j(x)$.

Conversely, suppose (11) holds. Obviously, $g$ is a set of ordered pairs.
Suppose $\aprn{x,y},\aprn{x,z}\in G$.
Then there exists $i\in I$ such that $\aprn{x,y}\in f_i$.
Similarly, there exists $j\in I$ such that $\aprn{x,z}\in f_j$.
Then $x\in A_i\cap A_j$ and from (11) we have $y=f_i(x)=f_j(x)=z$.
Therefore $g$ is a function.

Clearly $g$ is an extension of the functions $f_i$ since $f_i\sse g$.
If $h$ is also a function which is an extension of the functions $f_i$, then $g\sse h$.
Thus $g$ is appropriately called
the least common extension of the functions $f_i$.
Clearly, $\dom g = \bigcup\dom{f_i}$ and $\ran g = \bigcup \ran f_i$;
\end{solution}

\section{Indexed direct products}
We shall use the symbol $\prod$ for a direct product of a family of sets.

\begin{exercise}
Let $\setb{A_i}{i\in I}$ and $\setb{B_i}{i\in I}$ be families of sets with the same index set.
Prove $\prn{\prod_i A_i}\cap\prn{\prod_i B_i} = \prod_i (A_i\cap B_i)$.
\end{exercise}

\begin{solution}
Suppose $f\in \prn{\prod_i A_i}\cap\prn{\prod_i B_i}$.
Since $f\in \prod_i A_i$, $f$ is a function with domain $I$ such that $(\forall i\in I)f(i)\in A_i$.
Similarly, we have $(\forall i\in I)f(i)\in B_i$.
But that means $(\forall i\in I)f(i)\in A_i\cap B_i$.
Thus $\prod_i (A_i\cap B_i)$, and $\prn{\prod_i A_i}\cap\prn{\prod_i B_i} \sse \prod_i (A_i\cap B_i)$.
Since the previous argument is reversible, we obtain also the opposite inclusion.
\end{solution}

\begin{exercise}
Suppose that $\set{A_i}$, $\set{B_i}$, and $\set{C_i}$ are families with the same index set I.
Let $j\in I$ and suppose that $A_i=B_i=C_i$ for all $i\in I$ except $j$. Given
$A_j=B_j\cup C_j$, prove $\prod_i A_i = \prn{\prod_i B_i} \cup \prn{\prod C_i}$.
\end{exercise}

\begin{solution}
Suppose $f\in prod_i A_i$. Then $f(j)\in A_j=B_j\cup C_j$. Assume $f(j)\in B_j$.
Now, for all $i\in I$ with $i\neq j$, we have $f(i)\in A_i=B_i$.
Thus, for all $i\in I$, we have $f(i)\in B_i$, so $f\in\prod B_i$.
A similar argument applies if $f(j)\in C_j$.
In either case, $f\in \prn{\prod_i B_i} \cup \prn{\prod C_i}$.
Thus $\prod_i A_i \sse \prn{\prod_i B_i} \cup \prn{\prod C_i}$.

Conversely, suppose $f\in \prn{\prod_i B_i} \cup \prn{\prod C_i}$.
Without loss of generalization, assume $f\in \prn{\prod_i B_i}$.
Then $f(i)\in B_i$ for all $i\in I$.
For $i\neq j$ we have $f(i)\in B_i=A_i$.
And we have $f(j)\in B_j\sse B_j\cup C_j = A_j$.
Thus, for all $i\in I$, we have $f(i)\in A_i$, so $f\in \prod_i A_i$.
This shows that $\prod_i A_i \Sse \prn{\prod_i B_i} \cup \prn{\prod C_i}$.
\end{solution}

\begin{exercise}
Suppose that $\setb{A_{i,j}}{\aprn{i,j}\in I\times J}$ is a family of sets indexed on
a direct product. Define a natural bijection from $\prod_{I\times J} A_{i,j}$
to $\prod_i\prn{\prod_j A_{i,j}}$.
\end{exercise}

\begin{solution}
For a given $f\in \prod_{I\times J} A_{i,j}$, let $\phi(f)$ be the function defined on $I$
by $\phi(f)(i)=h(i)$, where $h(i)$ is the function with domain $J$ defined
by $h(i)(j)=f(i,j)$.
Since $f(i,j)\in A_{i,j}$, $\phi(f)(i)=h(i)\in \prod_j A_{i,j}$ for all $i\in I$.
This shows that
\[\phi(f)\in\prod_i\prn{\prod_j A_{i,j}}.\]
Thus $\phi$ defined by $f\mapsto \phi(f)$ is a function from $\prod_{I\times J} A_{i,j}$
to $\prod_i\prn{\prod_j A_{i,j}}$.

Suppose $\phi(f_1)=\phi(f_2)$.
Let $\aprn{i,j}\in I\times J$ be given.
Since $\phi(f_1)=\phi(f_2)$, we have $f_1(i,j)=\phi(f_1)(i)(j)=\phi(f_2)(i)(j)=f_2(i,j)$.
Hence $f_1=f_2$ and $\phi$ is injective.

Now let $g\in \prod_i\prn{\prod_j A_{i,j}}$.
Then $f(i,j)=g(i)(j)$ defines a function $f$ with domain $I\times J$ such that $(\forall i,j)f(i,j)\in A_{i,j}$.
Then $f\in \prod_{I\times J} A_{i,j}$ and $\phi(f)=g$.
Thus $\phi$ is surjective.
\end{solution}

\begin{exercise}
Let $\setb{A_i}{i\in I}$ be a family of sets and suppose that $K\sse I$.
Show that
\[f\mapsto \aprn{f\text{ restricted to }K, f\text{ restricted to }I - K}\]
is a bijection from $\prod_i A_i$ to $\prn{\prod_K A_i}\times\prn{\prod_{I-K} A_i}$.
\end{exercise}

\begin{solution}
Let $\phi$ denote the function $f\mapsto \aprn{f\text{ restricted to }K, f\text{ restricted to }I - K}$.
If $f\in \prod_i A_i$, then ($f$ restricted to $K$) is a member of $\prod_K A_i$ and
($f$ restricted to $I-K$) is a member of $\prod_{I-K} A_i$. Thus $\phi$ is a function from
$\prod_i A_i$ to $\prn{\prod_K A_i}\times\prn{\prod_{I-K} A_i}$.

Suppose $\phi(f_1)=\phi(f_2)$. Then
\[\aprn{f_1\text{ restricted to }K, f_1\text{ restricted to }I - K} = \aprn{f_2\text{ restricted to }K, f_2\text{ restricted to }I - K}.\]
Then for all $i\in I-K$, we have $f_1(i)=(f_1\text{ restricted to }I - K)(i)=(f_2\text{ restricted to }I - K)(i)=f_2(i)$.
Similarly, $f_1(i)=f_2(i)$ for all $i\in K$.
Thus $(\forall i\in I)f_1(i)=f_2(i)$, which means $f_1 = f_2$.
Therefore $\phi$ is injective.

Let $\aprn{g,h}$ be any element of $\prn{\prod_K A_i}\times\prn{\prod_{I-K} A_i}$.
Let $f=g\cup h$. Then $f$ is a function with domain $I$ since $\dom g = K$ and $\dom h=I-K$.
Also, $f(i)$ is equal to either $g(i)$ or $h(i)$, both of which are in $A_i$.
Thus $f\in\prod_I A_i$. Obviously, $(f\text{ restricted to }K)=g$
and $(f\text{ restricted to }I-K)=h$, so $\phi(f)=\aprn{g, h}$. Therefore $\phi$ is surjective.


\end{solution}

\chapter{Mathematical Configurations}
\section{Structures and configurations}

In the following exercises find the pairs of configurations which are isomorphic. For
each pair, either find an explicit bijection which effects the isomorphism or give reasons
why there is none. Assume that distinct symbols in the basic sets represent distinct
objects. (Exercise 4 is offered primarily as a puzzle.)

\begin{exercise}
\[
\begin{array}{ll}
    \text{Basic set} & \text{Structural set}\\
    \set{a,b,c,d,e} & \set{\set{a},\set{a,b},\set{b,c},\set{b,d,e}}\\
    \set{f,g,h,i,j} & \set{\set{f,g},\set{f,h},\set{g,i},\set{h,j}}\\
    \set{p,q,r,s,t} & \set{\set{p,q,r},\set{p,s},\set{p,t},\set{s}}
\end{array}
\]
\end{exercise}

\begin{solution}
Since no singleton appears in the structural set for $\set{f, g, h,i,j}$, it is not isomorphic
to either of the others.

Let us try to produce an isomorphism $\phi$ from the first configuration to the third.
The singleton $\set{a}$ must be mapped to $\set{s}$, so we have $\phi(a)=s$.
We set $\phi(b)=p$ to preserve the single common element among the non-singleton sets.
We set $\phi(c)=t$ as we must have $\set{b,c}$ mapped to $\set{p,t}$ ($\set{a,b}$ is already mapped to $\set{p,s}$).
Finally $d$ and $e$ must be mapped to $q$ and $r$, and there are two ways to do this.
So $\set{\aprn{a,s},\aprn{b,p},\aprn{c,t},\aprn{d,q},\aprn{e,r}}$ is an isomorphism from the
first configuration to the third.
\end{solution}

\begin{exercise}
\[
\begin{array}{ll}
    \text{Basic set} & \text{Structural set}\\
    \set{a,b,c,d,e} & \set{\set{a,b,c},\set{a,c,e},\set{b,c},\set{b,d}}\\
    \set{f,g,h,i,j} & \set{\set{f,g},\set{f,h,j},\set{g,h},\set{g,h,i}}\\
    \set{p,q,r,s,t} & \set{\set{p,r},\set{p,r,s},\set{p,t},\set{q,r,t}}
\end{array}
\]
\end{exercise}

\begin{solution}
Since the triplets in the first configuration intersect in a pair while those in the second
and third configurations intersect in a singleton, only the second and third configurations
can be isomorphic.

Let us try to produce an isomorphism $\phi$ from the second configuration to the third.
The two pairs have a single element in common, so we set $\phi(g)=p$ to preserve that common element.
This common element belongs to only one of the triplets, so $\set{g,h,i}$ must map to $\set{p,r,s}$.
In particular, $h$ must map to $r$ or $s$. But also, $\set{g,h}$ must map to $\set{p,t}$ or $\set{p,r}$,
so $h$ must map to $t$ or $r$. Therefore $\phi(h)=r$, and then $\phi(i)=s$.
Then $\set{f,g}$ must map to $\set{p,t}$, so $\phi(f)=t$. Finally, $\phi(j)=q$.
So $\set{\aprn{f,t},\aprn{g,p},\aprn{h,r},\aprn{i,s},\aprn{j,q}}$ is an isomorphism
from the second configuration to the third, and in fact it is the only isomorphism.
\end{solution}

\begin{exercise}
\[
\begin{array}{ll}
    \text{Basic set} & \text{Structural set}\\
    \set{a,b,c,d,e} & \set{\set{a,b,c},\set{a,c,d},\set{a,d,e},\set{b,c,d}}\\
    \set{f,g,h,i,j} & \set{\set{f,g,j},\set{f,h,j},\set{g,h,i},\set{h,i,j}}\\
    \set{p,q,r,s,t} & \set{\set{p,q,r},\set{p,q,t},\set{p,s,t},\set{q,r,s}}
\end{array}
\]
\end{exercise}

\begin{solution}
In the first configuration there is only one member of the structural set containing $e$.
In the other two configurations each member of the basic set appears as an element in
at least two members of the structural set. Therefore the first configuration is not isomorphic
to any of the others.

In the second configuration, both $h$ and $j$ appear in 3 members of the structural set.
In the third configuration, it is $p$ and $q$ that appear in 3 members of the structural set.
So we must have $h$ and $j$ mapped to $p$ and $q$, and there are two ways to do this.
Now note that $h$ and $j$ appear together in $\set{f,h,j}$ and $\set{h,i,j}$, while $p$ and $q$
appear together in $\set{p,q,r}$ and $\set{p,q,t}$.
So $f$ and $i$ must be mapped to $r$ and $t$, and there are two ways to do this.
That leaves $g$ to be mapped to $s$.

So $\set{\aprn{g,s},\aprn{f,r},\aprn{i,t},\aprn{h,p},\aprn{j,q}}$ is an isomorphism
from the second configuration to the third. Indeed, applying the bijection to
the second structural set results in the third.
\end{solution}


\begin{exercise}
\[
\begin{array}{ll}
    \text{Basic set} & \text{Structural set}\\
    \set{a,b,c,d,e,f,g,h} & \Bigl\{ \bigl\{ \set{a,b}, \set{c,d}, \set{e,f}, \set{g,h} \bigr\},\\
               & \hphantom{\Bigl\{} \bigl\{ \set{a,c}, \set{b,d}, \set{e,g}, \set{f,h} \bigr\},\\
               & \hphantom{\Bigl\{} \bigl\{ \set{a,d}, \set{b,c}, \set{e,h}, \set{f,g} \bigr\},\\
               & \hphantom{\Bigl\{} \bigl\{ \set{a,e}, \set{b,f}, \set{c,g}, \set{d,h} \bigr\},\\
               & \hphantom{\Bigl\{} \bigl\{ \set{a,f}, \set{b,h}, \set{c,e}, \set{d,g} \bigr\},\\
               & \hphantom{\Bigl\{} \bigl\{ \set{a,g}, \set{b,e}, \set{c,h}, \set{d,f} \bigr\},\\
               & \hphantom{\Bigl\{} \bigl\{ \set{a,h}, \set{b,g}, \set{c,f}, \set{d,e} \bigr\} \Bigr\}\\
    \set{s,t,u,v,w,x,y,z} & \Bigl\{ \bigl\{ \set{s,t}, \set{u,w}, \set{v,y}, \set{x,z} \bigr\},\\
               & \hphantom{\Bigl\{} \bigl\{ \set{s,u}, \set{t,w}, \set{v,x}, \set{y,z} \bigr\},\\
               & \hphantom{\Bigl\{} \bigl\{ \set{s,v}, \set{t,y}, \set{u,x}, \set{w,z} \bigr\},\\
               & \hphantom{\Bigl\{} \bigl\{ \set{s,w}, \set{t,x}, \set{u,y}, \set{v,z} \bigr\},\\
               & \hphantom{\Bigl\{} \bigl\{ \set{s,x}, \set{t,z}, \set{u,v}, \set{w,y} \bigr\},\\
               & \hphantom{\Bigl\{} \bigl\{ \set{s,y}, \set{t,v}, \set{u,z}, \set{w,x} \bigr\},\\
               & \hphantom{\Bigl\{} \bigl\{ \set{s,z}, \set{t,u}, \set{u,w}, \set{x,y} \bigr\} \Bigr\}\\
\end{array}
\]
\end{exercise}

\begin{solution}
They are isomorphic. There are 64 different isomorphisms. Eight contain the pair
$\aprn{a, s}$, eight contain $\aprn{a, t}$, etc.
One is $\set{\aprn{a,s},\aprn{b,t},\aprn{c,y},\aprn{d,v},\aprn{e,u},\aprn{f,w},\aprn{g,z},\aprn{h,x}}$.
\end{solution}

\section{Definitions, postulates, and theorems}
\begin{exercise}
Find all ordered sets $\aprn{A,B}$, where $A = \set{1, 2, 3}$.
\end{exercise}

\begin{solution}
There are 19 possible values for $B$ which make $\aprn{A,B}$ an ordered set.
One is the empty set.
Six are $\set{\aprn{a,b}}$ where $a$ and $b$ are distinct.
Six are $\set{\aprn{a,b},\aprn{b,c},\aprn{a,c}}$ where $a$, $b$, and $c$ are distinct.
Three are $\set{\aprn{a,b},\aprn{a,c}}$ where $a$, $b$, and $c$ are distinct.
Three are $\set{\aprn{a,c},\aprn{b,c}}$ where $a$, $b$, and $c$ are distinct.
\end{solution}

\begin{exercise}
A \textit{Steiner triple system} is a configuration $\aprn{A, B}$ such that $B\in\power^2(A)$ and
\begin{enumerate}[label=(\alph*)]
    \item every member of $B$ has three members;
    \item every two-element subset of $A$ is a subset of exactly one member of $B$.
\end{enumerate}
Find a Steiner triple system in which $A$ has seven members.
\end{exercise}

\begin{solution}
Let $A=\set{a,b,c,d,e,f,g}$.
Let us try to construct a Steiner triple system $B$ piece by piece.
First we add the triplet $\set{a,b,c}$ to $B$.
This contains the two-element subsets $\set{a,b}$ and $\set{a,c}$ of which $a$ is a member.
Next we add $\set{a,d,e}$ and $\set{a,f,g}$ to $B$ handle the rest of the two-element subsets
of which $a$ is a member.

We have so far $B=\set{\set{a,b,c}, \set{a,d,e}, \set{a,f,g}}$.

Moving on to $b$, we still need the two-element subsets $\set{b,d}$, $\set{b,e}$, $\set{b,f}$, and $\set{b,g}$
included in exactly one triplet.
We cannot use $\set{b,d,a}$ or $\set{b,d,c}$ because $\set{a,b,c}$ is already in $B$.
We cannot use $\set{b,d,e}$ since $\set{d,e}$ is already included in $\set{a,d,e}$.
Let us add the triplets $\set{b,d,f}$ and $\set{b,e,g}$ to $B$.

We have so far $B=\set{\set{a,b,c}, \set{a,d,e}, \set{a,f,g}, \set{b,d,f}, \set{b,e,g}}$.

Moving on to $c$, we still need the two-element subsets $\set{c,d}$, $\set{c,e}$,
$\set{c,f}$, and $\set{c,g}$ included in exactly one triplet.
We cannot use $\set{c,d,a}$ or $\set{c,d,b}$ because $\set{a,b,c}$ is already in $B$.
We cannot use $\set{c,d,e}$ since $\set{d,e}$ is already included in $\set{a,d,e}$.
We also cannot use $\set{c,d,f}$ since $\set{d,f}$ is already included in $\set{b,d,f}$.
Let us add the triplets $\set{c,d,g}$ and $\set{c,e,f}$ to $B$.

We have so far $B=\set{\set{a,b,c}, \set{a,d,e}, \set{a,f,g}, \set{b,d,f}, \set{b,e,g}, \set{c,d,g}, \set{c,e,f}}$.

We have now included all two-element subsets with $a$, two-element subsets with $b$, and two-element subsets with $c$
in exactly one triplet. One can verify the two-element subsets with $d$, $e$, $f$, and $g$, respectively,
are also included in exactly one triplet. Hence
\[B=\set{\set{a,b,c}, \set{a,d,e}, \set{a,f,g}, \set{b,e,f}, \set{b,g,d}, \set{c,d,f}, \set{c,e,g}}\]
is a Steiner triple system for $A$.
In fact, every Steiner triple system with 7 elements in its basic set is isomorphic to $\aprn{A, B}$.
\end{solution}

\begin{exercise}
Either of the configurations of Exercise 4, p. 59 might appropriately be called a
schedule for a round-robin tournament. Find an appropriate set of postulates for the
class of schedules for round-robin tournaments.
\end{exercise}

\begin{solution}
Let $\aprn{A,\mc T}$ be a configuration where $\mc T \in \power^3(A)$. Think of members of T as days,
members of a day as matches, and members of $A$ as players. We write $\exists 1$ for the quantifier
\textit{There exists exactly one}, which can be expressed with the usual quantifiers as we noted
on page 18. The postulates for a round-robin tournament are as follows.
\begin{enumerate}[label=(\roman*)]
    \item Every day consists of matches only, and matches are between two players.\\
        $(\forall D\in\mc T)(\forall m\in D)(\exists a,b\in A)a\neq b\land m=\set{a,b}$.
    \item Every two distinct players play exactly one match.\\
        $(\forall a,b\in A)a\neq b\implies(\exists 1\thinspace D\in\mc T)\set{a,b}\in D$.
    \item Each player plays exactly one match on each day.\\
        $(\forall a\in A)(\forall D\in\mc T)(\exists 1\thinspace b\in A)\set{a,b}\in D$.
\end{enumerate}
\end{solution}

\begin{exercise}
Using the fundamental relation, ``$x$ is a parent of $y$'', try to write a system of
postulates for genealogy which reflects the actual biological facts. (This is not easy.)
\end{exercise}

\begin{solution}
The obvious first postulate is that every person has exactly two parents, one a man
and the other a woman. To rule out the possibility that anyone is his own parent, his own
grandparent, his own great-grandparent, etc., we need a much more sophisticated postulate.
If we attempt this directly with quantifiers having domain P, we will need infinitely
many postulates. However, this difficulty can be avoided by considering that people can
be ordered by age. This leads to the following definition.

A genealogy is a configuration $\aprn{P,\aprn{M, W,\phi}}$ where
\begin{enumerate}[label=(\roman*)]
    \item $M$ and $W$ are complementary subsets of $P$.
    \item $\phi$ is a subset of $P\times P$. (We shall write $\phi(p,q)$ instead of $\aprn{p,q}\in\phi$.)
    \item $(\forall p\in P)(\exists m\in M)(\exists w\in W)$
        \[\phi(m, p),\quad \phi(w,p),\quad\text{and}\quad (\forall q\in P)\phi(q,p)\implies(q=m\lor q=w).\]
    \item There exists an ordering of the set $P$ (i.e., a subset $B$ of $P \times P$ such that $\aprn{P,B}$
        is an ordered set in the sense of page 59) such that $(\forall p,q\in P)\phi(p,q)\implies\aprn{p,q}\in B$
        (more concisely, $\phi\sse B$).
\end{enumerate}
Note that the last postulate involves quantification over $\power(P\times P)$ instead of merely
over $P$.
\end{solution}
\section{Consistency}

\section{The classification problem}
\begin{exercise}
Show that if we adjoin the postulate,
\[\text{$A$ has seven members},\]
to those for a Steiner triple system (Exercise 2, p. 62), the resulting postulate system is
categorical.
\end{exercise}

\begin{solution}
We can letter the members of some triple in the system $\set{a, b, c}$. Choose a fourth
element and call it $d$. The system must contain triples
$\set{a,d,\enspace}$, $\set{b,d,\enspace}$, and $\set{c,d,\enspace}$.
The blanks must be filled with new names, and we can take them as $e$, $f$, and $g$, respectively.
Since there are only seven elements altogether and $a$ must be involved in a triple
with $f$ and a triple with $g$, we must have the triple $\set{a, f, g}$. Similarly, the triples $\set{b,e,g}$
and $\set{c,e,f}$ are forced. Thus we see that every Steiner triple system with seven members
is isomorphic to the system
\[\aprn{\set{a,b,c,d,e,f,g},\set{\set{a,b,c},\set{a,d,e},\set{b,d,f},\set{c,d,g},\set{a,f,g},\set{b,e,g},\set{c,e,f}}}.\]
\end{solution}

\begin{exercise}
Show that if we adjoin the postulate,
\[\text{$A$ has six members},\]
to those for a schedule for a round-robin tournament (Exercise 3, p. 62), the resulting
postulate system is categorical.
\end{exercise}

\begin{solution}
Choose any two days.
The players can be so lettered that these days are $\set{\set{a, b},\set{c, d},\set{e, f}}$
and $\set{\set{a, c},\set{b, e},\set{d, f}}$.
Some third day must contain the match
$\set{a, d}$, and on this day $e$ must play against $c$ since he cannot play $a$ or $d$ and
he has already played $b$ and $f$.
This day must be $\set{\set{a,d},\set{c,e},\set{b,f}}$.
Similarly, the days containing
the matches $\set{a, e}$ and $\set{a, f}$ are uniquely determined.
Thus all five days are determined.
It is important to check that the resulting set of five days is indeed a round-robin
tournament. A round-robin tournament can be found for any even number of players.
\end{solution}

\begin{exercise}
Classify configurations $\aprn{A, R}$, where $A$ has two elements and $R$ is a binary relation
in $A$.
\end{exercise}

\begin{solution}
If $A=\set{a,b}$ has 2 elements, then $A\times A$ has 4 elements and there are $2^4=16$ subsets
of $A\times A$. So there are 16 possible configurations. They fall into 10 classes under isomorphism.

One is the empty set.
Two are isomorphic to $\set{\aprn{a,a}}$.
Two are isomorphic to $\set{\aprn{a,b}}$.
One is $\set{\aprn{a,b},\aprn{b,a}}$.
One is $\set{\aprn{a,a},\aprn{b,b}}$.
Two are isomorphic to $\set{\aprn{a,a},\aprn{b,b},\set{a,b}}$.
Two are isomorphic to $\set{\aprn{a,a},\aprn{a,b}}$.
Two are isomorphic to $\set{\aprn{a,a},\aprn{b,a}}$.
Two are isomorphic to $\set{\aprn{a,a},\aprn{b,a},\aprn{a,b}}$.
One is $A\times A$.
\end{solution}

\begin{exercise}
Classify configurations $\aprn{A,B}$, where $B \in\power^2(A)$ and
\begin{enumerate}[label=(\alph*)]
    \item $A$ has five members,
    \item $B$ has three members,
    \item every member of $B$ has three members.
\end{enumerate}
\end{exercise}

\begin{solution}
For a fixed basic set $\set{1,2,3,4,5}$, there are 120 structures satisfying the given conditions.
Of these
\begin{center}
    20 are isomorphic to $\set{\set{1,2,3},\set{2,3,4},\set{3,4,1}}$,\\
    60 are isomorphic to $\set{\set{1,2,3},\set{2,3,4},\set{3,4,5}}$,\\
    30 are isomorphic to $\set{\set{1,2,3},\set{3,4,5},\set{1,2,4}}$,\\
    10 are isomorphic to $\set{\set{1,2,3},\set{1,2,4},\set{1,2,5}}$.
\end{center}
\end{solution}

\chapter{Equivalence}
\section{Equivalence relations and partitions}

\begin{exercise}
    Consider the relation, ``is a brother of.'' Is it reflexive? symmetric? transitive?
\end{exercise}

\begin{solution}
It is not reflexive, since you would not say that Bob is a brother of himself.

It is not symmetric, since Bob being a brother of Sarah does not imply Sarah is a brother of Bob.

It is not transitive, since Bob could have the same father but not the same mother as Joe,
and Jack could have the same mother but not the same father as Joe. Then Bob is a brother of Joe,
and Joe is a brother of Jack, but Bob is not a brother of Jack.
\end{solution}

\begin{exercise}
Give an example of a relation in a set which is symmetric and transitive but not
reflexive.
\end{exercise}

\begin{solution}
If $A=\set{1,2}$, then $R=\set{\aprn{1,1}}$ is symmetric and transitive but not reflexive.

In general, if $A$ is not void, then $\eset$ is a relation in $A$ which is both
symmetric and transitive but not reflexive.
\end{solution}

\begin{exercise}
Let $E$ be an equivalence relation in the set $A$ and let $\phi$ be the corresponding quotient
map. Prove that
\[a\Erel b\iff \phi(a)\cap\phi(b)\neq\eset.\]
\end{exercise}

\begin{solution}
Suppose $a\Erel b$. Then $\phi(a)=\phi(b)$ by Proposition 5-1.3. Since $a\in\phi(a)=\phi(a)\cap\phi(b)$,
we have $\phi(a)\cap\phi(b)\neq\eset$.
Conversely, suppose $\phi(a)\cap\phi(b)\neq\eset$. Choose $x\in \phi(a)\cap\phi(b)$.
Then $a\Erel x$ and $b\Erel x$, so $a\Erel b$.
\end{solution}

\begin{exercise}
List the partitions of the set $\set{1, 2,3}$. These fall into equivalence classes under the
relation of isomorphism (for configurations consisting of a basic set and a partition of it).
Exhibit this division into classes.
\end{exercise}

\begin{solution}
There are five partitions of $\set{1,2,3}$, shown here grouped into three isomorphism classes:
\[
\set{\set{1,2,3}}\qquad
\set{\set{1},\set{2},\set{3}}\qquad
\set{\set{1},\set{2,3}},\thickspace\set{\set{2},\set{1,3}},\thickspace\set{\set{3},\set{2,1}}
\]
\end{solution}

\begin{exercise}
Write out the omitted parts of the proof of Theorem 5-1.5 in the detailed fashion
of Section 2-5.
\end{exercise}

\begin{solution}
To give a formal proof we must translate the hypotheses into formally quantified statements.
$\mc P$ \textit{is a partition of} $A$ becomes
\begin{enumerate}[label=(\arabic*)]
    \item $\mc P\sse \power(A)-\set{\eset}$ and
    \item $(\forall x\in A)(\exists Y\in\mc P)(x\in Y\land (\forall Z\in\mc P)x\in Z\implies Z=Y)$.
\end{enumerate}
The definition of $F$ is
\begin{enumerate}[label=(\arabic*)]
    \setcounter{enumi}{2}
    \item $F\sse A\times A$ and
    \item $(\forall x,y\in A)(x\mathrel{F}y \iff (\exists Z\in\mc P)x\in Z\land y\in Z)$.
\end{enumerate}
What we want to prove is
\begin{enumerate}[label=(\arabic*)]
    \setcounter{enumi}{4}
    \item $F\sse A\times A$,
    \item $(\forall x\in A)x\mathrel F x$,
    \item $(\forall x, y\in A)x\mathrel F y \implies y \mathrel F x$, and
    \item $(\forall x,y,z\in A)(x\mathrel F y\land y\mathrel F z)\implies x\mathrel F z$.
\end{enumerate}
The proof of (5) is trivial since (5) is (3). Let us prove (8).
\begin{enumerate}[label=(\arabic*)]
    \setcounter{enumi}{8}
    \item Let $a$, $b$, $c$ be given in $A$.
    \item \quad Assume $a\mathrel F b$ and $b\mathrel F c$.
    \item \quad\quad $a\mathrel F b\iff (\exists Z\in\mc P)a\in Z\land b\in Z$ \hfill by (4)
    \item \quad\quad $(\exists Z\in\mc P)a\in Z\land b\in Z$ \hfill by (10) and (11)
    \item \quad\quad Choose $D\in\mc P$ so that $a\in D\land b\in D$ \hfill by (12)
    \item \quad\quad Choose $E\in\mc P$ so that $b\in E\land c\in E$ \hfill Similar to (11), (12), (13)
    \item \quad\quad $(\exists Y\in\mc P)(b\in Y\land (\forall Z\in\mc P)b\in Z\implies Z=Y)$ \hfill by (2)
    \item \quad\quad Choose $G\in\mc P$ so that $b\in G$ and $(\forall Z\in\mc P)b\in Z\implies Z=G$ \hfill by (15)
    \item \quad\quad $b\in D \implies D=G$ \hfill by (16)
    \item \quad\quad $D=G$ \hfill by (13) and (17)
    \item \quad\quad $E=G$ \hfill Similar to the derivation of (18)
    \item \quad\quad $c\in D$ \hfill by (18), (19), and (14)
    \item \quad\quad $(\exists Z\in\mc P)a\in Z\land c\in Z$ \hfill by (20) and (13)
    \item \quad\quad $a\mathrel F c \iff (\exists Z\in\mc P) a\in Z\land c\in Z$ \hfill by (4)
    \item \quad\quad $a\mathrel F c$ \hfill by (22) and (21)
    \item \quad $a\mathrel F b\land b\mathrel F c \implies a\mathrel F c$ \hfill by (10) through (23)
    \item $(\forall x,y,z\in A)(x\mathrel F y\land y\mathrel F z)\implies x\mathrel F z$ \hfill by (9) through (24)
\end{enumerate}
Now let us prove (6).
\begin{enumerate}[label=(\arabic*)]
    \setcounter{enumi}{25}
    \item Let $a$ be given in $A$.
    \item \quad $a\mathrel F a \iff (\exists Z\in\mc P)a\in Z\land a\in Z)$ \hfill by (4)
    \item \quad $(\exists Y\in\mc P)(a\in Y\land (\forall Z\in\mc P)a\in Z\implies Z=Y)$ \hfill by (2)
    \item \quad Choose $G\in\mc P$ so that $a\in G\land (\forall Z\in\mc P)a\in Z\implies Z=G)$ \hfill by (28)
    \item \quad $(\exists Z\in\mc P)a\in Z\land a\in Z)$ \hfill by (29)
    \item \quad $a\mathrel F a$ \hfill by (30) and (27)
    \item $(\forall x\in A)x\mathrel F x$ \hfill by (26) through (31)
\end{enumerate}
Now let us prove (7).
\begin{enumerate}[label=(\arabic*)]
    \setcounter{enumi}{32}
    \item Let $a,b$ be given in $A$.
    \item \quad Assume $a\mathrel F b$.
    \item \quad\quad $a\mathrel F b\iff (\exists Z\in\mc P)a\in Z\land b\in Z$ \hfill by (4)
    \item \quad\quad $b\mathrel F a\iff (\exists Z\in\mc P)b\in Z\land a\in Z$ \hfill by (4)
    \item \quad\quad $(\exists Z\in\mc P)a\in Z\land b\in Z$ \hfill by (34) and (35)
    \item \quad\quad Choose $G\in\mc P$ so that $a\in G\land b\in G$ \hfill by (37)
    \item \quad\quad $(\exists Z\in\mc P)b\in Z\land a\in Z$ \hfill by (38)
    \item \quad\quad $b\mathrel F a$ \hfill by (39) and (36)
    \item \quad $a\mathrel F b \implies b\mathrel F a$ \hfill by (34) through (40)
    \item $(\forall x, y\in A)x\mathrel F y \implies y \mathrel F x$ \hfill by (33) through (41)
\end{enumerate}
\end{solution}

\begin{exercise}
Choose a definite integer $m$. For integers $a$ and $b$ let ``$a\equiv b$'' mean
``$a - b$ is divisible by $m$ (that is, there is an integer $x$ such that $a-b=mx$)''.
Prove that $\equiv$ is an equivalence relation. How many equivalence classes are there?
\end{exercise}

\begin{solution}
Since $a-a=0m$, we have $a\equiv a$. Thus $\equiv$ is reflexive.

Suppose $a\equiv b$. Then $a-b=xm$ for some integer $x$. But then $b-a=(-x)m$,
so $b\equiv a$. Thus $\equiv$ is symmetric.

Suppose $a\equiv b$ and $b\equiv c$. Then there exist integers $x_1,x_2$ such that
$a-b=x_1m$ and $b-c=x_2m$.
Then $a-c=(a-b)+(b-c)=x_1m+x_2m=(x_1+x_2)m$, so $a\equiv c$. Thus $\equiv$ is transitive.

There are $\abs{m}$ equivalence classes if $m\neq 0$. If $m=0$, then each integer is
the sole element of its equivalence class.
\end{solution}
\section{Factoring functions}

\begin{exercise}
Suppose that $f$ is a function with domain $A$. Prove that
\[\setb{\aprn{x,y}}{f(x)=f(y)}\]
is an equivalence relation in $A$. Let $\phi$ be the corresponding quotient map.
If $f = g\circ \phi$,
as in 5-2.1, prove that $g$ is injective.
\end{exercise}

\begin{solution}
Let $E=\setb{\aprn{x,y}}{f(x)=f(y)}$. Let $a,b,c\in A$ be given. Then $f(a)=f(a)$, so $a\Erel a$ and $E$
is reflexive.
If $a\Erel b$, then $f(a)=f(b)$. But then $f(b)=f(a)$ and so $b\Erel a$. Thus $E$ is symmetric.
Finally, if $a\Erel b$ and $b\Erel c$, then $f(a)=f(b)=f(c)$, so $a\Erel c$. Thus $E$ is transitive.

Suppose $f=g\circ \phi$ as in 5-2.1. Assume $g(q_1)=g(q_2)$, where $q_1$ and $q_2$ are equivalence
classes of $E$. Then $q_1=\phi(a_1)$ and $q_2=\phi(a_2)$ for some $a_1,a_2\in A$.
Then $f(a_1)=g(\phi(a_1))=g(\phi(a_2))=f(a_2)$, which means $a_1\Erel a_2$.
Then by Proposition 5-1.3 we have $q_1=\phi(a_1)=\phi(a_2)=q_2$. Therefore $g$ is injective.
\end{solution}

\begin{exercise}
Let $D$ and $E$ be equivalence relations in the sets $A$ and $B$, respectively.
Let the corresponding quotient maps and sets be $\phi$, $\psi$, $\mc Q$, and $\mc R$.
\begin{enumerate}[label=(\alph*)]
    \item Define an equivalence relation $F$ in $A\times B$ in terms of $D$ and $E$ so
        that the corresponding quotient set $\mc S$ has a natural bijection to $\mc Q\times\mc R$.
    \item Suppose that $g$ is a function from $A\times B$ to a set $T$ such that
        \[(\forall a_1,a_2\in A)(\forall b\in B)\quad a_2\mathrel{D} a_2\implies g(a_1,b)=g(a_2,b)\]
        and
        \[(\forall a\in A)(\forall b_1,b_2\in B)\quad b_1\mathrel E b_2\implies g(a,b_1)=g(a,b_2).\]
        Prove that there exists a unique function $h$ from $\mc Q\times \mc R$ to $T$ such that
        \[(\forall a\in A)(\forall b\in B)\quad g(a,b)=h(\phi(a),\psi(b)).\]
\end{enumerate}
\end{exercise}

\begin{solution}
We solve (a).

Define $F$ to be $\setb{\aprn{\aprn{a_1,b_1},\aprn{a_2,b_2}}}{\aprn{a_1,a_2}\in D\land \aprn{b_1,b_2}\in E}$.

We show that $F$ is reflexive. For any $\aprn{a,b}\in A\times B$, we have $a\mathrel D a$ and $b\mathrel E b$
since $D$ and $E$ are reflexive. Hence $\aprn{a,b}\mathrel F \aprn{a,b}$.

We show that $F$ is symmetric. Suppose
$\aprn{a_1,b_1}\mathrel F \aprn{a_2,b_2}$. Then $a_1\mathrel D a_2$ and $b_1\mathrel E b_2$.
Then since $D$ and $E$ are symmetric, we have $a_2\mathrel D a_1$ and $b_2\mathrel E b_1$.
Hence $\aprn{a_2,b_2}\mathrel F \aprn{a_1,b_1}$.

We show that $F$ is transitive. Suppose
$\aprn{a_1,b_1}\mathrel F \aprn{a_2,b_2}$ and $\aprn{a_2,b_2}\mathrel F \aprn{a_3,b_3}$.
Then $a_1\mathrel D a_2$, $b_1\mathrel E b_2$, $a_2\mathrel D a_3$, and $b_2\mathrel E b_3$.
Then since $D$ and $E$ are transitive, we have $a_1\mathrel D a_3$ and $b_1\mathrel E b_3$.
Hence $\aprn{a_1,b_1}\mathrel F \aprn{a_3,b_3}$.

Let $\lambda$ be the quotient map of $F$.
Let $H=\setb{\aprn{\lambda(\aprn{a,b}),\aprn{\phi(a),\psi(b)}}}{a\in A\land b\in B}$.
We show that $H$ is a bijection
from $\mc S$ to $\mc Q\times \mc R$.

First we show that $H$ is a function. Suppose $\aprn{\lambda(\aprn{a,b}),\aprn{\phi(a),\psi(b)}}$ and
$\aprn{\lambda(\aprn{a_2,b_2}),\aprn{\phi(a_2),\psi(b_2)}}$ are members of $H$ and $\lambda(\aprn{a,b})=\lambda(\aprn{a_2,b_2})$.
Then since $\lambda$ is the quotient map of $F$, we have $\aprn{a,b}\mathrel F \aprn{a_2,b_2}$.
That means $a\mathrel D a_2$ and $b\mathrel E b_2$, which means $\phi(a)=\phi(a_2)$ and $\psi(b)=\psi(b_2)$.
Hence $\aprn{\phi(a),\psi(b)}=\aprn{\phi(a_2),\psi(b_2)}$. Thus $H$ is a function.

The domain of $H$ is clearly $\ran\lambda=\mc S$. And the range of $H$ is clearly $\mc Q\times \mc R$, i.e.
$H$ is surjective on $\mc Q\times \mc R$.

It remains to prove that $H$ is injective.
Suppose $H(\lambda(\aprn{a,b}))=H(\lambda(\aprn{a_2,b_2}))$.
Then $\aprn{\phi(a),\psi(b)}=\aprn{\phi(a_2),\psi(b_2)}$.
Then $\phi(a)=\phi(a_2)$ and $\psi(b)=\psi(b_2)$.
Then $a\mathrel D a_2$ and $b\mathrel E b_2$.
Then $\aprn{a,b}\mathrel F \aprn{a_2,b_2}$.
Then $\lambda(\aprn{a,b})=\lambda(\aprn{a_2,b_2})$. Hence $H$ is injective.

We solve (b).
\end{solution}

\chapter{Order}
\chapter{Mathematical Induction}
\chapter{Fields}
\chapter{The Construction of the Real Numbers}
\chapter{Complex Numbers}
\chapter{Counting and the Size of Sets}
\chapter{Limits}
\chapter{Sums and Products}
\chapter{The Topology of Metric Spaces}
\chapter{Introduction to Analytic Functions}

\end{document}