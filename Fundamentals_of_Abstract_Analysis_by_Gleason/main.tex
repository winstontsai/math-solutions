\documentclass{report}
\usepackage[utf8]{inputenc}
\usepackage[left=1in,right=1in,top=1in,bottom=1in]{geometry}
\usepackage{fancyhdr}
\pagestyle{fancy}
\fancyhf{}
\fancyhead[L]{\leftmark}
\fancyhead[R]{\rightmark}
\fancyfoot[C]{\thepage}
\usepackage{hyperref}
\hypersetup{
    colorlinks=true, %set true if you want colored links
    linktoc=all,     %set to all if you want both sections and subsections linked
    linkcolor=blue,  %choose some color if you want links to stand out
}
\usepackage{parskip,xcolor,lipsum}
\usepackage{graphicx}
\graphicspath{ {./images/} }
\usepackage[no-files]{xsim}
\xsimsetup{
    solution/print = true,
    exercise/within = {section}
}
\usepackage{my_macros}

% @@@@@@@@@@@@@@@@@@@@@@@@@@@@@@@@@@@@@@@@@@@@@@@@@@@@@@@@@@@@@@@@@@@@@@@@@@@@@@@@@
% MACROS
% @@@@@@@@@@@@@@@@@@@@@@@@@@@@@@@@@@@@@@@@@@@@@@@@@@@@@@@@@@@@@@@@@@@@@@@@@@@@@@@@@
\renewcommand{\thesection}{\thechapter-\arabic{section}}

% @@@@@@@@@@@@@@@@@@@@@@@@@@@@@@@@@@@@@@@@@@@@@@@@@@@@@@@@@@@@@@@@@@@@@@@@@@@@@@@@@
% TITLE PAGE
% @@@@@@@@@@@@@@@@@@@@@@@@@@@@@@@@@@@@@@@@@@@@@@@@@@@@@@@@@@@@@@@@@@@@@@@@@@@@@@@@@
\begin{document}
\begin{titlepage}
  \begin{center}
    \vspace*{1cm}
    \LARGE
    Solutions to

    \vspace{0.5cm}

    \Huge
    \textbf{Fundamentals of Abstract Analysis}

    \huge
    Book Subtitle

    \LARGE
    \textbf{by Andrew M. Gleason}

    \vspace{0.2cm}

    \large
    ISBN 9780867202090

    \vfill

    \LARGE{Winston Tsai}
  \end{center}
\end{titlepage}
\newpage

\section*{Preface}
Gleason provides hints and solutions to most of the
exercises at the end of the book.
Those solutions which I feel are completely adequate will still be repeated here.


\newpage
\tableofcontents
\newpage

% @@@@@@@@@@@@@@@@@@@@@@@@@@@@@@@@@@@@@@@@@@@@@@@@@@@@@@@@@@@@@@@@@@@@@@@@@@@@@@@@@
% CHAPTER 1
% @@@@@@@@@@@@@@@@@@@@@@@@@@@@@@@@@@@@@@@@@@@@@@@@@@@@@@@@@@@@@@@@@@@@@@@@@@@@@@@@@
\chapter{Sets}

\section{The notion of set}

\begin{exercise}
Which of the following are bona fide sets?
\begin{itemize}
    \item The set of all even integers.
    \item The set of all large integers.
    \item The set of all irrational numbers which are square roots of integers.
    \item The set interesting numbers.
    \item The set of points in a given euclidean plane.
    \item The set of points in a given noneuclidean plane.
\end{itemize}
\end{exercise}

\begin{solution}
Per Definition 1-1.1, a set must be sufficiently well-defined. 
The phrases "large integer" and "interesting number" are much too vague. The others are all right.
\end{solution}


\begin{exercise}
Was the population of the United States at noon (EST) on January 1, 1960, an odd
or an even number? Is this question meaningful? Is the set of inhabitants of the United
States at that time a well-defined set?
\end{exercise}

\begin{solution}
We have no definition of ``inhabitant of the United States'' complete enough to
cover unambiguously all cases which arise in practice. Hence the set of all such
inhabitants is not well-defined in a mathematical sense.
\end{solution}


\section{Equality}
\section{Parentheses}
\section{Membership}
\section{The empty set}
\section{The list notation}

\begin{exercise}
List the sixteen sets that can be formed from $\eset$, using braces not more than three deep.
\end{exercise}

\begin{solution}
Here are the sets with braces at most one deep.
\begin{itemize}
    \item $\eset$.
    \item $\set{\eset}$.
\end{itemize}

Here are the sets with braces at most two deep.
\begin{itemize}
    \item $\eset$.
    \item $\set{\eset}$.
    \item $\set{\eset, \set{\eset}}$ and $\set{\set{\eset}}$.
\end{itemize}

Here are the sets with braces at most three deep.
\begin{itemize}
    \item $\eset$.
    \item $\set{\eset}$.
    \item $\set{\eset, \set{\eset}}$ and $\set{\set{\eset}}$.
    \item $\set{\set{\eset, \set{\eset}}}$ and $\set{\set{\set{\eset}}}$ and
    $\set{\set{\eset, \set{\eset}}, \set{\set{\eset}}}$.
    \item $\set{\set{\eset, \set{\eset}}, \set{\eset}}$ and $\set{\set{\set{\eset}}, \set{\eset}}$ and
    $\set{\set{\eset, \set{\eset}}, \set{\set{\eset}}, \set{\eset}}$.
    \item $\set{\set{\eset, \set{\eset}}, \eset}$ and $\set{\set{\set{\eset}}, \eset}$ and
    $\set{\set{\eset, \set{\eset}}, \set{\set{\eset}}, \eset}$.
    \item $\set{\set{\eset, \set{\eset}}, \set{\eset}, \eset}$ and $\set{\set{\set{\eset}}, \set{\eset}, \eset}$ and
    $\set{\set{\eset, \set{\eset}}, \set{\set{\eset}}, \set{\eset}, \eset}$.
\end{itemize}

\end{solution}

\section{Set inclusion}

\end{document}