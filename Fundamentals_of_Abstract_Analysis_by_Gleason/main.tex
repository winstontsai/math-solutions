\documentclass{report}
\usepackage[utf8]{inputenc}
\usepackage[left=1in,right=1in,top=1in,bottom=1in]{geometry}
\usepackage{fancyhdr}
\pagestyle{fancy}
\fancyhf{}
\fancyhead[L]{\leftmark}
\fancyhead[R]{\rightmark}
\fancyfoot[C]{\thepage}
\usepackage{hyperref}
\hypersetup{
    colorlinks=true, %set true if you want colored links
    linktoc=all,     %set to all if you want both sections and subsections linked
    linkcolor=blue,  %choose some color if you want links to stand out
}
\usepackage{graphicx}
\graphicspath{ {./images/} }
\usepackage{enumitem}
\setlist{nosep}
\usepackage{tasks}
\usepackage{parskip,xcolor,array,etoolbox,multicol}
\usepackage{my_macros}

% @@@@@@@@@@@@@@@@@@@@@@@@@@@@@@@@@@@@@@@@@@@@@@@@@@@@@@@@@@@@@@@@@@@@@@@@@@@@@@@@@
% MACROS
% @@@@@@@@@@@@@@@@@@@@@@@@@@@@@@@@@@@@@@@@@@@@@@@@@@@@@@@@@@@@@@@@@@@@@@@@@@@@@@@@@
\renewcommand*{\thesection}{\thechapter-\arabic{section}}
\renewcommand*{\land}{\text{ and }}
\renewcommand*{\lor}{\text{ or }}
\renewcommand*{\lnot}{\text{not }}
\newcommand*{\lnotd}{\text{not-}}
\newcommand*{\comp}[1]{\widetilde{#1}}

\newcounter{magicrownumbers}
\preto\array{\setcounter{magicrownumbers}{0}}
\preto\tabular{\setcounter{magicrownumbers}{0}}
\def\rownumber{}

% @@@@@@@@@@@@@@@@@@@@@@@@@@@@@@@@@@@@@@@@@@@@@@@@@@@@@@@@@@@@@@@@@@@@@@@@@@@@@@@@@
% TITLE PAGE
% @@@@@@@@@@@@@@@@@@@@@@@@@@@@@@@@@@@@@@@@@@@@@@@@@@@@@@@@@@@@@@@@@@@@@@@@@@@@@@@@@
\begin{document}
\begin{titlepage}
  \begin{center}
    \vspace*{1cm}
    \LARGE
    Solutions to

    \vspace{0.5cm}

    \Huge
    \textbf{Fundamentals of Abstract Analysis}

    \huge
    Book Subtitle

    \LARGE
    \textbf{by Andrew M. Gleason}

    \vspace{0.2cm}

    \large
    ISBN 9780867202090

    \vfill

    \LARGE{Winston Tsai}
  \end{center}
\end{titlepage}
\newpage

\section*{Preface}
Gleason provides hints and solutions to most of the
exercises at the end of the book.
Those solutions which I feel are completely adequate will still be repeated here.


\newpage
\tableofcontents
\newpage

% @@@@@@@@@@@@@@@@@@@@@@@@@@@@@@@@@@@@@@@@@@@@@@@@@@@@@@@@@@@@@@@@@@@@@@@@@@@@@@@@@
% CHAPTER 1
% @@@@@@@@@@@@@@@@@@@@@@@@@@@@@@@@@@@@@@@@@@@@@@@@@@@@@@@@@@@@@@@@@@@@@@@@@@@@@@@@@
\chapter{Sets}

\section{The notion of set}

\begin{exercise}
Which of the following are bona fide sets?
\begin{itemize}
    \item The set of all even integers.
    \item The set of all large integers.
    \item The set of all irrational numbers which are square roots of integers.
    \item The set interesting numbers.
    \item The set of points in a given euclidean plane.
    \item The set of points in a given noneuclidean plane.
\end{itemize}
\end{exercise}

\begin{solution}
Per Definition 1-1.1, a set must be sufficiently well-defined. 
The phrases "large integer" and "interesting number" are much too vague. The others are all right.
\end{solution}


\begin{exercise}
Was the population of the United States at noon (EST) on January 1, 1960, an odd
or an even number? Is this question meaningful? Is the set of inhabitants of the United
States at that time a well-defined set?
\end{exercise}

\begin{solution}
We have no definition of ``inhabitant of the United States'' complete enough to
cover unambiguously all cases which arise in practice. Hence the set of all such
inhabitants is not well-defined in a mathematical sense.
\end{solution}


\section{Equality}
\section{Parentheses}
\section{Membership}
\section{The empty set}
\section{The list notation}

\begin{exercise}
List the sixteen sets that can be formed from $\eset$, using braces not more than three deep.
\end{exercise}

\begin{solution}
Here are the sets with braces at most one deep.
\begin{itemize}
    \item $\eset$.
    \item $\set{\eset}$.
\end{itemize}

Here are the sets with braces at most two deep.
\begin{itemize}
    \item $\eset$.
    \item $\set{\eset}$.
    \item $\set{\eset, \set{\eset}}$ and $\set{\set{\eset}}$.
\end{itemize}

Here are the sets with braces at most three deep.
\begin{itemize}
    \item $\eset$.
    \item $\set{\eset}$.
    \item $\set{\eset, \set{\eset}}$ and $\set{\set{\eset}}$.
    \item $\set{\set{\eset, \set{\eset}}}$ and $\set{\set{\set{\eset}}}$ and
    $\set{\set{\eset, \set{\eset}}, \set{\set{\eset}}}$.
    \item $\set{\set{\eset, \set{\eset}}, \set{\eset}}$ and $\set{\set{\set{\eset}}, \set{\eset}}$ and
    $\set{\set{\eset, \set{\eset}}, \set{\set{\eset}}, \set{\eset}}$.
    \item $\set{\set{\eset, \set{\eset}}, \eset}$ and $\set{\set{\set{\eset}}, \eset}$ and
    $\set{\set{\eset, \set{\eset}}, \set{\set{\eset}}, \eset}$.
    \item $\set{\set{\eset, \set{\eset}}, \set{\eset}, \eset}$ and $\set{\set{\set{\eset}}, \set{\eset}, \eset}$ and
    $\set{\set{\eset, \set{\eset}}, \set{\set{\eset}}, \set{\eset}, \eset}$.
\end{itemize}

\end{solution}
\section{Set inclusion}


\chapter{Logic}
\section{Propositions and logical connectives}
\section{Tautologies}
\section{The conditional}
\begin{exercise}
Assume that the following propositions are true: $x\in A$, $x\nin B$, $x\in C$, $y\in A$,
$y\in B$, $y\nin C$, $z\nin A$, $z\nin B$, and $z\in C$. Which of the following are true?
\begin{enumerate}[label=(\alph*)]
    \item $y\nin B\lor z\in A$.
    \item $x\nin C\land y\in B$.
    \item $x\in C$ implies $z\in A$.
    \item $(x\in A\lor y\in B)$ implies $z\nin C$.
    \item $x\nin A$ if and only if $y\in C$.
    \item If $x\in A$ implies $z\in C$, then $y\nin A$ implies $z\in B$.
    \item If $z\in A$ implies $x\in B$, then $y\in B$.
    \item If $x\nin B$, then $y\in C$, $z\in B$, and $x\nin A$.
    \item $x\nin A$ only if $y\nin B$.
    \item $z\in C$ is a necessary condition for $y\in C$.
    \item $z\in C$ is a sufficient condition for $x\in A$.
    \item In order that $z\nin C$, it is necessary and sufficient that $x\in A$ imply $y\in B$.
\end{enumerate}
\end{exercise}

\begin{solution}
We can replace each basic (non)membership proposition with $T$ or $F$ based on
the propositions assumed to be true. That will make it easier to determine whether
each proposition as a whole is true.
\begin{enumerate}[label=(\alph*)]
    \item $F\lor F$.
    \item $F\land T$.
    \item $T$ implies $F$.
    \item ($T\lor T$) implies $F$.
    \item $F$ if and only if $F$.
    \item If $T$ implies $T$, then $F$ implies $F$.
    \item If $F$ implies $F$, then $T$.
    \item If $T$, then $F$, $F$, and $F$.
    \item $F$ only if $F$.
    \item $T$ is a necessary condition for $F$.
    \item $T$ is a sufficient condition for $T$.
    \item In order that $F$, it is necessary and sufficient that $T$ imply $T$.
\end{enumerate}
The propositions (e), (f), (g), (i), (j), (k), and (l) are true.
\end{solution}

\begin{exercise}
Suppose it is known that one of the following cases is true:

Case I. $x\in A$, $x\in B$, $x\nin C$, $y\in A$, $y\nin B$, and $y\in C$;

Case II. $x\nin A$, $x\in B$, $x\in C$, $y\nin A$, $y\nin B$, and $y\in C$.

Which of the following propositions are certainly true? Which are certainly false?

\begin{enumerate}[label=(\alph*)]
    \item $x\in C$ if and only if $y\in C$.
    \item ($x\in B\land y\in A$) if and only if ($x\in A\lor y\in B$).
    \item If $x\in A\land y\in C$, then $x\in C\lor y\in A$.
    \item ($x\in B\land y\in B$) or ($x\in C\land y\in C$).
    \item not-(if $x\in A$, then $y\in B$) implies ($x\in C\land y\in B$).
\end{enumerate}
\end{exercise}

\begin{solution}
This is like the previous exercise, except we need to determine which of the propositions is true
for each of the two cases. Then the propositions which are certainly true are those that are true for both cases,
and the propositions which are certainly false are those that are false for both cases.

For Case I, the propositions simplify to the following.

\begin{enumerate}[label=(\alph*)]
    \item $F$ if and only if $T$.
    \item ($T\land T$) if and only if ($T\lor F$).
    \item If $T\land T$, then $F\lor T$.
    \item ($T\land F$) or ($F\land T$).
    \item not-(if $T$, then $F$) implies ($F\land F$).
\end{enumerate}

For Case I, propositions (b) and (c) are true.

For Case II, the propositions simplify to the following.

\begin{enumerate}[label=(\alph*)]
    \item $T$ if and only if $T$.
    \item ($T\land F$) if and only if ($F\lor F$).
    \item If $F\land T$, then $T\lor F$.
    \item ($T\land F$) or ($T\land T$).
    \item not-(if $F$, then $F$) implies ($T\land F$).
\end{enumerate}

For Case II, propositions (a), (b), (c), (d), and (e) are true.

Therefore, knowing that Case I or Case II is true,
the propositions (b) and (c) are certainly true. None of the others are certainly false.
\end{solution}

\begin{exercise}
Suppose that the following propositions are true:
\begin{align*}
    &\text{if $x\in A\land y\in B$, then $y\in A$;}\\
    &\text{if $x\nin A\lor y\in A$, then $y\nin B$.}
\end{align*}
What simple conclusion can we draw?
\end{exercise}

\begin{solution}
We can conclude $y\nin B$. For suppose that $y\in B$. Either $x\in A\lor x\nin A$.
If $x\in A$, then the first hypothesis gives $y\in A$, from which the second hypothesis gives
$y\nin B$.
If $x\nin A$, then the second hypothesis again gives $y\nin B$.
In both cases we have a contradiction.
Therefore $y\nin B$.

In fact, if we assume $y\nin B$, then the first hypothesis is true because the antecedent is false,
and the second hypothesis is true because the consequent is true.
Therefore $y\nin B$ is true if and only if both hypotheses are true, so $y\nin B$ is the
best conclusion we can draw.

This result can also be derived with a truth table.
If $p$ stands for $x\in A$, $q$ for $y\in B$, and $r$ for $y\in A$, the hypotheses are
$(p\land q)\implies r$ and $(\lnot p\lor r)\implies\lnot q$.
We then have the following truth table.
\[
\begin{array}{ @{\makebox[2em][c]{\rownumber\space}} ccccc } 
p & q & r & (p\land q)\implies r & (\lnot p\lor r)\implies\lnot q 
\gdef\rownumber{\stepcounter{magicrownumbers}(\arabic{magicrownumbers})} \\
\hline
T & T & T & T & F \\ 
T & T & F & F & T \\ 
T & F & T & T & T \\ 
T & F & F & T & T \\ 
F & T & T & T & F \\ 
F & T & F & T & F \\ 
F & F & T & T & T \\ 
F & F & F & T & T \\ 
\end{array}
\]
If we consider only the rows for which both hypotheses are true, we are left with
exactly the rows for which $q$ is false: (3), (4), (7), and (8).
\end{solution}


\section{Propositional schemes and quantifiers}

\begin{exercise}
Express in terms of the ordinary quantifiers:
\begin{enumerate}[label=(\alph*)]
    \item There are at most two $x$'s such that $P(x)$.
    \item There are at least two $x$'s such that $P(x)$.
\end{enumerate}
\end{exercise}

\begin{solution}
\begin{enumerate}[label=(\alph*)]
    \item $(\forall x,y,z)\prn{P(x)\land P(y)\land P(z)}\implies (x=y\lor x=z\lor y=z)$
    \item $(\exists x,y)\prn{x\neq y \land P(x)\land P(y)}$
\end{enumerate}
\end{solution}

\begin{exercise}
Let $\phi(x, y)$ mean $x$ is a parent of $y$. Let $M$ be the set of all men (living or dead)
and $W$ be the set of all women.

The proposition, \textit{$a$ is the mother of $b$}, can be written \textit{$a\in W$ and $\phi(a, b)$}, while \textit{$a$ is the
brother (possibly half-brother) of $b$} can be written \textit{$a\in M$ and $a\neq b$ and $(\exists x) (\phi(x, a)$ and
$\phi(x, b))$}.

Because ordinary language is less precise than the language of quantified statements
and because there are different ways the biological facts can enter the interpretations of
the problems, the answers to the following problems are not all unique.

Express with quantifiers
\begin{enumerate}[label=(\alph*)]
    \item $a$ is the grandfather of $b$.
    \item $a$ is the grandson of $b$.
    \item $a$ is the aunt of $b$. 
    \item $a$ and $b$ are sisters.
    \item aand dare first cousins. 
    \item $a$ has neither brothers nor sisters.
    \item $a$ is the full brother of $b$.
    \item Every person has a father.
    \item Every person has at most one father.
    \item Every person has exactly two parents.
    \item Some person has at least two children.
    \item Every person has a grandfather.
    \item No one is his own grandfather.
\end{enumerate}

Let $\psi(x, y)$ mean \textit{$x$ is the ancestor of $y$}. Express $\phi(a,b)$ in terms of $\psi(x,y)$ and
quantifiers. Can you express $\psi(a, b)$ in terms of $\phi(x,y)$ and quantifiers?
\end{exercise}

\begin{solution}
\begin{enumerate}[label=(\alph*)]
    \item $a\in M\land (\exists x)\prn{\phi(a,x)\land \phi(x, b)}$
    \item $a\in M\land (\exists x)\prn{\phi(b,x)\land \phi(x, a)}$
    \item $a\in W\land (\exists x, y)\prn{\phi(x,b)\land \phi(y, x)\land \phi(y,a)\land x\neq a}$
    \item $a,b\in W\land a\neq b\land (\exists x)\prn{\phi(x, a)\land \phi(x, b)}$
    \item $a\neq b\land (\exists x, y, z)\prn{x\neq y\land \phi(x,a)\land \phi(y,b)\land \phi(z, x)\land \phi(z,y)}$
    \item $(\forall x)\prn{(\exists y)(\phi(y,x)\land \phi(y, a))\implies x = a}$
    \item $a\in M\land a\neq b\land (\exists x,y)(x\neq y\land \phi(x,a)\land \phi(x,b)\land \phi(y,a)\land \phi(y,b))$
    \item $(\forall x)(\exists y\in M)\phi(y, x)$
    \item $(\forall x)(\forall y,z\in M)(\phi(y, x)\land \phi(z,x) \implies y = z)$
    \item $(\forall x)(\exists y,z)(y\neq z\land \phi(y, x)\land \phi(z,x)
    \land (\forall w)(\phi(w,x)\implies (w=y\lor w=z)))$
    \item $(\exists x,y,z)(y\neq z\land \phi(x,y)\land \phi(x,z))$
    \item $(\forall x)(\exists z\in M)(\exists y)(\phi(y, x)\land \phi(z,y))$
    \item $(\forall x,y,z)((\phi(y, x)\land \phi(z,y)\land z\in M) \implies x\neq z)$
\end{enumerate}

$\phi(a,b)$ can be expressed as $\psi(a,b)\land \lnotd(\exists x)(\psi(a,x)\land \psi(x,b))$.

It is impossible to express $\psi(a, b)$ in terms of $\phi(x, y)$ and quantifiers.
\end{solution}


\section{Proof and inference}

\begin{exercise}
Refer to Exercise 2, Section 2-4. Prove formally that (h) implies (l).
\end{exercise}

\begin{solution}
The hypothesis is $(\forall x)(\exists y\in M)\phi(y, x)$.

\begin{enumerate}[label=(\roman*)]
    \item Let $a$ be any person.
    \item \quad $(\exists y\in M)\phi(y, a)$ \hfill by hypothesis;
    \item \quad choose $b$ so that $\phi(b,a)$ \hfill by (ii);
    \item \quad $(\exists y\in M)\phi(y, b)$ \hfill by hypothesis;
    \item \quad choose $c$ so that $c\in M$ and $\phi(c,b)$ \hfill by (iv);
    \item \quad $c\in M\land (\exists y)(\phi(c, y)\land \phi(y, a))$ \hfill by (iii) and (v);
    \item \quad $(\exists z\in M)(\exists y)(\phi(z,y)\land \phi(y, a))$ \hfill by (vi);
    \item $(\forall x)(\exists z\in M)(\exists y)(\phi(z,y)\land \phi(y, x))$ \hfill by (i) through (vii).
\end{enumerate}
\end{solution}

\section{Set formation}

\begin{exercise}
Which of the following sets are the same? What inclusion relations hold among
these sets? (All variables have domain $\R$.)

\begin{tasks}[label=](2)
    \task $A_1=\setb{x}{0 < x <2}$
    \task $A_2=\setb{x}{\abs{x} < 1}$
    \task $A_3=\setb{x}{x < x^2}$
    \task $A_4=\setb{x}{(\exists y)x=y^2}$
    \task $A_5=\setb{x}{(\exists y)x=1/y}$
    \task $A_6=\setb{x}{(\exists y)x=3y}$
    \task $A_7=\setb{x}{(\exists y,z)x=y^2 \land y+z^2 < 1}$
    \task $A_8=\setb{x}{(\forall y)\abs{x-y} = x - y}$
    \task $A_9=\setb{x}{(\forall y)\abs{x-y} = \abs{y} - \abs{x}}$
    \task $A_{10}=\setb{x}{(\exists y)y^2 + xy + x = 1}$
\end{tasks}
\end{exercise}

\begin{solution}
We have the following equalities:
\begin{tasks}[label=](2)
    \task $A_1=\setb{x}{0 < x <2}$
    \task $A_2=\setb{x}{-1 < x < 1}$
    \task $A_3=\setb{x}{x < 0}\cup \setb{x}{x > 1}$
    \task $A_4=\setb{x}{x\geq 0}$
    \task $A_5=\setb{x}{x\neq 0}$
    \task $A_6=\R$
    \task $A_7=\setb{x}{x\geq 0}$
    \task $A_8=\eset$
    \task $A_9=\set{0}$
    \task $A_{10}=\R$
\end{tasks}

So we have the following set equalities and inclusions:
\begin{itemize}
    \item $\eset = A_8 \sss A_9 \sss A_1 \sss A_4 = A_7 \sss A_6 = A_{10} = \R$
    \item $\eset \sss A_3 \sss A_5 \sss \R$
    \item $A_9 \sss A_2 \sss \R$
\end{itemize}
\end{solution}




\section{The set-theoretic paradoxes}
\section{Dummy variables}



\chapter{The Set-Theoretic Machinery}
\section{Binary set combinations}
Prove the following set-theoretic identities. Do each one twice, once using Venn
diagrams and once by formal manipulation using the identities in the text.

\begin{exercise}
$(A\cup B\cup C)\cap(B\cup D) = (A\cap D)\cup B\cup (C\cap D)$.

Write and prove the dual identity also.
\end{exercise}

\begin{solution}
Applying the distributive law twice to the left-hand side gives
\[
    (A\cup B\cup C)\cap(B\cup D)
    = (A\cap B)\cup (A\cap D)\cup (B\cap B)\cup (B\cap D)\cup (C\cap B)\cup (C\cap D).
\]
All intersections of involving $B$ are absorbed by by term $B\cap B=B$, so the result is
$(A\cap D)\cup B \cup (C\cap D)$.
\end{solution}

\begin{exercise}
$(A\cup B)\cap(B\cup C)\cap(C\cup A) = (A\cap B)\cup(B\cap C)\cup(C\cap A)$.

Show that this identity is its own dual.
\end{exercise}

\begin{solution}
Applying the distributive law three times to the left-hand side gives
\[
    (A\cap B\cap C)\cup(A\cap B\cap A)\cup(A\cap C\cap C)\cup(A\cap C\cap A)
    \cup(B\cap B\cap C)\cup(B\cap B\cap A)\cup(B\cap C\cap C)\cup(B\cap C\cap A).
\]
Removing duplicate sets from each term gives
\[
    (A\cap B\cap C)\cup(A\cap B)\cup(A\cap C)\cup(A\cap C)\cup(B\cap C)\cup(B\cap A)
    \cup(B\cap C)\cup(B\cap C\cap A).
\]
Removing duplicate terms gives
\[(A\cap B\cap C)\cup(A\cap B)\cup(A\cap C)\cup(B\cap C).\]
Finally, $A\cap B\cap C$ is absorbed by $A\cap B$ to give
\[(A\cap B)\cup(A\cap C)\cup(B\cap C) = (A\cap B)\cup(B\cap C)\cup(C\cap A).\]

To find the dual identity, interchange the $\cup$ and $\cap$ signs. That gives
the same equation but reversed, so this identity is its own dual.
\end{solution}

\begin{exercise}
$\prn{A\cup B\cup C}^{\sim}= \widetilde{A}\cap \widetilde{B}\cap \widetilde{C}$.

This is an extended form of (19). To give a formal proof from the identities in the text,
one must agree on a definite interpretation of $A\cup B\cup C$ in terms of the binary
combination $\cup$, and similarly for $\cap$.
\end{exercise}

\begin{solution} 
Let us agree that $A\cup B\cup C$ means $(A\cup B)\cup C$ and similarly for $\cap$.
We compute
\[
\begin{aligned}
    \prn{A\cup B\cup C}^{\sim} &= \prn{(A\cup B)\cup C}^{\sim} \\
    &= \prn{A\cup B}^{\sim} \cap \widetilde{C} \\
    &= \prn{\widetilde{A}\cap \widetilde{B}} \cap \widetilde{C} \\
    &= \widetilde{A}\cap \widetilde{B} \cap \widetilde{C}.
\end{aligned}
\]
\end{solution}

\begin{exercise}
$(A - B)\cup C = (A\cup C) - (B-C)$.
\end{exercise}

\begin{solution}
We compute
\[
\begin{aligned}
    (A - B)\cup C &= (A\cap \widetilde{B})\cup C \\
    &= (A\cup C)\cap(\widetilde{B}\cup C) \\
    &= (A\cup C)\cap\prn{B\cap \widetilde{C}}^{\sim} \\
    &= (A\cup C)-\prn{B - C}.
\end{aligned}
\]
\end{solution}

\begin{exercise}
If $A\cap B\cap C = \eset$, then $(A-B)\cup(B-C)\cup(C-A)=A\cup B\cup C$.
\end{exercise}

\begin{solution}
We have
\[
\begin{aligned}
    (A-B)\cup(B-C)\cup(C-A) &= (A\cap \comp{B})\cup(B\cap \comp{C})\cup(C\cap\comp{A})\\
    &= \prn{(A\cup B) \cap (A\cup\comp C)\cap (\comp B \cup B)\cap (\comp B \cup \comp C)}\cup(C\cap\comp A)\\
    &= \prn{(A\cup B\cup C) \cap (A\cup\comp C\cup C)\cap (\comp B \cup B\cup C)\cap (\comp B \cup \comp C\cup C)}\\
    & \quad \cap\prn{(A\cup B\cup\comp A) \cap (A\cup\comp C\cup\comp A)\cap (\comp B \cup B\cup\comp A)\cap (\comp B \cup \comp C\cup\comp A)}\\
\end{aligned}
\]
Since $A\cup\comp A = B\cup\comp B =C\cup\comp C= U$, the universal set, this reduces to
$(A\cup B\cup C) \cap (\comp B \cup \comp C\cup\comp A)$.
If $A\cap B\cap C=\eset$,  then $\comp A\cup\comp B\cup\comp C = \prn{A\cap B\cap C}^{\sim}=\comp\eset=U$,
so $(A\cup B\cup C) \cap (\comp B \cup \comp C\cup\comp A) = A\cup B\cup C$, as desired.
\end{solution}

\begin{exercise}
$(A-B)\cup(B-C)\cup(C-A)=(A-C)\cup(C-B)\cup(B-A)$.
\end{exercise}

\begin{solution}

\end{solution}

\begin{exercise}
$\prn{A\cup (\widetilde B\cap C)}^{\sim} = (B-A)\cup(\widetilde A-C)$.
\end{exercise}

\begin{solution}

\end{solution}

\begin{exercise}
The \textit{symmetric difference} of two sets, often denoted by $A\oplus B$, is defined
as $(A-B)\cup(B-A)$. Show that for all $A$, $B$, $C$,
\begin{gather*}
    A\oplus A=\eset, \qquad A\oplus\eset = A, \qquad A\oplus B = B\oplus A,\\
    A\oplus(B\oplus C) = (A\oplus B)\oplus C, \qquad A\cap(B\oplus C)=(A\cap B)\oplus(A\cap C).
\end{gather*}
\end{exercise}

\begin{solution}
\GetExerciseProperty{counter}
\end{solution}

\section{The power set }
\section{Ordered pairs and direct products}
\section{Functions}
\section{Relations}
\section{Indexed unions and intersections}
\section{Indexed direct products}


\chapter{Mathematical Configurations}
\chapter{Equivalence}
\chapter{Order}
\chapter{Mathematical Induction}
\chapter{Fields}
\chapter{The Construction of the Real Numbers}
\chapter{Complex Numbers}
\chapter{Counting and the Size of Sets}
\chapter{Limits}
\chapter{Sums and Products}
\chapter{The Topology of Metric Spaces}
\chapter{Introduction to Analytic Functions}

\end{document}