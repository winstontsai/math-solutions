\chapter{The Construction of the Real Numbers}


\section{The arithmetic of the natural numbers}

\begin{exercise}
Prove that $(\forall \mrm p,\mrm q,\mrm r)$ if $\mrm p + \mrm r = \mrm q + \mrm r$,
then $\mrm p=\mrm q$.
\end{exercise}

\begin{solution}
We use induction on $\mrm r$. Let $\mrm p$ and $\mrm q$ be arbitrary.
If $\mrm p + \mrm e = \mrm q + \mrm e$,
then $f(\mrm p)=f(\mrm q)$ which implies $\mrm p=\mrm q$ since $f$ is injective.
Suppose that if $\mrm p + \mrm n = \mrm q + \mrm n$, then $\mrm p=\mrm q$.
Assume $\mrm p + f(\mrm n) = \mrm q + f(\mrm n)$.
Then $f(\mrm p + \mrm n)=f(\mrm q + \mrm n)$.
Then $\mrm p + \mrm n=\mrm q + \mrm n$ since $f$ is injective.
By the induction hypothesis, we have $\mrm p=\mrm q$.
\end{solution}

\begin{exercise}
Prove that $(\forall \mrm p,\mrm q,\mrm r)\thinspace \mrm p<\mrm q$ if and only if
$\mrm p + \mrm r < \mrm q + \mrm r$.
\end{exercise}

\begin{solution}
Let $\mrm p$ and $\mrm q$ and $\mrm r$ be arbitrary.
If $\mrm p < \mrm q$, then there exists $\mrm x$ with $\mrm p + \mrm x = \mrm q$.
Then
\[(\mrm p + \mrm r) + \mrm x=(\mrm p + \mrm x) + \mrm r = \mrm q + \mrm r.\]
Therefore $\mrm p + \mrm r < \mrm q + \mrm r$.
Conversely, if $\mrm p + \mrm r < \mrm q + \mrm r$, then it cannot be the case that
$\mrm q \leq \mrm p$ by what we just showed. Thus $\mrm p < \mrm q$.
\end{solution}

\begin{exercise}
Show that the ordering of 9-1.2 is the only ordering of $\mc S$ for which
$(\forall \mrm s)\thinspace \mrm s < f(\mrm s)$.
\end{exercise}

\begin{solution}
Assume that $\prec$ is a strong ordering of $\mc S$ for which
\[(\forall \mrm s)\thinspace \mrm s \prec f(\mrm s).\tag{1}\]
We use induction on $\mrm r$ to prove that $\mrm p\prec \mrm p+\mrm r$ for all $\mrm r$.
By (1), we have $\mrm p \prec f(\mrm p)=\mrm p + \mrm e$.
Suppose $\mrm p \prec \mrm p + \mrm n$.
We have $\mrm p + \mrm n\prec f(\mrm p + \mrm n)=\mrm p + f(\mrm n)$. By transitivity, $\mrm p\prec \mrm p + f(\mrm n)$.
This completes the induction.
Now if $\mrm p < \mrm q$, choose $\mrm r$ so that $\mrm p + \mrm r = \mrm q$.
Then $\mrm p\prec \mrm q$.
Conversely, suppose $\mrm p\prec \mrm q$. Since $\prec$ is strong, $\mrm p\neq\mrm q$.
Therefore, either $\mrm p < \mrm q$ or $\mrm q < \mrm p$.
By what we have just proved, $\mrm q < \mrm p$ implies $\mrm q\prec\mrm p$, a contradiction.
Thus $\mrm p\prec \mrm q$ implies $\mrm p < \mrm q$.
\end{solution}

\begin{exercise}
Show that $\mc S$ is well-ordered by the ordering of 9-1.2.
\end{exercise}

\begin{solution}
First we prove that $\mrm e\leq\mrm r$ for all $\mrm r$ by induction on $\mrm r$.
Clearly $\mrm e\leq\mrm e$. Suppose $\mrm e\leq\mrm n$.
Since $\mrm n < f(\mrm n)$, we have $\mrm e \leq \mrm f(\mrm n)$. This completes the induction.

Next we prove that if $\mrm x < f(\mrm y)$, then $\mrm x \leq \mrm y$.
Let $\mrm y$ be arbitrary.
We always have $\mrm e \leq \mrm y$.
Suppose that if $\mrm n < f(\mrm y)$, then $\mrm n \leq \mrm y$.
Assume $f(\mrm n) < f(\mrm y)$. Then from \hyperref[ex:9-1.2]{Exercise 2} we have $\mrm n < \mrm y$.
Then $\mrm y=\mrm n+\mrm r\geq \mrm n + \mrm e=f(\mrm n)$ for some $\mrm r$.

Let $T$ be a subset of $\mc S$ with no least element. We show that $T$ is empty.
We use induction on $\mrm r$ to prove that $(\forall \mrm r,\mrm s)\thinspace \mrm s < \mrm r \implies \mrm s \nin T$.
We have $\mrm s\nless \mrm e$ for all $\mrm s$ since $\mrm e\leq \mrm s$ for all $\mrm s$.
Suppose $\mrm s < \mrm n \implies \mrm s \nin T$ for all $\mrm s$.
Let $\mrm s$ be arbitrary and assume $\mrm s < f(\mrm n)$.
Then $\mrm s\leq \mrm n$. If $s\in T$, then we must have $\mrm s\geq\mrm n$, in which case $\mrm s = \mrm n$.
But then $\mrm s$ is the least element of $T$, a contradiction.
Therefore $s\nin T$. This completes the induction.

We have shown that $(\forall \mrm r,\mrm s)\thinspace \mrm s < \mrm r \implies \mrm s \nin T$.
Now for any $\mrm x$, we have $\mrm x < f(\mrm x)$ and hence $\mrm x\nin T$.
Therefore $T$ is empty.
\end{solution}


\begin{exercise}
For each $\mrm p$ let $\mc{S}_{\mrm p}$ denote the subchain of $\mc S$ generated by $\mrm p$ (see 7-2.2).
Let $\prec$ be the relation in $\mc S$ defined by
\[\mrm p\prec\mrm q \qquad \text{if and only if} \qquad \mrm q\in\mc S_{\mrm p}.\]
Using only the properties of chains, prove that $\prec$ is a weak linear order relation in $\mc S$.
Prove that $\prec$ coincides with $\leq$.
\end{exercise}

\begin{solution}
We have $\mrm x\neq f(\mrm x)$ for all $\mrm x$. Since $\mrm e$ is the first element of $\mc S$,
$\mrm e\neq f(\mrm e)$. Suppose $\mrm n\neq f(\mrm n)$. Then since $f$ is injective, we have
$f(\mrm n)\neq f(f(\mrm n))$.

Next we prove that if $\mc S_{\mrm p}=\mc S_{\mrm q}$, then $\mrm p = \mrm q$.
We prove this by induction on $\mrm p$.
Let $\mrm q$ be arbitrary and suppose $\mc S_{\mrm e}=\mc S_{\mrm q}$. Then $\mrm e\in \mc S_{\mrm q}$.
From 7-2.1, either $\mrm e=\mrm q$ or $\mrm e\in f(\mc S_{\mrm q})$.
The latter case is not possible because $\mrm e$ is not in the range of $f$. Hence $\mrm e=\mrm q$.
Now assume that $\mc S_{\mrm n}=\mc S_{\mrm q}$ implies $\mrm n = \mrm q$ for any $\mrm q$.
Let $\mrm q$ be arbitrary and suppose $\mc S_{f(\mrm n)}=\mc S_{\mrm q}$.
Then $f(\mrm n)\in \mc S_{\mrm q}$.  Either $f(\mrm n)=\mrm q$ or $f(\mrm n)=f(\mrm x)$
for some $\mrm x\in S_{\mrm q}$. In the latter case, we have $\mrm n=\mrm x\in\mc S_{\mrm q}$.
Therefore $\mc S_{\mrm n}\sse \mc S_{\mrm q}=\mc S_{f(\mrm n)}$.
But we also have $\mc S_{f(\mrm n)}\sse\mc S_{\mrm n}$.
Hence $\mc S_{\mrm n}=\mc S_{f(\mrm n)}$, so $\mrm n=f(\mrm n)$. But $\mrm n\neq f(\mrm n)$.
Therefore $f(\mrm n)=\mrm q$. This completes the induction.

Now note that $\mrm q\in \mc S_{\mrm p}$ if and only if $\mc S_{\mrm q}\sse \mc S_{\mrm p}$, since
$\mc S_{\mrm q}$ is the smallest subchain containing $\mrm q$.
Transitivity and reflexivity of $\prec$ is now immediate. Antisymmetry follows from the result in the
previous paragraph. Therefore $\prec$ is a weak order relation.

We must prove that $\prec$ is linear.

First we show that $\mc S_{f(\mrm x)} = \mc S_{\mrm x} - \set{\mrm x}$ for all $\mrm x$.
Note that $\mc S_{f(\mrm x)}\cup\set{\mrm x}$ is a subchain containing $\mrm x$, so
$\mc S_{\mrm x}\sse \mc S_{f(\mrm x)}\cup\set{\mrm x}$.
This implies $\mc S_{\mrm x}-\set{\mrm x}\sse \mc S_{f(\mrm x)}$.
We must show that $\mc S_{f(\mrm x)}\sse \mc S_{\mrm x}-\set{\mrm x}$. This amounts to showing
that $\mrm x\nin \mc S_{f(\mrm x)}$ since $\mc S_{f(\mrm x)}\sse \mc S_{\mrm x}$.
We use induction.
If $\mrm e\in \mc S_{f(\mrm e)}$, then by 7-2.1, we have $\mrm e = f(\mrm e)$ or
$\mrm e\in f(\mc S_{f(\mrm e)})$. Since $\mrm e\nin\ran f$, $\mrm e\nin \mc S_{f(\mrm e)}$.
Assume $\mrm n\nin \mc S_{f(\mrm n)}$.
If $f(\mrm n)\in \mc S_{f(f(\mrm n))}$, then by 7-2.1, we have $f(\mrm n)=f(f(\mrm n))$
or $f(\mrm n)\in f(\mc S_{f(f(\mrm n))})$. In the former case, we get $\mrm n=f(\mrm n)$ which is false.
In the latter case, we get $f(\mrm n)=f(\mrm y)$ for some $\mrm y\in \mc S_{f(f(\mrm n))}$.
But then $\mrm n=\mrm y\in \mc S_{f(f(\mrm n))}\sse \mc S_{f(\mrm n)}$, which is false.
Therefore $f(\mrm n)\nin \mc S_{f(f(\mrm n))}$. This completes the induction.

Let $P(\mrm n)$ be the statement ``$\mrm n$ is comparable with every member of $\mc S$''.
Since $\mrm p\in \mc S=\mc S_{\mrm e}$ for all $\mrm p$, $\mrm e$ is comparable with every member
of $\mc S$. Assume $P(\mrm k)$. Let $\mrm q$ be any member of $\mc S$.
If $\mrm q\in\mc S_{\mrm k}$ and $\mrm q\neq \mrm k$, then $\mrm q\in \mc S_{f(\mrm k)}$.
If $\mrm q=\mrm k$, then $f(\mrm k)=f(\mrm q)\in \mc S_{\mrm q}$. If $\mrm q\nin\mc S_{\mrm k}$,
then $\mrm k\in \mc S_{\mrm q}$ (by $P(\mrm k)$), so $f(\mrm k)\in \mc S_{\mrm q}$.
Thus $\mrm q$ is comparable with $f(\mrm k)$.
Since $\mrm q$ was arbitrary, $P(f(\mrm k))$ holds.
Therefore $(\forall \mrm n)P(\mrm n)$.

That $\prec$ coincides with $\leq$ follows from \hyperref[ex:9-1.3]{Exercise 3},
since $(\forall s) s \prec f(s)$.
\end{solution}

\begin{exercise}
Show that there is a unique binary operation $\ast$ in $\mc S$ such that
\begin{enumalpha}
    \item $\mrm e \ast \mrm e = \mrm e$,
    \item $(\forall \mrm p)\thinspace f(\mrm p)\ast \mrm e = f(f(\mrm p \ast \mrm e))$, and
    \item $(\forall \mrm p, \mrm q)\thinspace \mrm p\ast f(\mrm q)=f(f(\mrm p\ast\mrm q))$.
\end{enumalpha}
Show that $\ast$ is commutative.
\end{exercise}

\begin{solution}
Let $g=f\circ f$. Then $\aprn{\mc S,g}$ is a chain.
From 7-3.6, there is a function $\psi$ from $\mc S$ to $\mc S$ such that $\psi(\mrm e)=\mrm e$
and $(\forall \mrm p)\thinspace\psi(f(\mrm p))=g(\psi(\mrm p))=f(f(\psi(\mrm p)))$.

From 7-3.7, there is a unique function from
$\mc S\times \mc S$ to $\mc S$ such that for all $\mrm p$ and $\mrm q$, $\mrm p\ast\mrm e = \psi(\mrm p)$
and $\mrm p\ast f(\mrm q) = f(f(\mrm p\ast\mrm q))$. This function immediately satisfies
(a) and (c).
For (b), we compute
\[f(\mrm p)\ast \mrm e = \psi(f(\mrm p))=f(f(\psi(\mrm p))).\]
Now we show that $\ast$ is commutative.

First we use induction on $\mrm p$ to show that $\mrm p\ast \mrm e=\mrm e\ast\mrm p$.
Obviously $\mrm e\ast\mrm e=\mrm e\ast\mrm e$. Suppose $\mrm n\ast\mrm e=\mrm e\ast\mrm n$.
Then
$f(\mrm n)\ast\mrm e=f(f(\mrm n\ast \mrm e))=f(f(\mrm e\ast\mrm n))=\mrm e\ast f(\mrm n)$,
where the last equality follows from (c). This completes the induction.

Next we use induction on $\mrm q$ to show that
$(\forall \mrm p, \mrm q)\thinspace f(\mrm p)\ast \mrm q=f(f(\mrm p\ast\mrm q))$.
Let $\mrm p$ be arbitrary. Then from (b) we get $f(\mrm p)\ast \mrm e=f(f(\mrm p\ast e))$.
Suppose $f(\mrm p)\ast \mrm n=f(f(\mrm p\ast \mrm n))$.
Then \[f(\mrm p)\ast f(\mrm n)=f(f(f(\mrm p)\ast\mrm n))=f(f(f(f(\mrm p\ast \mrm n))))
=f(f(\mrm p\ast f(\mrm n))).\]
This completes the induction.

Finally we prove that $(\forall \mrm p, \mrm q)\thinspace \mrm p\ast\mrm q=\mrm q\ast\mrm p$ with
induction on $\mrm q$.
Let $\mrm p$ be arbitrary.
Then $\mrm p\ast\mrm e=\mrm e\ast\mrm p$.
Suppose $\mrm p\ast\mrm n=\mrm n\ast\mrm p$.
Then
\[\mrm p\ast f(\mrm n)=f(f(\mrm p\ast\mrm n))=f(f(\mrm n\ast\mrm p))=f(\mrm n)\ast \mrm p.\]
This completes the induction.
\end{solution}


\section{Fractions and rational numbers}
\begin{exercise}
Suppose that $\alpha\cdot\alpha=\alpha$. Prove that $\alpha = \eps$.
\end{exercise}

\begin{solution}
We have $\alpha=\alpha\eps=\alpha\cdot (\alpha\cdot\alpha^*)=(\alpha\cdot\alpha)\cdot\alpha^*
=\alpha\cdot\alpha^*=\eps$.
\end{solution}

\begin{exercise}
Prove that there exists a unique map $\Omega$ from $\mc S$ to $\mc Q$ such that
\begin{align*}
    \Omega(\mrm x + \mrm y) &= \Omega(\mrm x) \oplus \Omega(\mrm y),\\
    \Omega(\mrm{xy}) &= \Omega(\mrm x)\cdot\Omega(\mrm y).
\end{align*}
Prove that this map is injective and order-preserving.
\end{exercise}

\begin{solution}
Define $\Omega$ from $\mc S$ to $\mc Q$ by $\Omega(\mrm x)=\overbar{\aprn{\mrm x,\mrm e}}$. 
Now we compute
\[\Omega(\mrm x) + \Omega(\mrm y)=\overbar{\aprn{\mrm x,\mrm e}} + \overbar{\aprn{\mrm y,\mrm e}}
=\overbar{\aprn{\mrm{xe}+\mrm{ey},\mrm{ee}}}
=\overbar{\aprn{\mrm{x}+\mrm{y},\mrm{e}}}-\Omega(\mrm x + \mrm y)\]
and
\[\Omega(\mrm x) \cdot \Omega(\mrm y)=\overbar{\aprn{\mrm x,\mrm e}} \cdot \overbar{\aprn{\mrm y,\mrm e}}
=\overbar{\aprn{\mrm{xy},\mrm{ee}}}
=\overbar{\aprn{\mrm{xy},\mrm{e}}}=\Omega(\mrm{xy}).\]
Note that $\Omega(\mrm e)=\overbar{\aprn{\mrm e,\mrm e}}=\eps$.

Let $\Omega'$ is another such map. We use induction to prove that $\Omega'=\Omega$.
We have $\Omega'(\mrm e)\cdot \Omega'(\mrm e) = \Omega'(\mrm e\mrm e)
=\Omega'(\mrm e)$. From \hyperref[ex:9-2.1]{Exercise 1}, we have $\Omega'(\mrm e)=\eps=\Omega(\mrm e)$.
Suppose $\Omega'(\mrm n)=\Omega(\mrm n)$
Then $\Omega'(f(\mrm n)) = \Omega'(\mrm n + \mrm e)=\Omega'(\mrm n) \oplus \Omega'(\mrm e)
=\Omega(\mrm n) \oplus \Omega(\mrm e)=\Omega(\mrm n + \mrm e)=\Omega(f(\mrm n))$.
This completes the induction.

Suppose $\mrm x < \mrm y$. Then $\mrm x + \mrm r = \mrm y$ for some $\mrm r$.
Then $\Omega(\mrm x)\oplus \Omega(\mrm r)=\Omega(\mrm x + \mrm r) = \Omega(\mrm y)$.
Hence $\Omega(\mrm x) \prec \Omega(\mrm y)$.
Therefore $\Omega$ is order-preserving and injective.
\end{solution}

\begin{exercise}
Let $F$ be any ordered field.
Show that there is an isomorphism of $\aprn{\mc Q,\oplus,\cdot}$ onto
$\aprn{P(F)\cap Q(F),+,\cdot}$, where the operations in the latter configuration are the
restrictions of those in $F$. Show that this isomorphism is unique.
Show also that this isomorphism is order-preserving.
\end{exercise}

\begin{solution}
From Proposition 8-5.8, an ordered field has characteristic 0. Therefore $N(F)$ is a simple chain.
Similar to \hyperref[ex:8-4.4]{Exercise 8-4.4}, there is a unique isomorphism $\phi$ from $\mc S$
to $N(F)$ such that
$\phi(\mrm e)=1$ and $\phi(f(\mrm s))=\phi(\mrm s) + 1$. Then by induction we have
$\phi(\mrm x + \mrm y) = \phi(\mrm x) + \phi(\mrm y)$ and
$\phi(\mrm x \mrm y) = \phi(\mrm x)\cdot\phi(\mrm y)$ for all $\mrm x$ and $\mrm y$.

Define $\psi$ from $\mc Q$ to $F$ by $\psi(\alpha)=\phi(\mrm p)/\phi(\mrm q)$ for some
representative $\aprn{\mrm p,\mrm q}$ of $\alpha$. This is well defined since if
$\aprn{\mrm p,\mrm q}\sim \aprn{\mrm r,\mrm s}$, then
$\phi(\mrm p)\cdot \phi(\mrm s)=\phi(\mrm p\mrm s)=\phi(\mrm q\mrm r)=\phi(\mrm q)\cdot \phi(\mrm r)$,
whence $\phi(\mrm p)/\phi(\mrm r)=\phi(\mrm q)/\phi(\mrm s)$.
Let $\aprn{\mrm p,\mrm q}$ be a representative of $\alpha\in\mc Q$ and let
$\aprn{\mrm r,\mrm s}$ be a representative of $\beta\in\mc Q$.
Then $\alpha\oplus\beta=\overbar{\aprn{\mrm p\mrm s + \mrm q\mrm r, \mrm q\mrm s}}$.
Then we compute
\[
\psi(\alpha)+\psi(\beta)=\phi(\mrm p)/\phi(\mrm q) + \phi(\mrm r)/\phi(\mrm s)
=\frac{\phi(\mrm p)\cdot \phi(\mrm s)+\phi(\mrm q)\cdot \phi(\mrm r)}{\phi(\mrm q)\cdot \phi(\mrm s)}
=\frac{\phi(\mrm p\mrm s + \mrm q\mrm r)}{\phi(\mrm q\mrm s)}
=\psi(\alpha\oplus\beta).
\]
Since $\alpha\cdot\beta=\overbar{\aprn{\mrm p\mrm r, \mrm q\mrm s}}$, we also have
\[\psi(\alpha)\cdot\psi(\beta)=\frac{\phi(\mrm p)}{\phi(\mrm q)}\cdot \frac{\phi(\mrm r)}{\phi(\mrm s)}
=\frac{\phi(\mrm p\mrm r)}{\phi(\mrm q\mrm s)}
=\psi(\alpha\cdot\beta).
\]
Suppose $\alpha\prec\beta$. Then $\alpha+\gamma=\beta$ for some $\gamma$. 
Then we have
\[\psi(\alpha)+\psi(\gamma)=\psi(\alpha+\gamma)=\psi(\beta).\]
Since $\psi(\gamma)$ is positive in $F$, we have $\psi(\alpha)\prec\psi(\beta)$.
Therefore $\psi$ is order-preserving and injective.
The range of $\psi$ is $\setb{x/y}{x,y\in N(F)}$, and this was shown in the answer to
\hyperref[ex:8-5.1]{Exercise 8-5.1} to be $P(F)\cap Q(F)$.

To prove uniqueness, let $\psi'$ be another isomorphism of $\aprn{\mc Q,\oplus,\cdot}$ onto
$\aprn{P(F)\cap Q(F),+,\cdot}$. Consider the map $\Omega$ from $\mc S$ to $\mc Q$
from \hyperref[ex:9-2.2]{Exercise 9-2.2}.

We prove that $\psi'(\Omega(\mrm x))=\psi(\Omega(\mrm x))$ for all $\mrm x\in\mc S$.
We have $\psi(\Omega(\mrm e))=\phi(\mrm e)/\phi(\mrm e)=1/1=1$.
Now we compute $\psi'(\Omega(\mrm e))\psi'(\Omega(\mrm e))=\psi'(\Omega(\mrm e)\Omega(\mrm e))
=\psi'(\Omega(\mrm e\mrm e))=\psi'(\Omega(\mrm e))\cdot 1$. By cancellation we have
$\psi'(\Omega(\mrm e))=1$.
Now suppose $\psi'(\Omega(\mrm n))=\psi(\Omega(\mrm n))$.
We compute $
\psi'(\Omega(f(\mrm n)))=\psi'(\Omega(\mrm n+\mrm e))=\psi'(\Omega(\mrm n))+\psi'(\Omega(\mrm e))
=\psi(\Omega(\mrm n))+\psi(\Omega(\mrm e))
=\psi(\Omega(\mrm n+\mrm e))=\psi(\Omega(f(\mrm n)))$.
This completes the induction.

Now $\psi(\Omega(\mrm q))\cdot\psi(\Omega(\mrm q)^{\ast})
=\psi(\Omega(\mrm q)\cdot \Omega(\mrm q)^{\ast})=\psi(\eps)=1$, so
$1/\psi(\Omega(\mrm q)) = \psi(\Omega(\mrm q)^{\ast})$. The same applies to $\psi'$.
Since $\psi$ and $\psi'$ agree for $\Omega(\mrm q)$, they also agree on $\Omega(\mrm q)^{\ast}$
for any $\mrm q$.

Now note that $\Omega(\mrm q)^{\ast}=\overbar{\aprn{\mrm e,\mrm q}}$ for all $\mrm q$,
since $\Omega(\mrm q)\cdot \overbar{\aprn{\mrm e,\mrm q}}
=\overbar{\aprn{\mrm q,\mrm e}}\cdot \overbar{\aprn{\mrm e,\mrm q}}
=\overbar{\aprn{\mrm q,\mrm q}}=\eps$.
Since $\overbar{\aprn{\mrm p,\mrm q}}=\overbar{\aprn{\mrm p,\mrm e}}\cdot \overbar{\aprn{\mrm e,\mrm q}}
=\Omega(\mrm p)\cdot \Omega(\mrm q)^{\ast}$, we see that for any $\alpha\in\mc Q$, we have
$\alpha=\Omega(\mrm p)\cdot \Omega(\mrm q)^{\ast}$ for some $\mrm p,\mrm q\in\mc S$.
Then
\[\psi(\alpha)=\psi(\Omega(\mrm p)\cdot \Omega(\mrm q)^{\ast})
=\psi(\Omega(\mrm p))\cdot \psi(\Omega(\mrm q)^{\ast})
=\psi'(\Omega(\mrm p))\cdot \psi'(\Omega(\mrm q)^{\ast})
=\psi'(\Omega(\mrm p)\cdot \Omega(\mrm q)^{\ast})=\psi'(\alpha).\]
Therefore $\psi'=\psi$.
\end{solution}

\begin{exercise}
Show by argument with representatives that there is a unique relation $L$ in $\mc Q$ such
that $\overbar{\aprn{\mrm p,\mrm q}}\mathrel L \overbar{\aprn{\mrm r,\mrm s}}$
if and only if $\mrm{ps} < \mrm{qr}$. Prove that $L$ is a strong linear order relation.
Then prove that $L$ coincides with the order relation of 9-2.6.
\end{exercise}

\begin{solution}
Define $L=\setb{\aprn{\alpha,\beta}}{(\exists \mrm w,\mrm x,\mrm y,\mrm z)\thinspace
\aprn{\mrm w,\mrm x}\in\alpha, \aprn{\mrm y,\mrm z}\in\beta, \mrm{wz}<\mrm{xy}}$.
Suppose $\overbar{\aprn{\mrm p,\mrm q}}\mathrel L \overbar{\aprn{\mrm r,\mrm s}}$.
Choose $\mrm w,\mrm x,\mrm y,\mrm z$ so that $\aprn{\mrm w,\mrm x}\in\overbar{\aprn{\mrm p,\mrm q}}$,
$\aprn{\mrm y,\mrm z}\in\overbar{\aprn{\mrm r,\mrm s}}$, and $\mrm{wz}<\mrm{xy}$.
Then $\mrm w\mrm q=\mrm x\mrm p$ and $\mrm y\mrm s=\mrm z\mrm r$.
Then we have $\mrm{xyps}=\mrm{xpys}=\mrm{wqzr}=\mrm{wzqr}<\mrm{xyqr}$. Therefore $\mrm{ps} < \mrm{qr}$.
We have prove $\overbar{\aprn{\mrm p,\mrm q}}\mathrel L \overbar{\aprn{\mrm r,\mrm s}}$ implies $\mrm{ps} < \mrm{qr}$.
The converse is trivial.

We prove that $L$ is a strong order relation.
If $\overbar{\aprn{\mrm p,\mrm q}}\mathrel L \overbar{\aprn{\mrm r,\mrm s}}$ and
$\overbar{\aprn{\mrm r,\mrm s}} \mathrel L \overbar{\aprn{\mrm t,\mrm u}}$,
then $\mrm{ps} < \mrm{qr}$ and $\mrm{ru} < \mrm{st}$.
Then $\mrm{psu} < \mrm{qru}$ and $\mrm{qru} < \mrm{qst}$, so $\mrm{pus} < \mrm{qts}$, so $\mrm{pu} < \mrm{qt}$.
Hence $\overbar{\aprn{\mrm p,\mrm q}}\mathrel L \overbar{\aprn{\mrm t,\mrm u}}$
This shows that $L$ is transitive.
$L$ is also strictly irreflexive since $\mrm{pq}\nless \mrm{qp}$, so it is never the case that
$\overbar{\aprn{\mrm p,\mrm q}}\mathrel L \overbar{\aprn{\mrm p,\mrm q}}$.

We prove that $L$ is linear. Given $\overbar{\aprn{\mrm p,\mrm q}}\in \mc Q$ and
$\overbar{\aprn{\mrm r,\mrm s}}\in\mc Q$, we have by linearity of $<$ that $\mrm{ps}=\mrm{qr}$, $\mrm{ps} < \mrm{qr}$,
or $\mrm{qr} < \mrm{ps}$.
These correspond exactly to $\overbar{\aprn{\mrm p,\mrm q}}=\overbar{\aprn{\mrm r,\mrm s}}$,
$\overbar{\aprn{\mrm p,\mrm q}}\mathrel L\overbar{\aprn{\mrm r,\mrm s}}$, and
$\overbar{\aprn{\mrm r,\mrm s}}\mathrel L\overbar{\aprn{\mrm p,\mrm q}}$.

If $\overbar{\aprn{\mrm p,\mrm q}}\mathrel L\overbar{\aprn{\mrm r,\mrm s}}$, then choose $\mrm t$
so that $\mrm{ps} + \mrm t = \mrm{qr}$. Then $\overbar{\aprn{\mrm p,\mrm q}} + \overbar{\aprn{\mrm t,\mrm{qs}}}
=\overbar{\aprn{\mrm r,\mrm s}}$. This shows that $\alpha\mathrel L\beta \implies \alpha\prec\beta$.
The reverse implication follows because $L$ is linear.
Thus $L$ coincides with the order relation of 9-2.6.
\end{solution}

\begin{exercise}
Define binary operations $+$ and $\times$ in $\mc S\times\mc S$ as follows:
\begin{align*}
    \aprn{\mrm p, \mrm q} + \aprn{\mrm r,\mrm s} &= \aprn{\mrm{ps+qr}, \mrm{qs}},\\
    \aprn{\mrm p, \mrm q}\times\aprn{\mrm r,\mrm s} &= \aprn{\mrm{pr},\mrm{qs}}.
\end{align*}
Show that these operations are commutative and associative, but that the distributive
law does not hold. Show that the operations $\oplus$ and $\cdot$ in $\mc Q$ can be obtained from the
operations $+$ and $\times$ by the general method for the construction of functions on quotient spaces
(Theorem 5-2.1 and Exercise 2(b), p. 69).
\end{exercise}

\begin{solution}
We have
\[\aprn{\mrm p, \mrm q} + \aprn{\mrm r,\mrm s}=\aprn{\mrm{ps+qr}, \mrm{qs}}
=\aprn{\mrm{rq+sp}, \mrm{sq}}=\aprn{\mrm r, \mrm s} + \aprn{\mrm p,\mrm q}\]
so $+$ is commutative.

We have
\[\aprn{\mrm p, \mrm q} \times \aprn{\mrm r,\mrm s}=\aprn{\mrm{pr}, \mrm{qs}}
=\aprn{\mrm{rp}, \mrm{sq}}=\aprn{\mrm r, \mrm s} \times \aprn{\mrm p,\mrm q}\]
so $\times$ is commutative.

We have
\begin{align*}
(\aprn{\mrm p, \mrm q} + \aprn{\mrm r,\mrm s})+ \aprn{\mrm t,\mrm u}
&=\aprn{\mrm{ps}+\mrm{qr}, \mrm{qs}} + \aprn{\mrm t,\mrm u}\\
&=\aprn{\mrm{psu}+\mrm{qru}+\mrm{qst}, \mrm{qsu}}\\
&=\aprn{\mrm{p}, \mrm q} + \aprn{\mrm{ru}+\mrm{st},\mrm{su}}\\
&=\aprn{\mrm p, \mrm q} + (\aprn{\mrm r,\mrm s}+ \aprn{\mrm t,\mrm u})
\end{align*}
so $+$ is associative.

We have
\begin{align*}
(\aprn{\mrm p, \mrm q} \times \aprn{\mrm r,\mrm s})\times \aprn{\mrm t,\mrm u}
&=\aprn{\mrm{pr}, \mrm{qs}}\times \aprn{\mrm t,\mrm u}\\
&=\aprn{(\mrm{pr})\mrm t, (\mrm{qs})\mrm u}\\
&=\aprn{\mrm{p}(\mrm{rt}), \mrm q(\mrm{su})}\\
&=\aprn{\mrm{p}, \mrm q} \times \aprn{\mrm{rt},\mrm{su}}\\
&=\aprn{\mrm p, \mrm q} \times (\aprn{\mrm r,\mrm s}\times \aprn{\mrm t,\mrm u})
\end{align*}
so $\times$ is associative.

The distributive law does not hold because
\[\aprn{\mrm p, \mrm q}\times (\aprn{\mrm r,\mrm s} + \aprn{\mrm t,\mrm u})
= \aprn{\mrm p, \mrm q}\times \aprn{\mrm{ru}+\mrm{st},\mrm{su}}
= \aprn{\mrm{pru}+\mrm{pst},\mrm{qsu}}\]
while
\[(\aprn{\mrm p, \mrm q} \times \aprn{\mrm r,\mrm s}) + (\aprn{\mrm p, \mrm q} \times \aprn{\mrm t,\mrm u})
= \aprn{\mrm{pr}, \mrm{qs}} + \aprn{\mrm{pt}, \mrm{qu}}
= \aprn{\mrm{prqu} + \mrm{qspt}, \mrm{qsqu}}
\]
In particular, the distributive law would imply $\mrm{qsu} = \mrm{qqsu}$ for all $\mrm q$, $\mrm s$, and $\mrm u$.
By cancellation, that implies $\mrm e=\mrm q$ for all $\mrm q$, which is false since $\mrm e\neq f(\mrm e)$.

Let $A=\mc S\times\mc S$ and $B=\mc S\times\mc S$. Let $T = \mc Q$.
Let $g$ from $A\times B$ to $T$ be defined by $g(\aprn{\mrm p,\mrm q}, \aprn{\mrm r,\mrm s})
=\overbar{\aprn{\mrm p,\mrm q} + \aprn{\mrm r,\mrm s}}=\overbar{\aprn{\mrm{ps}+\mrm{qr},\mrm{qs}}}$.
As shown in the text,
\[
\aprn{\mrm p,\mrm q}\sim\aprn{\mrm t,\mrm u} \implies
g(\aprn{\mrm p,\mrm q}, \aprn{\mrm r,\mrm s})=g(\aprn{\mrm t,\mrm u}, \aprn{\mrm r,\mrm s})
\]
and
\[
\aprn{\mrm r,\mrm s}\sim\aprn{\mrm v,\mrm w}\implies
g(\aprn{\mrm p,\mrm q}, \aprn{\mrm r,\mrm s})=g(\aprn{\mrm p,\mrm q}, \aprn{\mrm v,\mrm w}).
\]
Letting $D$ and $E$ both be the equivalence relation $\sim$ in $A\times B$,
\hyperref[ex:5-2.2]{Exercise 5-2.2} tells us there is a unique function
$h$ from $\mc Q\times\mc Q$ to $\mc Q$ such that
\[h(\overbar{\aprn{\mrm p,\mrm q}},\overbar{\aprn{\mrm r,\mrm s}})=g(\aprn{\mrm p,\mrm q},\aprn{\mrm r,\mrm s}).\]
Since $g(\aprn{\mrm p,\mrm q},\aprn{\mrm r,\mrm s})=\overbar{\aprn{\mrm{ps}+\mrm{qr},\mrm{qs}}}$,
we see that $h$ is precisely $\oplus$ in $\mc Q$.
The situation for $\cdot$ in $\mc Q$ is analogous.
\end{solution}



\section{The positive real numbers}




\section{Real numbers}
