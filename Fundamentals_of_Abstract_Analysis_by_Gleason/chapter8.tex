\chapter{Fields}
\section{Binary operations}
\begin{exercise}
Restate the definition of isomorphism for configurations consisting of a set and a
binary operation in that set, using terms adapted to this class of configurations. Let the
binary operation in both configurations be denoted by some medial symbol.
\end{exercise}

\begin{solution}
Let $\aprn{A,\oplus}$ and $\aprn{B,\otimes}$ be configurations consisting of basic sets
$A$ and $B$ and binary operations $\oplus$ and $\otimes$ on $A$ and $B$, respectively.

Then an isomorphism from $A$ to $B$ is a bijection $\phi$ from $A$ to $B$ such that
$\phi(a_1\oplus a_2)=\phi(a_1)\otimes\phi(a_2)$ for all $a_1,a_2\in A$.
\end{solution}

\begin{exercise}
Suppose that $b$ is a binary operation in the set $S$. Prove that if $b$ has both a
left-identity and a right-identity then it has just one of each kind and these are the same.
Give examples to prove that it is possible that $b$ has
\begin{enumalpha}
    \item more than one left-identity and more than one left-zero,
    \item more than one left-identity and more than one right-zero,
    \item more than one left-identity and a zero,
    \item an identity and more than one left-zero,
    \item a left-identity which is also a right-zero ($S$ having more than one element).
\end{enumalpha}
\end{exercise}

\begin{solution}
If $l$ is a left-identity and $r$ is a right identity, then we have $b(l,r)=r$ since $l$ is a
left-identity and $b(l,r)=l$ since $r$ is a right identity. So $b$ has an identity $u=l=r$,
and any left or right-identity is equal to $u$.

For the following examples, we give tables which give the elements of $S$ and define
the binary operation $b$ on $S$. Note that the values in a row for a left-identity must match value
for the corresponding column, and the values in a row for a left-zero are just the left-zero.
The same applies to right-identites and right-zeros, except the roles of row and column are swapped.

\begin{enumalpha}
    \item
    \[\begin{array}{c|cccc}
          & a & b & c & d\\
        \hline
        a & a & b & c & d\\
        b & a & b & c & d\\
        c & c & c & c & c\\
        d & d & d & d & d
    \end{array}\]
    Here $a$ and $b$ are left-identities while $c$ and $d$ are left-zeros.

    \item
    \[\begin{array}{c|cccc}
          & a & b & c & d\\
        \hline
        a & a & b & c & d\\
        b & a & b & c & d\\
        c & c & c & c & d\\
        d & d & d & c & d
    \end{array}\]
    Here $a$ and $b$ are left-identities while $c$ and $d$ are right-zeros.

    \item
    \[\begin{array}{c|ccc}
          & a & b & c\\
        \hline
        a & a & b & c\\
        b & a & b & c\\
        c & c & c & c
    \end{array}\]
    Here $a$ and $b$ are left-identities while $c$ is a zero.

    \item
    \[\begin{array}{c|ccc}
          & a & b & c\\
        \hline
        a & a & b & c\\
        b & b & b & b\\
        c & c & c & c
    \end{array}\]
    Here $a$ is an identity while $b$ and $c$ are left-zeros.

    \item
    \[\begin{array}{c|cc}
          & a & b\\
        \hline
        a & a & b\\
        b & a & a
    \end{array}\]
    Here $a$ is a left-identity and a right-zero.
\end{enumalpha}
\end{solution}

\begin{exercise}
Which of the following binary operations defined on the ordinary numbers are
associative? commutative? have a left-identity? a right-identity? a left-zero? a right-
zero?
\begin{tasks}[label=(\alph*),label-width=1.4em](2)
    \task $(x,y)\mapsto x$
    \task $(x,y)\mapsto \abs{x-y}$
    \task $(x,y)\mapsto x+y+xy$
    \task $(x,y)\mapsto xy-2x-y+4$
    \task $(x,y)\mapsto xy+y-1$
    \task $(x,y)\mapsto \sqrt{x^2+y^2}$\quad (let $S$ be the nonnegative numbers)
    \task $(x,y)\mapsto \max(x,y)$
    \task $(x,y)\mapsto \max(x^2-y,y^2-x)$
\end{tasks}
\end{exercise}

\begin{solution}
\begin{enumalpha}
    \item Associative because $b(x,b(y,z))=x=b(x,y)=b(b(x,y),z)$.
    Not commutative because $b(2,1)=2\neq 1=b(1,2)$.
    No left-identity because $b(x,x+1)=x\neq x+1$.
    Any number $y$ is a right-identity because $b(x,y)=x$.
    Any number $x$ is a left-zero because $b(x,y)=x$.
    No right-zero because $b(y+1,y)=y+1\neq y$.

    \item Not associative because $b(3,b(2,1)) = \abs{3-\abs{2-1}}=2\neq0=\abs{\abs{3-2}-1}=b(b(3,2),1)$.
    Commutative because $b(x,y)=\abs{x-y}=\abs{y-x}=b(y,x)$.
    No left-identity because $b(x,x)=\abs{x-x}=0\neq x$ if $x\neq 0$ and $b(0,-1)=\abs{0-1}=1\neq -1$.
    No right-identity for a similar reason.
    No left-zero because $b(x,x+1)=\abs{x-(x+1)}=1$ and $b(x,x+2)=\abs{x-(x+2)}=2$.
    No right-zero for a similar reason.

    \item Associative because $b(x,b(y,z))=x+b(y,z)+x\cdot b(y,z)=x+(y+z+yz)+x\cdot(y+z+yz)
    =x+y+z+yz+xy+xz+xyz$ and $b(b(x,y),z)=b(x,y) + z + b(x,y)\cdot z=(x+y+xy) + z + (x+y+xy)\cdot z
    =x + y + z + xy + xz + yz + xyz$.
    Commutative because $b(x,y)=x+y+xy=y+x+yx=b(y,z)$.
    Identity is 0 because $b(0,y)=0+y+0y=y$.
    Zero is $-1$ because $b(-1,y)=-1+y-y=-1$.

    \item Not associative because $b(b(1,0),0)=b(2,0)=0$ while $b(1,b(0,0))
    =b(1,4)=2$.
    Not commutative because $b(1,0)=2$ while $b(0,1)=3$.
    Left-identity is 2 because $b(2,y)=2y-2\cdot 2 - y + 4 = y$.
    No right-identity because $b(0,y)=0$ implies $-y+4=0$ which implies $y=4$, but $b(1,4)=4-2-4+4=2$.
    No left-zero because $b(x,0)=x$ implies $x=4/3$ while $b(x,1)=x$ implies $x=3/2$.
    Right-zero is 2 because $b(x,2)=2x-2x-2+4=2$.

    \item Not associative because $b(1,b(0,0))=b(1,-1)=-3$ while $b(b(1,0),0)=b(-1,0)=-1$.
    Not commutative because $b(1,0)=-1$ while $b(0,1)=0$.
    No left-identity because $b(x,0)=-1\neq 0$.
    Right identity is 1 because $b(x,1)=x+1-1=x$.
    Left-zero is $-1$ because $b(-1,y)=-y+y-1=-1$.
    No right-zero because $b(0,y)=y$ implies $y-1=y$ which is impossible.

    \item Associative because
    $b(x,b(y,z))=\sqrt{x^2+y^2+z^2}=b(b(x,y),z)$.
    Clearly commutative.
    Identity is 0 because $b(0,y)=\sqrt{y^2}=y$.
    No zero element because $b(x,1)=x$ implies $\sqrt{x^2+1}=x$, which is impossible.

    \item Clearly associative and commutative. No identity element since $\max(x,y)=y$ for all $y$
    implies $x\leq y$ for all $y$, which is impossible.
    No zero element since $\max(x,y)=x$ for all $y$ implies $y\leq x$ for all $y$, which is impossible. 

    \item Not associative because $b(0,b(1,2))=b(0,3)=9$ while $b(b(0,1),2)=b(1,2)=3$.
    Commutative because $b(x,y)=\max(x^2-y,y^2-x)=\max(y^2-x,x^2-y)=b(y,x)$.
    No identity because $b(x,0)=0$ implies $\max(x^2,-x)=0$ which implies $x=0$, but $b(0,2)=4\neq 2$.
    No zero element since $b(x,0)=x$ implies $x^2=x$ or $-x=x$, which implies $x=1$ or $x=0$.
    But $b(1,1)=0\neq 1$ and $b(0,2)=4\neq 0$.
\end{enumalpha}
\end{solution}

\section{Fields}
\begin{exercise}
Determine which of the following systems endowed with the usual addition and
multiplication operations are fields: the positive integers, the nonnegative integers, the
integers, the positive rationals, the rationals, the nonnegative real numbers, the real
numbers. In each case, find which postulates are valid and which fail.
\end{exercise}

\begin{solution}
For the positive integers, there is no additive identity and only $1$ has a multiplicative
inverse.

For the nonnegative integers, only $0$ has an additive inverse and only $1$ has a multiplicate inverse.

For the integers, only $1$ has a multiplicate inverse.

For the positive rationals, there is no additive identity.

The rationals are a field.

For the nonnegative real numbers, only $0$ has an additive inverse.

The real numbers are a field.
\end{solution}

\begin{exercise}
Verify that $\aprn{F, a, m}$ of the Example on p. 96 is indeed a field.
\end{exercise}

\begin{solution}
The following tables show that the commutative, associative, and distributive laws hold.

Addition is commutative:
\[
\begin{array}{cccc}
    x & y & x+y & y+x\\
    \hline
    q & q & p & p\\
    q & p & q & q\\
    p & q & q & q\\
    p & p & p & p\\
\end{array}
\]
Addition is associative:
\[
\begin{array}{ccccc}
    w & x & y & (w+x)+y & w+(x+y)\\
    \hline
    q & q & q & q & q\\
    q & q & p & p & p\\
    q & p & q & p & p\\
    q & p & p & q & q\\
    p & q & q & p & p\\
    p & q & p & q & q\\
    p & p & q & q & q\\
    p & p & p & p & p\\
\end{array}
\]
Multiplication is commutative:
\[
\begin{array}{cccc}
    x & y & xy & yx\\
    \hline
    q & q & q & q\\
    q & p & p & p\\
    p & q & p & p\\
    p & p & p & p\\
\end{array}
\]
Multiplication is associative:
\[
\begin{array}{ccccc}
    w & x & y & (wx)y & w(xy)\\
    \hline
    q & q & q & q & q\\
    q & q & p & p & p\\
    q & p & q & p & p\\
    q & p & p & p & p\\
    p & q & q & p & p\\
    p & q & p & p & p\\
    p & p & q & p & p\\
    p & p & p & p & p\\
\end{array}
\]
The distributive law holds:
\[
\begin{array}{ccccc}
    w & x & y & w(x+y) & (wx) + (wy)\\
    \hline
    q & q & q & p & p\\
    q & q & p & q & q\\
    q & p & q & q & q\\
    q & p & p & p & p\\
    p & q & q & p & p\\
    p & q & p & p & p\\
    p & p & q & p & p\\
    p & p & p & p & p\\
\end{array}
\]
From the table for $a$, one checks that $p$ is the additive identity $z$ since $p+p=p$
and $q+p=q$.
Similarly, $q$ is the multiplicative identity $u$ since $pq=p$ and $qq=q$.
We have $p+p=p$ and $q+q=p$, so additive inverses exist.
Finally, we have $qq=q$, so multiplicative inverses exist for all elements besides the additive identity $p$.
\end{solution}

\begin{exercise}
Show that the last part of postulate (iv) for fields can be replaced by
\begin{center}
    \textit{F contains at least two elements;}
\end{center}
that is, prove that this replacement leads to an equivalent system of postulates.
\end{exercise}

\begin{solution}
Clearly postulate (iv) implies $F$ contains at least two elements.

Suppose we replace the last part of posulate (iv) with another postulate (v) that
says that $F$ contains at least two elements.
Let $z$ and $u$ be elements of $F$ satisfying the first three parts of postulate (iv).
Assume for contradiction that $z=u$.
Since $F$ has at least two elements, let $a$ be an element distinct from $z=u$.
From postulate (iv), we have $z+z=z$.
Then we have $a(z+z)=az$. Applying the distributive law, we have $az+az=az$.
Since $z=u$, we have $au+au=au$. Applying postulate (iv), we have $a+a=a$.
From postulate (iv), there exists $c$ such that $a+c=z$.
So we have $a=a+z=a+(a+c)=(a+a)+c=a+c=z$, contradicting the fact that $a$ is distinct from $z$.
Therefore $z\neq u$.
\end{solution}

\begin{exercise}
Rephrase the definition of isomorphism for fields using terms appropriate to this
class of configurations.
\end{exercise}

\begin{solution}
Let $\aprn{F,+,\cdot}$ and $\aprn{G,+,\cdot}$ be fields.
Then an isomorphism from $F$ to $G$ is a bijective function $\phi$ from $F$ to $G$ such that
$\phi(x+y)=\phi(x) + \phi(y)$ and $\phi(x\cdot y)=\phi(x)\cdot\phi(y)$ for all $x,y\in F$.
\end{solution}

\begin{exercise}
Let $G$ be a subfield of the field $F$. Show that the additive identity of $G$ must coincide
with the additive identity of $F$ and the multiplicative identity of $G$ must coincide with
the multiplicative identity of $F$. (If this were not true, we could not afford to denote the
additive identity in any field by 0 and the multiplicative identity by 1. Observe that the
second statement would be false if we allowed a field to have only one element.)
\end{exercise}

\begin{solution}
Let $0_G$ be the additive identiy of $G$. Let $0_F$ be the additive identity of $F$.
Then we have $0_F+0_G=0_G=0_G+0_G$. There exists $x\in F$ such that $0_G+x=0_F$.
Then $0_G=0_G+0_F=0_G+0_G+x=0_F+0_G+x=0_F+0_F=0_F$.

Let $1_G$ be the multiplicative identiy of $G$. Let $1_F$ be the multiplicative identity of $F$.
Then we have $1_F1_G=1_G=1_G1_G$. Since $1_G\neq 0_G=0_F$, there exists $x\in F$ such that $1_Gx=1_F$.
Then $1_G=1_G1_F=1_G1_Gx=1_F1_Gx=1_F1_F=1_F$.
\end{solution}

\section{The elementary arithmetic of fields}
\begin{exercise}
Supply the proofs not given in the text.
\end{exercise}

\begin{solution}
We prove 8-3.1(c). Let $a\neq 0$ and $b$ be given. Choose $c$ so that $ac=1$. Then
$a(cb)=(ac)b=1b=b$, hence we may choose $x=cb$.

We prove 8-3.1(d). Let $a\neq 0$ and $x$ and $y$ be given such that $ax=ay$.
Choose $c$ so that $ca=1$. Then $x=1x=(ca)x=c(ax)=c(ay)=(ca)y=1y=y$.

We prove 8-3.3(a). Since $(b-c)+c=b$ by definition of $b-c$, we have $(a+(b-c))+c=a+((b-c)+c)=a+b$.
The result follows from the definition of $(a+b)-c$.

We prove 8-3.3(b). We have $(b+c)+((a-b)-c)=((b+c)+(a-b))-c=(c+(b+(a-b)))-c=(c+a)-c=c+(a-c)=a$.
The result follows from the definition of $a-(b+c)$.

We prove 8-3.3(c). By defintion, $-(-a)$ is the unique solution of the equation $(-a)+x=0$.
But by definition of $-a$ we have $(-a)+a=0$. Hence $-(-a)=a$.

We prove 8-3.3(e). By defintion, $-(ab)$ is the unique solution of the equation $ab+x=0$.
But we have $ab+(-a)b=(a+(-a))b=0b=0$. Hence $(-a)b=-(ab)$.

We prove 8-3.3(f). Applying (e) and (c), we have $(-a)(-b)=-(a(-b))=-((-b)a)=-(-(ba))=ba=ab$.

We prove 8-3.3(g). Suppose $ab=0$. If $b\neq 0$, choose $c$ such that $bc=1$.
Then $0=0c=(ab)c=a(bc)=a1=a$.

We prove 8-3.3(i).  Let $a/b=x$ and $c/d=y$. Then $a=bx$ and $c=dy$. We must show that $x/y=ad/bc$.
But $ad=bxd$ and $bc=bdy$. So we must show that $x/y=(bdx)/(bdy)$.
Since $(bdy)(x/y)=(bd)(y(x/y))=bdx$, the result follows.

We prove 8-3.3(j). We have $bd((a/b)+(c/d))=bd(a/b)+bd(c/d)=d(b(a/b))+b(d(c/d))=da+bc=ad+bc$.
The result follows from the definition of $(ad+bc)/bd$.

We prove 8-3.3(k). Using (d), we have $bd((a/b)-(c/d))=bd(a/b)-bd(c/d)=d(b(a/b))-b(d(c/d))=da-bc=ad-bc$.
The result follows from the definition of $(ad-bc)/bd$.
\end{solution}

\begin{exercise}
Let $S$ be a subset of a field $F$. Prove that the following conditions are necessary and
sufficient for $S$ to be a subfield:
\begin{enumalpha}
    \item if $s,t\in S$, then $s-t\in S$;
    \item if $s,t\in S$ and $t\neq0$, then $s/t\in S$; and
    \item $S$ has at least two elements.
\end{enumalpha}
On the other hand, show that if (b) is replaced by
\begin{enumalpha}
    \item[(b')] if $s,t\in S$, then $st\in S$,
\end{enumalpha}
then (a), (b'), and (c) are not sufficient for $S$ to be a subfield.
\end{exercise}

\begin{solution}
Suppose $S$ is a subfield. Let $s,t\in S$.
Since $S$ is a subfield, choose $c\in S$ such that $c+t=s$.
Now $s-t=(c+t)-t=c+(t-t)=c+0=c\in S$. Thus (a) holds.
If $t\neq 0$, choose $d\in S$ such that $td=s$.
Then $s/t=(td)/(t1)=(t/t)\cdot(d/1)=1d=d\in S$. Thus (b) holds.
Finally, (c) holds by definition of a field.

Conversely, suppose (a), (b), and (c) hold for a subset $S$ of $F$.
Let $a$ be the restriction of $+$ to $S\times S$ and let $m$ be the restriction of $\cdot$ to $S\times S$.
We show that $\aprn{S,a,m}$ is a field.
We first show that $\ran a\sse S$ and $\ran m\sse S$. Let $s,t\in S$.
Then $0=t-t\in S$, $-t=0-t\in S$, and $s+t=s-(-t)\in S$ by successive applications of (a).
Thus $\ran a\sse S$. If $t=0$, then $st=0\in S$. If $t\neq 0$, then $1=t/t\in S$, $1/t\in S$,
and $st=s/(1/t)\in S$ by successive applications of (b). Hence $\ran m\sse S$.

Now, note that the associative, commutative, and distributive laws automatically hold in $S$.
Since $S$ contains at least two elements, it contains at least one element $t\neq 0$.
The previous argument then shows that $0\in S$ and $1\in S$. Then the first and
last parts of postulate (iv) for a field hold automatically.
THe second and third parts are satisfied because we showed previously that $t\in S\implies -t\in S$
and $0\neq t\in S\implies 1/t\in S$.

The the integers are a subset of the real numbers which satisfy (a), (b'), and (c), but integers
are not a subfield of $\R$.
\end{solution}

\begin{exercise}
Suppose that $F$ and $G$ are fields and that $\phi$ is a function from $F$ to $G$ such that
\[\phi(a+b)=\phi(a)+\phi(b) \qquad \text{and} \qquad \phi(ab)=\phi(a)\phi(b).\]
Prove:
\begin{enumalpha}
    \item $\phi(a-b)=\phi(a)-\phi(b)$ and $\phi(0)=0$;
    \item either $\ran \phi=\set{0}$ or $\phi$ is injective and $\ran\phi$ is a subfield of $G$;
    \item in the latter case, $\phi(1)=1$ and $\phi(a/b)=\phi(a)/\phi(b)$ if $b\neq 0$.
\end{enumalpha}
(Note that the operation signs on the left refer to $F$ and those on the right to $G$. Universal
quantification over $F$ is understood.)
\end{exercise}

\begin{solution}
\begin{enumalpha}
    \item We have $\phi(a-b)+\phi(b)=\phi((a-b)+b)=\phi(a)$, so $\phi(a)-\phi(b) = \phi(a-b)$.
    We have $\phi(0)+\phi(0)=\phi(0+0)=\phi(0)=\phi(0)+0$, so $\phi(0)=0$.
    
    \item If $\phi$ is not injective, then choose $x,y\in F$ such that $x\neq y$ and
    $\phi(x)=\phi(y)$. Then $\phi(x-y)=\phi(x)-\phi(y)=0$, and $x-y\neq 0$.
    Let $c=x-y$, and choose $d\in F$ such that $cd=1$.
    Let $z\in F$ be arbitrary. Then $\phi(z)=\phi(1z)=\phi(cdz)=\phi(c)\phi(d)\phi(z)=0\phi(d)\phi(z)=0$.
    Thus $\ran\phi=\set{0}$.
    
    Now suppose $\ran\phi\neq\set{0}$. Then $\phi$ is injective. We use the previous exercise
    to show that $\ran\phi$ is a subfield of $G$.
    Suppose $p,q\in\ran\phi$. Choose $a,b\in F$ so that $\phi(a)=p$ and
    $\phi(b)=q$. Then $p-q=\phi(a)-\phi(b)=\phi(a-b)$.
    If $q\neq 0$, then $\phi(b)\neq 0$
    which implies $b\neq 0$ since $\phi$ is injective.
    Then we have $\phi(a/b)\phi(b)=\phi((a/b)b)=\phi(a)$,
    whence $\phi(a/b)=\phi(a)/\phi(b)=p/q$. 
    Finally, we have $\phi(1)=\phi(1/1)=\phi(1)/\phi(1)=1$. Since $0,1\in\ran\phi$, $\ran\phi$ has
    at least two elements.
    By the previous exercise, $\ran\phi$ is a subfield of $G$.

    \item This was shown in (b).
\end{enumalpha}
\end{solution}

\section{Whole numbers and rational numbers}
\begin{exercise}
If $F$ is a modular field, then $I(F) = N(F)$.
\end{exercise}

\begin{solution}
Suppose $F$ is a modular field, meaning $0\in N(F)$.
We must show that $I(F)=N(F)$. We already have $N(F)\sse I(F)$ by definition of $I(F)$.
Suppose $x\in I(F)$.
From Lemma 8-4.5, we can write $x=u-v$ for some $u,v\in N(F)$.
Consider the set $B=\setb{z}{u-z\in N(F)}$. We show by induction that $N(F)\sse B$.
We shall use the fact that for all $n\in N(F)$, $n-1=0$ or $n-1\in N(F)$, which was shown
in the proof of Lemma 8-4.5.
Since $u\in N(F)$, we have $u-1=0\in N(F)$ or $u-1\in N(F)$, whence $1\in B$.
Suppose $z\in B$.
We have $u-(z+1)=(u-z)-1$. We have $u-z\in N(F)$ since $z\in B$.
But then $(u-z)-1=0\in N(F)$ or $(u-z)-1\in N(F)$. Hence $z+1\in B$.
Therefore $N(F)\sse B$.
Since $v\in N(F)\sse B$, we have $x=u-v\in N(F)$.
Therefore $I(F)\sse N(F)$.
\end{solution}

\begin{exercise}
If $F$ is a modular field, then $Q(F) = N(F)$.
\end{exercise}

\begin{solution}
Suppose $a\in N(F)-\set{0}$. Then $x \mapsto ax$ is a function from $N(F)$ to $N(F)$ by
8-4.3 and it is injective by 8-3.1(d). Since $N(F)$ is finite, this function must also be
surjective, so we can choose $b\in N(F)$ so that $ab = 1$. In other words, $1/a \in N(F)$. We
have proved $(\forall a\in N(F)-\set{0})1/a\in N(F)$. Now $Q(F) = N(F)$ follows from the
result of \hyperref[ex:8-4.1]{Exercise 1}, 8-4.3, and the definition of $Q(F)$.

The solution above relies on the fact that $N(F)$ is finite. To see this, consider the function
$g$ that maps the natural numbers onto $N(F)$. In the text, it is shown that for a modular field,
there is an $n$ such that $g(n)=1$. One can then prove by induction that
for any $k\in N(F)$, we have $g(k)=g(m)$ for some $1\leq m < n$. Therefore $N(F)=\ran g$
is the finite set $\setb{g(m)}{1\leq m < n}$.
\end{solution}

\begin{exercise}
Prove the existence of a least subfield in any field by an argument analogous to the
proof of 7-2.1. Be sure to note how the result of Exercise 5, p. 98, is involved.
\end{exercise}

\begin{solution}
Let $\mc B$ be the family of all subfields of a field $F$. Note that $F$ itself is a subfield.
Let $G=\bigcap_{B\in\mc B}B$. We show that $G$ is a subfield, in which case it is clearly
the smallest subfield.

Let $s,t\in G$. Then $s,t\in B$ for all $B\in\mc B$. Then $s-t\in B$ for all $B\in\mc B$.
Hence $s-t\in G$.
If $t\neq 0$, then $s/t\in B$ for all $B\in\mc B$.
Hence $s/t\in G$.
Finally, from Exercise 5, p. 98, we know that all subfields contain the same identity
elements $0$ and $1$ as $F$. Therefore $0,1\in G$ and $G$ is a subfield.
\end{solution}

\begin{exercise}
Suppose $F$ and $G$ are fields of characteristic 0.
\begin{enumalpha}
    \item Show that there exists a unique function $\phi$ from $N(F)$ to $N(G)$ such that
    \begin{enumerate}
        \item[(i)] $\phi(1)=1$
        \item[(ii)] $(\forall x)\phi(x+1)=\phi(x)+1$.
    \end{enumerate}
    Furthermore, show that $\phi$ is a bijection,
    \begin{enumroman}
        \item[(iii)] $(\forall x,y)\phi(x+y)=\phi(x)+\phi(y)$, and
        \item[(iv)] $(\forall x,y)\phi(xy)=\phi(x)\phi(y)$.
    \end{enumroman}

    \item Show that $\phi$ can be extended to be a bijection from $I(F)$ to $I(G)$ so that (iii) and
    (iv) remain valid. (We keep the same symbol for the extended function. Since the
    domain of the quantifiers is understood to be the domain of $\phi$, when $\phi$ is extended,
    (iii) and (iv) become more inclusive.)

    \item Show that $\phi$ can be further extended to be a bijection from $Q(F)$ to $Q(G)$ so that
    (iii) and (iv) remain valid.

    \item Finish the proof of Theorem 8-4.11.
\end{enumalpha}
\end{exercise}

\begin{solution}
\begin{enumalpha}
    \item From Proposition 8-4.10 we know that $N(F)$ and $N(G)$ are simple chains with
    successor functions $x\mapsto x+1$ and first elements $1$.
    From Theorem 7-3.8, there is a unique isomorphism $\phi$
    from $N(F)$ to $N(G)$, and in fact $\phi(1)=1$ and $\phi(x+1)=\phi(x)+1$ for all $x$.
    We are left to prove (iii) and (iv).

    Let $x\in N(F)$ be given. We show by induction on $y$ that $\phi(x+y)=\phi(x)+\phi(y)$
    for all $y\in N(F)$. We already have $\phi(x+1)=\phi(x)+1$. Suppose
    $\phi(x+z)=\phi(x)+\phi(z)$. Then $\phi(x+z+1)=\phi(x+z)+1=\phi(x)+\phi(z)+1=\phi(x)+\phi(z+1)$.
    Therefore $\phi(x+y)=\phi(x)+\phi(y)$ for all $x$ and $y$.

    Let $x\in N(F)$ be given. We show by induction on $y$ that $\phi(xy)=\phi(x)\phi(y)$
    for all $y\in N(F)$. We already have $\phi(x1)=\phi(x)=\phi(x)1=\phi(x)\phi(1)$. Suppose
    $\phi(xz)=\phi(x)\phi(z)$. Then $\phi(x(z+1))=\phi(xz+x)=\phi(xz)+\phi(x)=\phi(x)\phi(z)+\phi(x)
    =\phi(x)(\phi(z)+1)=\phi(x)\phi(z+1)$.
    Therefore $\phi(xy)=\phi(x)\phi(y)$ for all $x$ and $y$.

    \item Let $h=\setb{\aprn{u-v,\phi(u)-\phi(v)}}{u,v\in N(F)}$.

    First we show that $h$ is a function. Suppose
    \[\aprn{u_1-v_1,\phi(u_1)-\phi(v_1)}\in h\qquad \text{and} \qquad
    \aprn{u_2-v_2,\phi(u_2)-\phi(v_2)}\in h\] where $u_1,v_1,u_2,v_2\in N(F)$ and
    $u_1-v_1=u_2-v_2$. Then $u_1+v_2=u_2+v_1$,
    so $\phi(u_1)+\phi(v_2)=\phi(u_1+v_2)=\phi(u_2+v_1)=\phi(u_2)+\phi(v_1)$.
    Then $\phi(u_1)-\phi(v_1)=\phi(u_2)-\phi(v_2)$. So $h$ is a function.
    The previous argument can be reversed to show that $h$ is injective since $\phi$
    is injective.
    From Lemma 8-4.5 we have $\dom h=I(F)$. Also, $\ran h=I(G)$ since $\phi$ is bijective.
    Therefore $h$ is a bijection from $I(F)$ to $I(G)$.

    Now we compute $h(u_1-v_1)+h(u_2-v_2)=\phi(u_1)-\phi(v_1)+\phi(u_2)-\phi(v_2)
    =\phi(u_1)+\phi(u_2)-(\phi(v_1)+\phi(v_2))=\phi(u_1+u_2)-\phi(v_1+v_2)=
    h((u_1+u_2)-(v_1+v_2))=h((u_1-v_1)+(u_2-v_2))$. Hence (iii) holds.
    We also compute $h(u_1-v_1)h(u_2-v_2)=(\phi(u_1)-\phi(v_1))(\phi(u_2)-\phi(v_2))
    =\phi(u_1)\phi(u_2)-\phi(u_1)\phi(v_2)-\phi(v_1)\phi(u_2)+\phi(v_1)\phi(v_2)
    =\phi(u_1u_2)-\phi(u_1v_2)-\phi(v_1u_2)+\phi(v_1v_2)
    =h(u_1u_2-u_1v_2)+h(v_1v_2-v_1u_2)=h(u_1u_2-u_1v_2 + v_1v_2-v_1u_2)=h((u_1-v_1)(u_2-v_2))$.
    Hence (iv) holds.

    Finally $h$ is an extension of $\phi$ since $h(n)=h((n+1)-1)=\phi(n+1)-1=\phi(n)+1-1=\phi(n)$
    for all $n\in N(F)$.
    
    \item Let $k=\setb{\aprn{r/s,h(r)/h(s)}}{r,s\in I(F)\land s\neq 0}$.
    Note that $h(0)=h(1-1)=h(1)-h(1)=0$, so $h(s)\neq 0$ if $s\neq 0$.

    First we show that $k$ is a function. Suppose
    \[\aprn{u_1/v_1,h(u_1)/h(v_1)}\in h\qquad \text{and} \qquad
    \aprn{u_2/v_2,h(u_2)/h(v_2)}\in h\]
    where $u_1,v_1,u_2,v_2\in I(F)$, $v_1\neq 0$, $v_2\neq 0$, and $u_1/v_1=u_2/v_2$.
    Then $u_1v_2=u_2v_1$. Then $h(u_1)h(v_2)=h(u_1v_2)=h(u_2v_1)=h(u_2)h(v_1)$.
    Then $h(u_1)/h(v_1)=h(u_2)/h(v_2)$. Hence $k$ is a function.
    The previous argument can be reversed to show that $k$ is injective since $h$ is injective.
    It follows from the definition of $Q(F)$ and $Q(G)$ that $\dom k=Q(F)$ and $\ran k=Q(G)$
    since $h$ is bijective.
    Therefore $k$ is a bijection from $Q(F)$ to $Q(G)$.

    Now we compute $k(u_1/v_1+u_2/v_2)=k((u_1v_2+u_2v_1)/(v_1v_2))=h(u_1v_2+u_2v_1)/h(v_1v_2)
    =(h(u_1)h(v_2)+h(u_2)h(v_1))/(h(v_1)h(v_2))=h(u_1)/h(v_1) + h(u_2)/h(v_2)=k(u_1/v_1)+k(u_2/v_2)$.
    So (iii) holds.
    We also have $k(u_1/v_1)k(u_2/v_2)=(h(u_1)/h(v_1))(h(u_2)/h(v_2))=h(u_1u_2)/h(v_1v_2)$
    and $k((u_1/v_1)(u_2/v_2))=k((u_1u_2)/(v_1v_2))=h(u_1u_2)/h(v_1v_2)$.
    So (iv) holds.

    Finally, $k$ is an extension of $h$ since $k(i)=k(i/1)=h(i)/h(1)=h(i)/1=h(i)$ for all $i\in I(F)$.

    \item In (c) we showed that $Q(F)$ and $Q(G)$ are isomorphic with isomorphism $k$.
    Let $\psi$ be any isomorphism from $Q(F)$ to $Q(G)$.
    We use \hyperref[ex:8-3.3]{Exercise 8-3.3}.

    First we use induction to show that $\psi$ is an extension of $\phi$.
    We have $\psi(1)=1=\phi(1)$.
    If $n\in N(F)$ and $\psi(n)=\phi(n)$, then $\psi(n+1)=\psi(n)+\psi(1)=\phi(n)+\phi(1)=\phi(n+1)$.
    By induction, $\psi$ is an extension of $\phi$.

    Now if $u,v\in N(F)$, then $\psi(u-v)=\psi(u)-\psi(v)=\phi(u)-\phi(v)=h(u-v)$.
    Therefore $\psi$ is an extension of $h$. 

    Finally, if $r,s\in I(F)$ and $s\neq 0$, then $\psi(r/s)=\psi(r)/\psi(s)=\phi(r)/\phi(s)
    =k(r/s)$. Therefore $\psi=k$. So the isomorphism from $Q(F)$ to $Q(G)$ is unique.
\end{enumalpha}
\end{solution}

\section{Ordered fields}
\begin{exercise}
Let $F$ be any field of characteristic $0$. Prove that $Q(F)$ can be ordered in exactly
one way. Deduce that if $F$ and $G$ are ordered fields and $\phi$ is the isomorphism of $Q(F)$
onto $Q(G)$ (Theorem 8-4.11), then $\phi$ is order-preserving. (Note. Although the prime
subfield of a field of characteristic $0$ can always be ordered in exactly one way, the field
itself may not be orderable or it may be orderable in several ways.)
\end{exercise}

\begin{solution}
Suppose $P$ is the set of all positive elements in some ordering of $Q(F)$. Let
\[S(F) = \setb{q}{(\exists x, y\in N(F))\thinspace q = x/y}.\]
We shall prove $P = S$, hence that the set of positive
elements of $Q(F)$ is uniquely determined, and thus $Q(F)$ can be ordered in only one way.
Since $Q(F)$ is a subfield of $F$, it follows easily from the definition of $N(F)$ that $N(Q(F)) =
N(F)$. By 8-5.6 $N(F) = N(Q(F)) \sse P$. By 8-5.5(k) and 8-5.1(d), if $x,y \in N(F)$,
then $x/y \in P$. Hence $S\sse P$. Let $q\in P$. We can write $q = x/y$, where $x, y\in I(F)$.
Now $y\neq 0$, so either $y \in N(F)$ or $-y\in N(F)$. Since we have also $q = (-x)/(-y)$
we can arrange that $y\in N(F)$. If $x \nin N(F)$, then $-x\in N(F)$ ($x\neq 0$ because $q\in P$
and therefore $q\neq 0$), so $-q \in S \sse P$; but we cannot have both $q\in P$ and $-q\in P$.
Therefore $x\in N(F)$ and $q\in S$. This proves $P\sse S$.

This proves that $Q(F)$ can be ordered in at most one way since $x < y$ iff $0<y-x$ iff $y-x\in P=S(F)$.

Let $F$ and $G$ be ordered fields.
Now, the isomorphism $\phi$ from $Q(F)$ to $Q(G)$ was unique, and the construction showed
that it takes $N(F)$ onto $N(G)$. Hence it takes $S(F)$ onto $S(G)$. Since $S(F) = P(F)\cap Q(F)$
as shown above and $S(G)= P(G) \cap Q(G)$, $\phi$ must be order-preserving. For if $x,y\in Q(F)$
and $x<y$, then $y-x\in P(F)\cap Q(F)=S(F)$. But then
$\phi(y)-\phi(x)=\phi(y-x)\in \phi(S(F))=S(G)\sse P(G)$, which means $\phi(x)<\phi(y)$.

Note that we have not proved that $Q(F)$ can always be ordered for a field $F$ of characteristic 0.
Once we construct an ordered field $G$ of characteristic 0 in Chapter 9,
we can define an order for any prime subfield $Q(F)$ by $x<y$ iff $\phi(x)<\phi(y)$, where $\phi$ is the
unique isomorphism from $Q(F)$ to $Q(G)$.
\end{solution}

\section{Archimedean ordered fields}

\begin{exercise}
Prove that if an ordered field has any one of the properties (a), (b), (c) of 8-6.2,
then it is Archimedean.
\end{exercise}

\begin{solution}
Let $F$ be an ordered field.

Suppose $F$ has property (a), that is,
$(\forall x > 0)(\forall y)(\exists n\in N(F)) nx > y$. Since $1>0$, we have
$(\forall y)(\exists n\in N(F)) n = n1 > y$. Hence $F$ is Archimedean.

Suppose $F$ has property (b), that is,
$(\forall x > 0)(\exists n\in N(F)) 1/n > x$.
Let $y\in F$ be given. Assume $y > 0$. Then $1/y > 0$, and there exists $n\in N(F)$
such that $1/n < 1/y$. Then $n > y$. If $y\leq 0$, we can just choose $n=1$ so that $n > y$.
We have shown $(\forall y)(\exists n\in N(F)) n > y$. Hence $F$ is Archimedean.

Suppose $F$ has property (c), that is, $x<y$ implies $(\exists r\in Q(F))\thinspace x < r < y$.
Let $z\in F$ be given.
If $z < 1$, we can just choose $n=1$ to get $n>z$. So assume $z\geq 1$.
Since $z < z + 1$, there exists $r\in Q(F)$ such that $z < r < z + 1$.
Choose $s,t\in I(F)$ so that $t\neq 0$ and $r = s/t$. Since $r=(-s)/(-t)$, we may assume
$t \in N(F)$. Then $1\leq t$, and since $1\leq z < r$, we have $1\leq z < r < rt = s$.
Since $s$ is positive, we have $s\in N(F)$.
Hence $F$ is Archimedean.
\end{solution}

\begin{exercise}
Let $F$ be any field of characteristic $0$. Show that the unique order which can be
imposed on $Q(F)$ (Exercise, p. 107) is Archimedean.
\end{exercise}

\begin{solution}
Let $x\in Q(F)$. If $x< 1$, we can choose $n=1>x$. So assume $1\leq x$.
Then there exists $s,t\in N(F)$ such that $x=s/t$.
Now since $1\leq x$ and $1\leq t$, we have $x\leq xt=s < s + 1$.
Note that $s+1\in N(F)$.
\end{solution}

\begin{exercise}
Show that the ordered field of rational functions discussed in 8-5.1 is not Archimedean.
\end{exercise}

\begin{solution}
Let $F$ be the field of rational functions ordered by the first order relation given in the text. The
members of $N(F)$ are the constant functions with positive integral values. Evidently the
identity function exceeds all of these, so $F$ is not Archimedean.
\end{solution} 

\section{Complete ordered fields}

