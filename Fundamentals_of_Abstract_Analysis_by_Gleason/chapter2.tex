\chapter{Logic}
\section{Propositions and logical connectives}
\section{Tautologies}
\section{The conditional}
\begin{exercise}
Assume that the following propositions are true: $x\in A$, $x\nin B$, $x\in C$, $y\in A$,
$y\in B$, $y\nin C$, $z\nin A$, $z\nin B$, and $z\in C$. Which of the following are true?
\begin{enumalpha}
    \item $y\nin B\lor z\in A$.
    \item $x\nin C\land y\in B$.
    \item $x\in C$ implies $z\in A$.
    \item $(x\in A\lor y\in B)$ implies $z\nin C$.
    \item $x\nin A$ if and only if $y\in C$.
    \item If $x\in A$ implies $z\in C$, then $y\nin A$ implies $z\in B$.
    \item If $z\in A$ implies $x\in B$, then $y\in B$.
    \item If $x\nin B$, then $y\in C$, $z\in B$, and $x\nin A$.
    \item $x\nin A$ only if $y\nin B$.
    \item $z\in C$ is a necessary condition for $y\in C$.
    \item $z\in C$ is a sufficient condition for $x\in A$.
    \item In order that $z\nin C$, it is necessary and sufficient that $x\in A$ imply $y\in B$.
\end{enumalpha}
\end{exercise}

\begin{solution}
We can replace each basic (non)membership proposition with $T$ or $F$ based on
the propositions assumed to be true. That will make it easier to determine whether
each proposition as a whole is true.
\begin{enumalpha}
    \item $F\lor F$.
    \item $F\land T$.
    \item $T$ implies $F$.
    \item ($T\lor T$) implies $F$.
    \item $F$ if and only if $F$.
    \item If $T$ implies $T$, then $F$ implies $F$.
    \item If $F$ implies $F$, then $T$.
    \item If $T$, then $F$, $F$, and $F$.
    \item $F$ only if $F$.
    \item $T$ is a necessary condition for $F$.
    \item $T$ is a sufficient condition for $T$.
    \item In order that $F$, it is necessary and sufficient that $T$ imply $T$.
\end{enumalpha}
The propositions (e), (f), (g), (i), (j), (k), and (l) are true.
\end{solution}

\begin{exercise}
Suppose it is known that one of the following cases is true:

Case I. $x\in A$, $x\in B$, $x\nin C$, $y\in A$, $y\nin B$, and $y\in C$;

Case II. $x\nin A$, $x\in B$, $x\in C$, $y\nin A$, $y\nin B$, and $y\in C$.

Which of the following propositions are certainly true? Which are certainly false?

\begin{enumalpha}
    \item $x\in C$ if and only if $y\in C$.
    \item ($x\in B\land y\in A$) if and only if ($x\in A\lor y\in B$).
    \item If $x\in A\land y\in C$, then $x\in C\lor y\in A$.
    \item ($x\in B\land y\in B$) or ($x\in C\land y\in C$).
    \item not-(if $x\in A$, then $y\in B$) implies ($x\in C\land y\in B$).
\end{enumalpha}
\end{exercise}

\begin{solution}
This is like the previous exercise, except we need to determine which of the propositions is true
for each of the two cases. Then the propositions which are certainly true are those that are true for both cases,
and the propositions which are certainly false are those that are false for both cases.

For Case I, the propositions simplify to the following.

\begin{enumalpha}
    \item $F$ if and only if $T$.
    \item ($T\land T$) if and only if ($T\lor F$).
    \item If $T\land T$, then $F\lor T$.
    \item ($T\land F$) or ($F\land T$).
    \item not-(if $T$, then $F$) implies ($F\land F$).
\end{enumalpha}

For Case I, propositions (b) and (c) are true.

For Case II, the propositions simplify to the following.

\begin{enumalpha}
    \item $T$ if and only if $T$.
    \item ($T\land F$) if and only if ($F\lor F$).
    \item If $F\land T$, then $T\lor F$.
    \item ($T\land F$) or ($T\land T$).
    \item not-(if $F$, then $F$) implies ($T\land F$).
\end{enumalpha}

For Case II, propositions (a), (b), (c), (d), and (e) are true.

Therefore, knowing that Case I or Case II is true,
the propositions (b) and (c) are certainly true. None of the others are certainly false.
\end{solution}

\begin{exercise}
Suppose that the following propositions are true:
\begin{align*}
    &\text{if $x\in A\land y\in B$, then $y\in A$;}\\
    &\text{if $x\nin A\lor y\in A$, then $y\nin B$.}
\end{align*}
What simple conclusion can we draw?
\end{exercise}

\begin{solution}
We can conclude $y\nin B$. For suppose that $y\in B$. Either $x\in A\lor x\nin A$.
If $x\in A$, then the first hypothesis gives $y\in A$, from which the second hypothesis gives
$y\nin B$.
If $x\nin A$, then the second hypothesis again gives $y\nin B$.
In both cases we have a contradiction.
Therefore $y\nin B$.

In fact, if we assume $y\nin B$, then the first hypothesis is true because the antecedent is false,
and the second hypothesis is true because the consequent is true.
Therefore $y\nin B$ is true if and only if both hypotheses are true, so $y\nin B$ is the
best conclusion we can draw.

This result can also be derived with a truth table.
If $p$ stands for $x\in A$, $q$ for $y\in B$, and $r$ for $y\in A$, the hypotheses are
$(p\land q)\implies r$ and $(\lnot p\lor r)\implies\lnot q$.
We then have the following truth table.
\[
\begin{array}{ @{\makebox[2em][c]{\rownumber\space}} ccccc } 
p & q & r & (p\land q)\implies r & (\lnot p\lor r)\implies\lnot q 
\gdef\rownumber{\stepcounter{magicrownumbers}(\arabic{magicrownumbers})} \\
\hline
T & T & T & T & F \\ 
T & T & F & F & T \\ 
T & F & T & T & T \\ 
T & F & F & T & T \\ 
F & T & T & T & F \\ 
F & T & F & T & F \\ 
F & F & T & T & T \\ 
F & F & F & T & T \\ 
\end{array}
\]
If we consider only the rows for which both hypotheses are true, we are left with
exactly the rows for which $q$ is false: (3), (4), (7), and (8).
\end{solution}


\section{Propositional schemes and quantifiers}

\begin{exercise}
Express in terms of the ordinary quantifiers:
\begin{enumalpha}
    \item There are at most two $x$'s such that $P(x)$.
    \item There are at least two $x$'s such that $P(x)$.
\end{enumalpha}
\end{exercise}

\begin{solution}
\begin{enumalpha}
    \item $(\forall x,y,z)\prn{P(x)\land P(y)\land P(z)}\implies (x=y\lor x=z\lor y=z)$
    \item $(\exists x,y)\prn{x\neq y \land P(x)\land P(y)}$
\end{enumalpha}
\end{solution}

\begin{exercise}
Let $\phi(x, y)$ mean $x$ is a parent of $y$. Let $M$ be the set of all men (living or dead)
and $W$ be the set of all women.

The proposition, \textit{$a$ is the mother of $b$}, can be written \textit{$a\in W$ and $\phi(a, b)$}, while \textit{$a$ is the
brother (possibly half-brother) of $b$} can be written \textit{$a\in M$ and $a\neq b$ and $(\exists x) (\phi(x, a)$ and
$\phi(x, b))$}.

Because ordinary language is less precise than the language of quantified statements
and because there are different ways the biological facts can enter the interpretations of
the problems, the answers to the following problems are not all unique.

Express with quantifiers
\begin{enumalpha}
    \item $a$ is the grandfather of $b$.
    \item $a$ is the grandson of $b$.
    \item $a$ is the aunt of $b$. 
    \item $a$ and $b$ are sisters.
    \item $a$ and $d$ are first cousins. 
    \item $a$ has neither brothers nor sisters.
    \item $a$ is the full brother of $b$.
    \item Every person has a father.
    \item Every person has at most one father.
    \item Every person has exactly two parents.
    \item Some person has at least two children.
    \item Every person has a grandfather.
    \item No one is his own grandfather.
\end{enumalpha}

Let $\psi(x, y)$ mean \textit{$x$ is the ancestor of $y$}. Express $\phi(a,b)$ in terms of $\psi(x,y)$ and
quantifiers. Can you express $\psi(a, b)$ in terms of $\phi(x,y)$ and quantifiers?
\end{exercise}

\begin{solution}
\begin{enumalpha}
    \item $a\in M\land (\exists x)\prn{\phi(a,x)\land \phi(x, b)}$
    \item $a\in M\land (\exists x)\prn{\phi(b,x)\land \phi(x, a)}$
    \item $a\in W\land (\exists x, y)\prn{\phi(x,b)\land \phi(y, x)\land \phi(y,a)\land x\neq a}$
    \item $a,b\in W\land a\neq b\land (\exists x)\prn{\phi(x, a)\land \phi(x, b)}$
    \item $a\neq b\land (\exists x, y, z)\prn{x\neq y\land \phi(x,a)\land \phi(y,b)\land \phi(z, x)\land \phi(z,y)}$
    \item $(\forall x)\prn{(\exists y)(\phi(y,x)\land \phi(y, a))\implies x = a}$
    \item $a\in M\land a\neq b\land (\exists x,y)(x\neq y\land \phi(x,a)\land \phi(x,b)\land \phi(y,a)\land \phi(y,b))$
    \item $(\forall x)(\exists y\in M)\phi(y, x)$
    \item $(\forall x)(\forall y,z\in M)(\phi(y, x)\land \phi(z,x) \implies y = z)$
    \item $(\forall x)(\exists y,z)(y\neq z\land \phi(y, x)\land \phi(z,x)
    \land (\forall w)(\phi(w,x)\implies (w=y\lor w=z)))$
    \item $(\exists x,y,z)(y\neq z\land \phi(x,y)\land \phi(x,z))$
    \item $(\forall x)(\exists z\in M)(\exists y)(\phi(y, x)\land \phi(z,y))$
    \item $(\forall x,y,z)((\phi(y, x)\land \phi(z,y)\land z\in M) \implies x\neq z)$
\end{enumalpha}

$\phi(a,b)$ can be expressed as $\psi(a,b)\land \lnotd(\exists x)(\psi(a,x)\land \psi(x,b))$.

It is impossible to express $\psi(a, b)$ in terms of $\phi(x, y)$ and quantifiers.
\end{solution}


\section{Proof and inference}

\begin{exercise}
Refer to Exercise 2, Section 2-4. Prove formally that (h) implies (l).
\end{exercise}

\begin{solution}
The hypothesis is $(\forall x)(\exists y\in M)\phi(y, x)$.

\begin{enumroman}
    \item Let $a$ be any person.
    \item \quad $(\exists y\in M)\phi(y, a)$ \hfill by hypothesis;
    \item \quad choose $b$ so that $\phi(b,a)$ \hfill by (ii);
    \item \quad $(\exists y\in M)\phi(y, b)$ \hfill by hypothesis;
    \item \quad choose $c$ so that $c\in M$ and $\phi(c,b)$ \hfill by (iv);
    \item \quad $c\in M\land (\exists y)(\phi(c, y)\land \phi(y, a))$ \hfill by (iii) and (v);
    \item \quad $(\exists z\in M)(\exists y)(\phi(z,y)\land \phi(y, a))$ \hfill by (vi);
    \item $(\forall x)(\exists z\in M)(\exists y)(\phi(z,y)\land \phi(y, x))$ \hfill by (i) through (vii).
\end{enumroman}
\end{solution}

\section{Set formation}

\begin{exercise}
Which of the following sets are the same? What inclusion relations hold among
these sets? (All variables have domain $\R$.)

\begin{tasks}[label=](2)
    \task $A_1=\setb{x}{0 < x <2}$
    \task $A_2=\setb{x}{\abs{x} < 1}$
    \task $A_3=\setb{x}{x < x^2}$
    \task $A_4=\setb{x}{(\exists y)x=y^2}$
    \task $A_5=\setb{x}{(\exists y)x=1/y}$
    \task $A_6=\setb{x}{(\exists y)x=3y}$
    \task $A_7=\setb{x}{(\exists y,z)x=y^2 \land y+z^2 < 1}$
    \task $A_8=\setb{x}{(\forall y)\abs{x-y} = x - y}$
    \task $A_9=\setb{x}{(\forall y)\abs{x-y} = \abs{y} - \abs{x}}$
    \task $A_{10}=\setb{x}{(\exists y)y^2 + xy + x = 1}$
\end{tasks}
\end{exercise}

\begin{solution}
We have the following equalities:
\begin{tasks}[label=](2)
    \task $A_1=\setb{x}{0 < x <2}$
    \task $A_2=\setb{x}{-1 < x < 1}$
    \task $A_3=\setb{x}{x < 0}\cup \setb{x}{x > 1}$
    \task $A_4=\setb{x}{x\geq 0}$
    \task $A_5=\setb{x}{x\neq 0}$
    \task $A_6=\R$
    \task $A_7=\setb{x}{x\geq 0}$
    \task $A_8=\eset$
    \task $A_9=\set{0}$
    \task $A_{10}=\R$
\end{tasks}

So we have the following set equalities and inclusions:
\begin{itemize}
    \item $\eset = A_8 \sss A_9 \sss A_1 \sss A_4 = A_7 \sss A_6 = A_{10} = \R$
    \item $\eset \sss A_3 \sss A_5 \sss \R$
    \item $A_9 \sss A_2 \sss \R$
\end{itemize}
\end{solution}




\section{The set-theoretic paradoxes}
\section{Dummy variables}
