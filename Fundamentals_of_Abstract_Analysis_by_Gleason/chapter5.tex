\chapter{Equivalence}
\section{Equivalence relations and partitions}

\begin{exercise}
    Consider the relation, ``is a brother of.'' Is it reflexive? symmetric? transitive?
\end{exercise}

\begin{solution}
It is not reflexive, since you would not say that Bob is a brother of himself.

It is not symmetric, since Bob being a brother of Sarah does not imply Sarah is a brother of Bob.

It is not transitive, since Bob could have the same father but not the same mother as Joe,
and Jack could have the same mother but not the same father as Joe. Then Bob is a brother of Joe,
and Joe is a brother of Jack, but Bob is not a brother of Jack.
\end{solution}

\begin{exercise}
Give an example of a relation in a set which is symmetric and transitive but not
reflexive.
\end{exercise}

\begin{solution}
If $A=\set{1,2}$, then $R=\set{\aprn{1,1}}$ is symmetric and transitive but not reflexive.

In general, if $A$ is not void, then $\eset$ is a relation in $A$ which is both
symmetric and transitive but not reflexive.
\end{solution}

\begin{exercise}
Let $E$ be an equivalence relation in the set $A$ and let $\phi$ be the corresponding quotient
map. Prove that
\[a\Erel b\iff \phi(a)\cap\phi(b)\neq\eset.\]
\end{exercise}

\begin{solution}
Suppose $a\Erel b$. Then $\phi(a)=\phi(b)$ by Proposition 5-1.3. Since $a\in\phi(a)=\phi(a)\cap\phi(b)$,
we have $\phi(a)\cap\phi(b)\neq\eset$.
Conversely, suppose $\phi(a)\cap\phi(b)\neq\eset$. Choose $x\in \phi(a)\cap\phi(b)$.
Then $a\Erel x$ and $b\Erel x$, so $a\Erel b$.
\end{solution}

\begin{exercise}
List the partitions of the set $\set{1, 2,3}$. These fall into equivalence classes under the
relation of isomorphism (for configurations consisting of a basic set and a partition of it).
Exhibit this division into classes.
\end{exercise}

\begin{solution}
There are five partitions of $\set{1,2,3}$, shown here grouped into three isomorphism classes:
\[
\set{\set{1,2,3}}\qquad
\set{\set{1},\set{2},\set{3}}\qquad
\set{\set{1},\set{2,3}},\thickspace\set{\set{2},\set{1,3}},\thickspace\set{\set{3},\set{2,1}}
\]
\end{solution}

\begin{exercise}
Write out the omitted parts of the proof of Theorem 5-1.5 in the detailed fashion
of Section 2-5.
\end{exercise}

\begin{solution}
To give a formal proof we must translate the hypotheses into formally quantified statements.
$\mc P$ \textit{is a partition of} $A$ becomes
\begin{enumarabic}
    \item $\mc P\sse \power(A)-\set{\eset}$ and
    \item $(\forall x\in A)(\exists Y\in\mc P)(x\in Y\land (\forall Z\in\mc P)x\in Z\implies Z=Y)$.
\end{enumarabic}
The definition of $F$ is
\begin{enumarabic}
    \setcounter{enumi}{2}
    \item $F\sse A\times A$ and
    \item $(\forall x,y\in A)(x\mathrel{F}y \iff (\exists Z\in\mc P)x\in Z\land y\in Z)$.
\end{enumarabic}
What we want to prove is
\begin{enumarabic}
    \setcounter{enumi}{4}
    \item $F\sse A\times A$,
    \item $(\forall x\in A)x\mathrel F x$,
    \item $(\forall x, y\in A)x\mathrel F y \implies y \mathrel F x$, and
    \item $(\forall x,y,z\in A)(x\mathrel F y\land y\mathrel F z)\implies x\mathrel F z$.
\end{enumarabic}
The proof of (5) is trivial since (5) is (3). Let us prove (8).
\begin{enumarabic}
    \setcounter{enumi}{8}
    \item Let $a$, $b$, $c$ be given in $A$.
    \item \quad Assume $a\mathrel F b$ and $b\mathrel F c$.
    \item \quad\quad $a\mathrel F b\iff (\exists Z\in\mc P)a\in Z\land b\in Z$ \hfill by (4)
    \item \quad\quad $(\exists Z\in\mc P)a\in Z\land b\in Z$ \hfill by (10) and (11)
    \item \quad\quad Choose $D\in\mc P$ so that $a\in D\land b\in D$ \hfill by (12)
    \item \quad\quad Choose $E\in\mc P$ so that $b\in E\land c\in E$ \hfill Similar to (11), (12), (13)
    \item \quad\quad $(\exists Y\in\mc P)(b\in Y\land (\forall Z\in\mc P)b\in Z\implies Z=Y)$ \hfill by (2)
    \item \quad\quad Choose $G\in\mc P$ so that $b\in G$ and $(\forall Z\in\mc P)b\in Z\implies Z=G$ \hfill by (15)
    \item \quad\quad $b\in D \implies D=G$ \hfill by (16)
    \item \quad\quad $D=G$ \hfill by (13) and (17)
    \item \quad\quad $E=G$ \hfill Similar to the derivation of (18)
    \item \quad\quad $c\in D$ \hfill by (18), (19), and (14)
    \item \quad\quad $(\exists Z\in\mc P)a\in Z\land c\in Z$ \hfill by (20) and (13)
    \item \quad\quad $a\mathrel F c \iff (\exists Z\in\mc P) a\in Z\land c\in Z$ \hfill by (4)
    \item \quad\quad $a\mathrel F c$ \hfill by (22) and (21)
    \item \quad $a\mathrel F b\land b\mathrel F c \implies a\mathrel F c$ \hfill by (10) through (23)
    \item $(\forall x,y,z\in A)(x\mathrel F y\land y\mathrel F z)\implies x\mathrel F z$ \hfill by (9) through (24)
\end{enumarabic}
Now let us prove (6).
\begin{enumarabic}
    \setcounter{enumi}{25}
    \item Let $a$ be given in $A$.
    \item \quad $a\mathrel F a \iff (\exists Z\in\mc P)a\in Z\land a\in Z)$ \hfill by (4)
    \item \quad $(\exists Y\in\mc P)(a\in Y\land (\forall Z\in\mc P)a\in Z\implies Z=Y)$ \hfill by (2)
    \item \quad Choose $G\in\mc P$ so that $a\in G\land (\forall Z\in\mc P)a\in Z\implies Z=G)$ \hfill by (28)
    \item \quad $(\exists Z\in\mc P)a\in Z\land a\in Z)$ \hfill by (29)
    \item \quad $a\mathrel F a$ \hfill by (30) and (27)
    \item $(\forall x\in A)x\mathrel F x$ \hfill by (26) through (31)
\end{enumarabic}
Now let us prove (7).
\begin{enumarabic}
    \setcounter{enumi}{32}
    \item Let $a,b$ be given in $A$.
    \item \quad Assume $a\mathrel F b$.
    \item \quad\quad $a\mathrel F b\iff (\exists Z\in\mc P)a\in Z\land b\in Z$ \hfill by (4)
    \item \quad\quad $b\mathrel F a\iff (\exists Z\in\mc P)b\in Z\land a\in Z$ \hfill by (4)
    \item \quad\quad $(\exists Z\in\mc P)a\in Z\land b\in Z$ \hfill by (34) and (35)
    \item \quad\quad Choose $G\in\mc P$ so that $a\in G\land b\in G$ \hfill by (37)
    \item \quad\quad $(\exists Z\in\mc P)b\in Z\land a\in Z$ \hfill by (38)
    \item \quad\quad $b\mathrel F a$ \hfill by (39) and (36)
    \item \quad $a\mathrel F b \implies b\mathrel F a$ \hfill by (34) through (40)
    \item $(\forall x, y\in A)x\mathrel F y \implies y \mathrel F x$ \hfill by (33) through (41)
\end{enumarabic}
\end{solution}

\begin{exercise}
Choose a definite integer $m$. For integers $a$ and $b$ let ``$a\equiv b$'' mean
``$a - b$ is divisible by $m$ (that is, there is an integer $x$ such that $a-b=mx$)''.
Prove that $\equiv$ is an equivalence relation. How many equivalence classes are there?
\end{exercise}

\begin{solution}
Since $a-a=0m$, we have $a\equiv a$. Thus $\equiv$ is reflexive.

Suppose $a\equiv b$. Then $a-b=xm$ for some integer $x$. But then $b-a=(-x)m$,
so $b\equiv a$. Thus $\equiv$ is symmetric.

Suppose $a\equiv b$ and $b\equiv c$. Then there exist integers $x_1,x_2$ such that
$a-b=x_1m$ and $b-c=x_2m$.
Then $a-c=(a-b)+(b-c)=x_1m+x_2m=(x_1+x_2)m$, so $a\equiv c$. Thus $\equiv$ is transitive.

There are $\abs{m}$ equivalence classes if $m\neq 0$. If $m=0$, then each integer is
the sole element of its equivalence class.
\end{solution}
\section{Factoring functions}

\begin{exercise}
Suppose that $f$ is a function with domain $A$. Prove that
\[\setb{\aprn{x,y}}{f(x)=f(y)}\]
is an equivalence relation in $A$. Let $\phi$ be the corresponding quotient map.
If $f = g\circ \phi$,
as in 5-2.1, prove that $g$ is injective.
\end{exercise}

\begin{solution}
Let $E=\setb{\aprn{x,y}}{f(x)=f(y)}$. Let $a,b,c\in A$ be given. Then $f(a)=f(a)$, so $a\Erel a$ and $E$
is reflexive.
If $a\Erel b$, then $f(a)=f(b)$. But then $f(b)=f(a)$ and so $b\Erel a$. Thus $E$ is symmetric.
Finally, if $a\Erel b$ and $b\Erel c$, then $f(a)=f(b)=f(c)$, so $a\Erel c$. Thus $E$ is transitive.

Suppose $f=g\circ \phi$ as in 5-2.1. Assume $g(q_1)=g(q_2)$, where $q_1$ and $q_2$ are equivalence
classes of $E$. Then $q_1=\phi(a_1)$ and $q_2=\phi(a_2)$ for some $a_1,a_2\in A$.
Then $f(a_1)=g(\phi(a_1))=g(\phi(a_2))=f(a_2)$, which means $a_1\Erel a_2$.
Then by Proposition 5-1.3 we have $q_1=\phi(a_1)=\phi(a_2)=q_2$. Therefore $g$ is injective.
\end{solution}

\begin{exercise}
Let $D$ and $E$ be equivalence relations in the sets $A$ and $B$, respectively.
Let the corresponding quotient maps and sets be $\phi$, $\psi$, $\mc Q$, and $\mc R$.
\begin{enumalpha}
    \item Define an equivalence relation $F$ in $A\times B$ in terms of $D$ and $E$ so
        that the corresponding quotient set $\mc S$ has a natural bijection to $\mc Q\times\mc R$.
    \item Suppose that $g$ is a function from $A\times B$ to a set $T$ such that
        \[(\forall a_1,a_2\in A)(\forall b\in B)\quad a_1\mathrel{D} a_2\implies g(a_1,b)=g(a_2,b)\]
        and
        \[(\forall a\in A)(\forall b_1,b_2\in B)\quad b_1\mathrel E b_2\implies g(a,b_1)=g(a,b_2).\]
        Prove that there exists a unique function $h$ from $\mc Q\times \mc R$ to $T$ such that
        \[(\forall a\in A)(\forall b\in B)\quad g(a,b)=h(\phi(a),\psi(b)).\]
\end{enumalpha}
\end{exercise}

\begin{solution}
\begin{enumerate}[label=(\alph*)]
    \item Define $F$ to be $\setb{\aprn{\aprn{a_1,b_1},\aprn{a_2,b_2}}}{\aprn{a_1,a_2}\in D\land \aprn{b_1,b_2}\in E}$.

    We show that $F$ is reflexive. For any $\aprn{a,b}\in A\times B$, we have $a\mathrel D a$ and $b\mathrel E b$
    since $D$ and $E$ are reflexive. Hence $\aprn{a,b}\mathrel F \aprn{a,b}$.

    We show that $F$ is symmetric. Suppose
    $\aprn{a_1,b_1}\mathrel F \aprn{a_2,b_2}$. Then $a_1\mathrel D a_2$ and $b_1\mathrel E b_2$.
    Then since $D$ and $E$ are symmetric, we have $a_2\mathrel D a_1$ and $b_2\mathrel E b_1$.
    Hence $\aprn{a_2,b_2}\mathrel F \aprn{a_1,b_1}$.

    We show that $F$ is transitive. Suppose
    $\aprn{a_1,b_1}\mathrel F \aprn{a_2,b_2}$ and $\aprn{a_2,b_2}\mathrel F \aprn{a_3,b_3}$.
    Then $a_1\mathrel D a_2$, $b_1\mathrel E b_2$, $a_2\mathrel D a_3$, and $b_2\mathrel E b_3$.
    Then since $D$ and $E$ are transitive, we have $a_1\mathrel D a_3$ and $b_1\mathrel E b_3$.
    Hence $\aprn{a_1,b_1}\mathrel F \aprn{a_3,b_3}$.

    Let $\lambda$ be the quotient map of $F$.
    Let $H=\setb{\aprn{\lambda(\aprn{a,b}),\aprn{\phi(a),\psi(b)}}}{a\in A\land b\in B}$.
    We show that $H$ is a bijection
    from $\mc S$ to $\mc Q\times \mc R$.

    First we show that $H$ is a function.
    
    Let $\aprn{\lambda(\aprn{a,b}),\aprn{\phi(a),\psi(b)}}\in H$ and
    $\aprn{\lambda(\aprn{a_2,b_2}),\aprn{\phi(a_2),\psi(b_2)}}\in H$,
    and assume $\lambda(\aprn{a,b})=\lambda(\aprn{a_2,b_2})$.
    Then since $\lambda$ is the quotient map of $F$, we have $\aprn{a,b}\mathrel F \aprn{a_2,b_2}$.
    That means $a\mathrel D a_2$ and $b\mathrel E b_2$, which means $\phi(a)=\phi(a_2)$ and $\psi(b)=\psi(b_2)$.
    Hence $\aprn{\phi(a),\psi(b)}=\aprn{\phi(a_2),\psi(b_2)}$. Thus $H$ is a function.

    The domain of $H$ is clearly $\ran\lambda=\mc S$. And the range of $H$ is clearly $\mc Q\times \mc R$, i.e.
    $H$ is surjective on $\mc Q\times \mc R$. It remains to prove that $H$ is injective.

    Suppose $H(\lambda(\aprn{a,b}))=H(\lambda(\aprn{a_2,b_2}))$, that is, $\aprn{\phi(a),\psi(b)}=\aprn{\phi(a_2),\psi(b_2)}$.
    Then $\phi(a)=\phi(a_2)$ and $\psi(b)=\psi(b_2)$.
    Then $a\mathrel D a_2$ and $b\mathrel E b_2$.
    Then $\aprn{a,b}\mathrel F \aprn{a_2,b_2}$.
    Then $\lambda(\aprn{a,b})=\lambda(\aprn{a_2,b_2})$. Hence $H$ is injective.

    \item Define $h$ from $\mc Q\times \mc R$ to $T$ by $h(\phi(a),\psi(b))=g(a,b)$.
    In symbols, \[h=\setb{\aprn{\aprn{\phi(a),\psi(b)}, g(a,b)}}{a\in A\land b\in B}.\]
    We show that $h$ is a function.
    Suppose $\aprn{\phi(a_1),\psi(b_1)} = \aprn{\phi(a_2),\psi(b_2)}$.
    Then $\phi(a_1)=\phi(a_2)$ and $\psi(b_1)=\psi(b_2)$.
    Then $a_1\mathrel D a_2$ and $b_1\mathrel E b_2$.
    Since $a_1\mathrel D a_2$, we have $g(a_1,b_1)=g(a_2,b_1)$.
    Similarly, since $b_1\mathrel E b_2$, we have $g(a_2,b_1)=g(a_2,b_2)$.
    Then $h(\phi(a_1),\psi(b_1))=g(a_1,b_1)=g(a_2,b_1)=g(a_2,b_2)=h(\phi(a_2),\psi(b_2))$.
    Thus $h$ is a function.

    This function $h$ is unique, for if $h_2$ is another such function and
    $\aprn{\phi(a),\psi(b)}\in \mc Q\times \mc R$, then
    $h_2(\phi(a),\psi(b))=g(a,b)=h(\phi(a),\psi(b))$. Thus $h_2=h$.
\end{enumerate}
\end{solution}
