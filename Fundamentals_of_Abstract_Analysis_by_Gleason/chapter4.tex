\chapter{Mathematical Configurations}
\section{Structures and configurations}

In the following exercises find the pairs of configurations which are isomorphic. For
each pair, either find an explicit bijection which effects the isomorphism or give reasons
why there is none. Assume that distinct symbols in the basic sets represent distinct
objects. (Exercise 4 is offered primarily as a puzzle.)

\begin{exercise}
\[
\begin{array}{ll}
    \text{Basic set} & \text{Structural set}\\
    \set{a,b,c,d,e} & \set{\set{a},\set{a,b},\set{b,c},\set{b,d,e}}\\
    \set{f,g,h,i,j} & \set{\set{f,g},\set{f,h},\set{g,i},\set{h,j}}\\
    \set{p,q,r,s,t} & \set{\set{p,q,r},\set{p,s},\set{p,t},\set{s}}
\end{array}
\]
\end{exercise}

\begin{solution}
Since no singleton appears in the structural set for $\set{f, g, h,i,j}$, it is not isomorphic
to either of the others.

Let us try to produce an isomorphism $\phi$ from the first configuration to the third.
The singleton $\set{a}$ must be mapped to $\set{s}$, so we have $\phi(a)=s$.
We set $\phi(b)=p$ to preserve the single common element among the non-singleton sets.
We set $\phi(c)=t$ as we must have $\set{b,c}$ mapped to $\set{p,t}$ ($\set{a,b}$ is already mapped to $\set{p,s}$).
Finally $d$ and $e$ must be mapped to $q$ and $r$, and there are two ways to do this.
So $\set{\aprn{a,s},\aprn{b,p},\aprn{c,t},\aprn{d,q},\aprn{e,r}}$ is an isomorphism from the
first configuration to the third.
\end{solution}

\begin{exercise}
\[
\begin{array}{ll}
    \text{Basic set} & \text{Structural set}\\
    \set{a,b,c,d,e} & \set{\set{a,b,c},\set{a,c,e},\set{b,c},\set{b,d}}\\
    \set{f,g,h,i,j} & \set{\set{f,g},\set{f,h,j},\set{g,h},\set{g,h,i}}\\
    \set{p,q,r,s,t} & \set{\set{p,r},\set{p,r,s},\set{p,t},\set{q,r,t}}
\end{array}
\]
\end{exercise}

\begin{solution}
Since the triplets in the first configuration intersect in a pair while those in the second
and third configurations intersect in a singleton, only the second and third configurations
can be isomorphic.

Let us try to produce an isomorphism $\phi$ from the second configuration to the third.
The two pairs have a single element in common, so we set $\phi(g)=p$ to preserve that common element.
This common element belongs to only one of the triplets, so $\set{g,h,i}$ must map to $\set{p,r,s}$.
In particular, $h$ must map to $r$ or $s$. But also, $\set{g,h}$ must map to $\set{p,t}$ or $\set{p,r}$,
so $h$ must map to $t$ or $r$. Therefore $\phi(h)=r$, and then $\phi(i)=s$.
Then $\set{f,g}$ must map to $\set{p,t}$, so $\phi(f)=t$. Finally, $\phi(j)=q$.
So $\set{\aprn{f,t},\aprn{g,p},\aprn{h,r},\aprn{i,s},\aprn{j,q}}$ is an isomorphism
from the second configuration to the third, and in fact it is the only isomorphism.
\end{solution}

\begin{exercise}
\[
\begin{array}{ll}
    \text{Basic set} & \text{Structural set}\\
    \set{a,b,c,d,e} & \set{\set{a,b,c},\set{a,c,d},\set{a,d,e},\set{b,c,d}}\\
    \set{f,g,h,i,j} & \set{\set{f,g,j},\set{f,h,j},\set{g,h,i},\set{h,i,j}}\\
    \set{p,q,r,s,t} & \set{\set{p,q,r},\set{p,q,t},\set{p,s,t},\set{q,r,s}}
\end{array}
\]
\end{exercise}

\begin{solution}
In the first configuration there is only one member of the structural set containing $e$.
In the other two configurations each member of the basic set appears as an element in
at least two members of the structural set. Therefore the first configuration is not isomorphic
to any of the others.

In the second configuration, both $h$ and $j$ appear in 3 members of the structural set.
In the third configuration, it is $p$ and $q$ that appear in 3 members of the structural set.
So we must have $h$ and $j$ mapped to $p$ and $q$, and there are two ways to do this.
Now note that $h$ and $j$ appear together in $\set{f,h,j}$ and $\set{h,i,j}$, while $p$ and $q$
appear together in $\set{p,q,r}$ and $\set{p,q,t}$.
So $f$ and $i$ must be mapped to $r$ and $t$, and there are two ways to do this.
That leaves $g$ to be mapped to $s$.

So $\set{\aprn{g,s},\aprn{f,r},\aprn{i,t},\aprn{h,p},\aprn{j,q}}$ is an isomorphism
from the second configuration to the third. Indeed, applying the bijection to
the second structural set results in the third.
\end{solution}


\begin{exercise}
\[
\begin{array}{ll}
    \text{Basic set} & \text{Structural set}\\
    \set{a,b,c,d,e,f,g,h} & \Bigl\{ \bigl\{ \set{a,b}, \set{c,d}, \set{e,f}, \set{g,h} \bigr\},\\
               & \hphantom{\Bigl\{} \bigl\{ \set{a,c}, \set{b,d}, \set{e,g}, \set{f,h} \bigr\},\\
               & \hphantom{\Bigl\{} \bigl\{ \set{a,d}, \set{b,c}, \set{e,h}, \set{f,g} \bigr\},\\
               & \hphantom{\Bigl\{} \bigl\{ \set{a,e}, \set{b,f}, \set{c,g}, \set{d,h} \bigr\},\\
               & \hphantom{\Bigl\{} \bigl\{ \set{a,f}, \set{b,h}, \set{c,e}, \set{d,g} \bigr\},\\
               & \hphantom{\Bigl\{} \bigl\{ \set{a,g}, \set{b,e}, \set{c,h}, \set{d,f} \bigr\},\\
               & \hphantom{\Bigl\{} \bigl\{ \set{a,h}, \set{b,g}, \set{c,f}, \set{d,e} \bigr\} \Bigr\}\\
    \set{s,t,u,v,w,x,y,z} & \Bigl\{ \bigl\{ \set{s,t}, \set{u,w}, \set{v,y}, \set{x,z} \bigr\},\\
               & \hphantom{\Bigl\{} \bigl\{ \set{s,u}, \set{t,w}, \set{v,x}, \set{y,z} \bigr\},\\
               & \hphantom{\Bigl\{} \bigl\{ \set{s,v}, \set{t,y}, \set{u,x}, \set{w,z} \bigr\},\\
               & \hphantom{\Bigl\{} \bigl\{ \set{s,w}, \set{t,x}, \set{u,y}, \set{v,z} \bigr\},\\
               & \hphantom{\Bigl\{} \bigl\{ \set{s,x}, \set{t,z}, \set{u,v}, \set{w,y} \bigr\},\\
               & \hphantom{\Bigl\{} \bigl\{ \set{s,y}, \set{t,v}, \set{u,z}, \set{w,x} \bigr\},\\
               & \hphantom{\Bigl\{} \bigl\{ \set{s,z}, \set{t,u}, \set{u,w}, \set{x,y} \bigr\} \Bigr\}\\
\end{array}
\]
\end{exercise}

\begin{solution}
They are isomorphic. There are 64 different isomorphisms. Eight contain the pair
$\aprn{a, s}$, eight contain $\aprn{a, t}$, etc.
One is $\set{\aprn{a,s},\aprn{b,t},\aprn{c,y},\aprn{d,v},\aprn{e,u},\aprn{f,w},\aprn{g,z},\aprn{h,x}}$.
\end{solution}

\section{Definitions, postulates, and theorems}
\begin{exercise}
Find all ordered sets $\aprn{A,B}$, where $A = \set{1, 2, 3}$.
\end{exercise}

\begin{solution}
There are 19 possible values for $B$ which make $\aprn{A,B}$ an ordered set.
One is the empty set.
Six are $\set{\aprn{a,b}}$ where $a$ and $b$ are distinct.
Six are $\set{\aprn{a,b},\aprn{b,c},\aprn{a,c}}$ where $a$, $b$, and $c$ are distinct.
Three are $\set{\aprn{a,b},\aprn{a,c}}$ where $a$, $b$, and $c$ are distinct.
Three are $\set{\aprn{a,c},\aprn{b,c}}$ where $a$, $b$, and $c$ are distinct.
\end{solution}

\begin{exercise}
A \textit{Steiner triple system} is a configuration $\aprn{A, B}$ such that $B\in\power^2(A)$ and
\begin{enumerate}[label=(\alph*)]
    \item every member of $B$ has three members;
    \item every two-element subset of $A$ is a subset of exactly one member of $B$.
\end{enumerate}
Find a Steiner triple system in which $A$ has seven members.
\end{exercise}

\begin{solution}
Let $A=\set{a,b,c,d,e,f,g}$.
Let us try to construct a Steiner triple system $B$ piece by piece.
First we add the triplet $\set{a,b,c}$ to $B$.
This contains the two-element subsets $\set{a,b}$ and $\set{a,c}$ of which $a$ is a member.
Next we add $\set{a,d,e}$ and $\set{a,f,g}$ to $B$ handle the rest of the two-element subsets
of which $a$ is a member.

We have so far $B=\set{\set{a,b,c}, \set{a,d,e}, \set{a,f,g}}$.

Moving on to $b$, we still need the two-element subsets $\set{b,d}$, $\set{b,e}$, $\set{b,f}$, and $\set{b,g}$
included in exactly one triplet.
We cannot use $\set{b,d,a}$ or $\set{b,d,c}$ because $\set{a,b,c}$ is already in $B$.
We cannot use $\set{b,d,e}$ since $\set{d,e}$ is already included in $\set{a,d,e}$.
Let us add the triplets $\set{b,d,f}$ and $\set{b,e,g}$ to $B$.

We have so far $B=\set{\set{a,b,c}, \set{a,d,e}, \set{a,f,g}, \set{b,d,f}, \set{b,e,g}}$.

Moving on to $c$, we still need the two-element subsets $\set{c,d}$, $\set{c,e}$,
$\set{c,f}$, and $\set{c,g}$ included in exactly one triplet.
We cannot use $\set{c,d,a}$ or $\set{c,d,b}$ because $\set{a,b,c}$ is already in $B$.
We cannot use $\set{c,d,e}$ since $\set{d,e}$ is already included in $\set{a,d,e}$.
We also cannot use $\set{c,d,f}$ since $\set{d,f}$ is already included in $\set{b,d,f}$.
Let us add the triplets $\set{c,d,g}$ and $\set{c,e,f}$ to $B$.

We have so far $B=\set{\set{a,b,c}, \set{a,d,e}, \set{a,f,g}, \set{b,d,f}, \set{b,e,g}, \set{c,d,g}, \set{c,e,f}}$.

We have now included all two-element subsets with $a$, two-element subsets with $b$, and two-element subsets with $c$
in exactly one triplet. One can verify the two-element subsets with $d$, $e$, $f$, and $g$, respectively,
are also included in exactly one triplet. Hence
\[B=\set{\set{a,b,c}, \set{a,d,e}, \set{a,f,g}, \set{b,e,f}, \set{b,g,d}, \set{c,d,f}, \set{c,e,g}}\]
is a Steiner triple system for $A$.
In fact, every Steiner triple system with 7 elements in its basic set is isomorphic to $\aprn{A, B}$.
\end{solution}

\begin{exercise}
Either of the configurations of Exercise 4, p. 59 might appropriately be called a
schedule for a round-robin tournament. Find an appropriate set of postulates for the
class of schedules for round-robin tournaments.
\end{exercise}

\begin{solution}
Let $\aprn{A,\mc T}$ be a configuration where $\mc T \in \power^3(A)$. Think of members of T as days,
members of a day as matches, and members of $A$ as players. We write $\exists 1$ for the quantifier
\textit{There exists exactly one}, which can be expressed with the usual quantifiers as we noted
on page 18. The postulates for a round-robin tournament are as follows.
\begin{enumerate}[label=(\roman*)]
    \item Every day consists of matches only, and matches are between two players.\\
        $(\forall D\in\mc T)(\forall m\in D)(\exists a,b\in A)a\neq b\land m=\set{a,b}$.
    \item Every two distinct players play exactly one match.\\
        $(\forall a,b\in A)a\neq b\implies(\exists 1\thinspace D\in\mc T)\set{a,b}\in D$.
    \item Each player plays exactly one match on each day.\\
        $(\forall a\in A)(\forall D\in\mc T)(\exists 1\thinspace b\in A)\set{a,b}\in D$.
\end{enumerate}
\end{solution}

\begin{exercise}
Using the fundamental relation, ``$x$ is a parent of $y$'', try to write a system of
postulates for genealogy which reflects the actual biological facts. (This is not easy.)
\end{exercise}

\begin{solution}
The obvious first postulate is that every person has exactly two parents, one a man
and the other a woman. To rule out the possibility that anyone is his own parent, his own
grandparent, his own great-grandparent, etc., we need a much more sophisticated postulate.
If we attempt this directly with quantifiers having domain P, we will need infinitely
many postulates. However, this difficulty can be avoided by considering that people can
be ordered by age. This leads to the following definition.

A genealogy is a configuration $\aprn{P,\aprn{M, W,\phi}}$ where
\begin{enumerate}[label=(\roman*)]
    \item $M$ and $W$ are complementary subsets of $P$.
    \item $\phi$ is a subset of $P\times P$. (We shall write $\phi(p,q)$ instead of $\aprn{p,q}\in\phi$.)
    \item $(\forall p\in P)(\exists m\in M)(\exists w\in W)$
        \[\phi(m, p),\quad \phi(w,p),\quad\text{and}\quad (\forall q\in P)\phi(q,p)\implies(q=m\lor q=w).\]
    \item There exists an ordering of the set $P$ (i.e., a subset $B$ of $P \times P$ such that $\aprn{P,B}$
        is an ordered set in the sense of page 59) such that $(\forall p,q\in P)\phi(p,q)\implies\aprn{p,q}\in B$
        (more concisely, $\phi\sse B$).
\end{enumerate}
Note that the last postulate involves quantification over $\power(P\times P)$ instead of merely
over $P$.
\end{solution}
\section{Consistency}

\section{The classification problem}
\begin{exercise}
Show that if we adjoin the postulate,
\[\text{$A$ has seven members},\]
to those for a Steiner triple system (Exercise 2, p. 62), the resulting postulate system is
categorical.
\end{exercise}

\begin{solution}
We can letter the members of some triple in the system $\set{a, b, c}$. Choose a fourth
element and call it $d$. The system must contain triples
$\set{a,d,\enspace}$, $\set{b,d,\enspace}$, and $\set{c,d,\enspace}$.
The blanks must be filled with new names, and we can take them as $e$, $f$, and $g$, respectively.
Since there are only seven elements altogether and $a$ must be involved in a triple
with $f$ and a triple with $g$, we must have the triple $\set{a, f, g}$. Similarly, the triples $\set{b,e,g}$
and $\set{c,e,f}$ are forced. Thus we see that every Steiner triple system with seven members
is isomorphic to the system
\[\aprn{\set{a,b,c,d,e,f,g},\set{\set{a,b,c},\set{a,d,e},\set{b,d,f},\set{c,d,g},\set{a,f,g},\set{b,e,g},\set{c,e,f}}}.\]
\end{solution}

\begin{exercise}
Show that if we adjoin the postulate,
\[\text{$A$ has six members},\]
to those for a schedule for a round-robin tournament (Exercise 3, p. 62), the resulting
postulate system is categorical.
\end{exercise}

\begin{solution}
Choose any two days.
The players can be so lettered that these days are $\set{\set{a, b},\set{c, d},\set{e, f}}$
and $\set{\set{a, c},\set{b, e},\set{d, f}}$.
Some third day must contain the match
$\set{a, d}$, and on this day $e$ must play against $c$ since he cannot play $a$ or $d$ and
he has already played $b$ and $f$.
This day must be $\set{\set{a,d},\set{c,e},\set{b,f}}$.
Similarly, the days containing
the matches $\set{a, e}$ and $\set{a, f}$ are uniquely determined.
Thus all five days are determined.
It is important to check that the resulting set of five days is indeed a round-robin
tournament. A round-robin tournament can be found for any even number of players.
\end{solution}

\begin{exercise}
Classify configurations $\aprn{A, R}$, where $A$ has two elements and $R$ is a binary relation
in $A$.
\end{exercise}

\begin{solution}
If $A=\set{a,b}$ has 2 elements, then $A\times A$ has 4 elements and there are $2^4=16$ subsets
of $A\times A$. So there are 16 possible configurations. They fall into 10 classes under isomorphism.

One is the empty set.
Two are isomorphic to $\set{\aprn{a,a}}$.
Two are isomorphic to $\set{\aprn{a,b}}$.
One is $\set{\aprn{a,b},\aprn{b,a}}$.
One is $\set{\aprn{a,a},\aprn{b,b}}$.
Two are isomorphic to $\set{\aprn{a,a},\aprn{b,b},\set{a,b}}$.
Two are isomorphic to $\set{\aprn{a,a},\aprn{a,b}}$.
Two are isomorphic to $\set{\aprn{a,a},\aprn{b,a}}$.
Two are isomorphic to $\set{\aprn{a,a},\aprn{b,a},\aprn{a,b}}$.
One is $A\times A$.
\end{solution}

\begin{exercise}
Classify configurations $\aprn{A,B}$, where $B \in\power^2(A)$ and
\begin{enumerate}[label=(\alph*)]
    \item $A$ has five members,
    \item $B$ has three members,
    \item every member of $B$ has three members.
\end{enumerate}
\end{exercise}

\begin{solution}
For a fixed basic set $\set{1,2,3,4,5}$, there are 120 structures satisfying the given conditions.
Of these
\begin{center}
    20 are isomorphic to $\set{\set{1,2,3},\set{2,3,4},\set{3,4,1}}$,\\
    60 are isomorphic to $\set{\set{1,2,3},\set{2,3,4},\set{3,4,5}}$,\\
    30 are isomorphic to $\set{\set{1,2,3},\set{3,4,5},\set{1,2,4}}$,\\
    10 are isomorphic to $\set{\set{1,2,3},\set{1,2,4},\set{1,2,5}}$.
\end{center}
\end{solution}
